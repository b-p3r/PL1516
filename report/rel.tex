\documentclass[pdftex,12pt,a4paper]{report}

\usepackage[dvipsnames]{xcolor}
\usepackage[pdftex]{graphicx}
\usepackage{float}
\usepackage{fancyvrb}
\fvset{xleftmargin=2em}

\usepackage{pgfplots}
\pgfplotsset{width=10cm,compat=1.9}
\usepackage{tikzscale}
\usepackage{pgfplotstable}
\usepackage{booktabs}
\usepackage[font=small,labelfont=bf,tableposition=top]{caption}

\usepackage[utf8]{inputenc}
\usepackage[portuges]{babel}
\usepackage[T1]{fontenc}
\usepackage{times}
%\usepackage{lmodern}
\usepackage[obeyspaces,spaces]{url}
\usepackage[left=20mm,right=20mm,top=25mm,bottom=25mm]{geometry}
\usepackage{titlesec}
\usepackage{mathtools}
\usepackage{amsfonts}

%identa 1º paragrafo de capitulos e secções
\usepackage{indentfirst}
\usepackage{url}
%\usepackage{alltt}
\usepackage[]{hyperref}
\usepackage{xspace}

\hypersetup{
%pdftitle={Trabalho 1 - Gestão de Projeto},
%pdfauthor={Bruno Pereira},
%pdfsubject={Investigação Operacional},
%pdfkeywords={keyword1, keyword2}},
bookmarksnumbered=true,     
bookmarksopen=true,         
bookmarksopenlevel=1,       
colorlinks=true,            
pdfstartview=Fit,           
pdfpagemode=UseOutlines, % this is the option you were lookin for
pdfpagelayout=TwoPageRight
		}

\usepackage{minted}
\usemintedstyle{borland}
\setminted{frame=lines,
framesep=2mm,
baselinestretch=1.2,
fontsize=\footnotesize,
linenos, 
breaklines,
breakautoindent=false,
autogobble
}
\usepackage{caption}
\newenvironment{longlisting}{\captionsetup{type=listing}}{}
\def\darius{\textsf{Darius}\xspace}

\def\java{\texttt{Java}\xspace}

\def\pe{\emph{Publicação Eletrónica}\xspace}

\newcommand{\HRule}{\rule{\linewidth}{0.5mm}}


\begin{document}

\begin{titlepage}


\begin{minipage}{0.3\textwidth}
\begin{flushleft} 
\includegraphics[width=1.1\textwidth]{./report/logo.png}
\end{flushleft}
\end{minipage}
\hfill
\begin{minipage}{0.6\textwidth}
\begin{flushright} 

\fontfamily{pag}\selectfont{\large \textsc{Departamento de Informátoca}\\[0.1cm]
\large \bfseries Mestrado Integrado em Engenharia Informática \\ [0.1cm]
\large \bfseries \textit{Processamento de Linguagens}\\[4mm]
}
\noindent\rule{\textwidth}{0.7mm}
\end{flushright}
\end{minipage}\\[1cm]


\vspace{3cm}


\begin{center}

\fontfamily{pag}\selectfont{\textsc{\Huge Trabalho Prático nº 1}\\[1cm]


{\large \bfseries \emph{Filtros de Texto} \\[2cm] }



\vspace{3cm}

\begin{minipage}{0.4\textwidth}
\begin{flushright} \large
	Bruno Pereira\\
\textbf{Aluno nº 72628} 
\end{flushright}
\end{minipage}


\vfill

\emph{\large Braga, {\large \today}}
}
\end{center}

\end{titlepage}

\begin{abstract}
Isto é um resumo do relatório de \pe focando o contexto do trb (muito
sucinto),os objetivos concretos e os resultados atingidos.

Algum texto curto mas que entusiasme a leitura do relatório de \pe.

\end{abstract}

\tableofcontents

\chapter*{Introducao} \label{intro}

\begin{description}
  \item [Enquadramento] \textbf{bla bla} bla bla
  \item [Conteudo documento] \textsf{ble ble} \texttt{ble} ble
  \item [Resultados -- pontos a evidenciar] \textit{bli bli bli bli}
  \item [Estrutura do documento] \underline{blo blo blo}
\end{description}

Letras gregas sao estas $ \alpha \beta \gamma \delta $ que aqui demonstro

Exemplo simples de fracao \[ \frac{\frac{a * b + c}{4-3}}{3*5} \] simples

\section*{Estrutura do Relatorio} 


Explicar como esta? organizado o documento, referindo os capitulos existentes
em~\cite{yu09} e a sua articulacao explicando o conteudo de cada um.  No
capitulo \ref{ae} faz-se uma anA?lise detalhada do problema proposto de modo
a poder-se especificar  as entradas, resultados e formas de transformaA?A?o.\\
etc. \ldots


No capitulo~\ref{concl} termina-se o relatorio com uma sintese do que foi dito,
as conclusoes e o trabalho futuro

1     Objectivos e Organizacao
Este trabalho pratico tem como principais objectivos:

    o aumentar a experiencia de uso do ambiente Linux, da linguagem imperativa C (para codificacao das estruturas
      de dados e respectivos algoritmos de manipulacao), e de algumas ferramentas de apoio a programacao;
    o aumentar a capacidade de escrever Expressoes Regulares (ER) para descricao de padroes de frases;
    o desenvolver, a partir de ERs, sistematica e automaticamente Processadores de Linguagens Regulares, que filtrem
      ou transformem textos;
    o utilizar geradores de filtros de texto, como o Flex

Para o efeito, esta folha contem varios enunciados, dos quais devera resolver pelo menos um.



\chapter{Análise e Especificação}
\label{ae}
\section{Descrição Informal do Problema}



\section{Especificação dos Requisitos}

\subsection{Dados}

\subsection{Pedidos}

\subsection{Relações}



2     Enunciados
Para sistematizar o trabalho que se pede em cada uma das propostas seguintes, considere que deve, em qualquer um
dos casos, realizar a seguinte lista de tarefas:

    1. Especificar os padrões de frases que quer encontrar no texto-fonte, através de ERs.
    2. Identificar as acções semânticas a realizar como reacção ao reconhecimento de cada um desses padrões.
    3. Identificar as Estruturas de Dados globais que possa eventualmente precisar para armazenar temporariamente a
       informação que vai extraindo do texto-fonte ou que vai construindo à medida que o processamento avança.
    4. Desenvolver um Filtro de Texto para fazer o reconhecimento dos padrões identificados e proceder à transformação
       pretendida, com recurso ao Gerador Flex.

2.2    Normalizador de ficheiros BibTeX
BibTeX é uma ferramenta de formatação de citações bibliográficas em documentos LATEX, criada com o objectivo de
facilitar a separação da base de dados com a bibliografia consultada da sua apresentação no fim do documento LATEX
em edição. BibTeX foi criada por Oren Patashnik e Leslie Lamport em 1985, tendo cada entrada nessa base de dados
textual o aspecto que se ilustra a seguir:
@InProceedings{CPBFH07e,
  author =    {Daniela da Cruz and Maria Joao Varanda Pereira
               and Mario Beron and Ruben Fonseca and
               Pedro Rangel Henriques},
  title =     {Comparing Generators for Language-based Tools},
  booktitle = {Proceedings of the 1.st Conference on Compiler
               Related Technologies and Applications, CoRTA’07
               --- Universidade da Beira Interior, Portugal},
  year =      {2007},
  editor =    {},
  month =     {Jul},
}

De modo a familiarizar-se com o formato do BibTeX poderá consultar o ficheiro lp.bib disponı́vel em http://www.
di.uminho.pt/~prh/lp.bib e ainda a página oficial do formato referido (http://www.bibtex.org/), devendo para
já saber que a primeira palavra (logo a seguir ao caracter ”@”) designa a categoria da referência (havendo em BibTeX
pelo menos 14 diferentes).
As tarefas que deverá executar neste trabalho prático são:

a) Analise o documento BibTeX referido acima e faça a contagem das categorias (phDThesis, Misc, InProceeding,
    etc.), que ocorrem no documento. No final, deverá produzir um documento em formato HTML com o nome das
    categorias encontradas e respectivas contagens.
b) Desenvolva uma ferramenta de normalização (sempre que um campo está entre aspas, troque para chavetas e
    escreva o nome dos autores no formato ”N. Apelido”) e faça uma ferramenta de pretty-printing que indente
    corretamente cada campo, escreva um autor por linha e coloque sempre no inı́cio os campos autor e tı́tulo.
c) Construa um Grafo que mostre, para um dado autor (escolhido pelo utilizador) todos os autores que publicam
     normalmente com o autor em causa.
     Recorrendo à linguagem Dot do GraphViz2 , gere um ficheiro com esse grafo de modo a que possa, posteriormente,
     usar uma das ferramentas que processam Dot 3 para desenhar o dito grafo de associações de autores.

2.3    From treebanks to probablilistic grammar


\chapter{Concepção/Desenho da Resolução}
\section{Estruturas de Dados}
\section{Algoritmos}



\chapter{Codificação e Testes}

\section{Alternativas, Decisões e Problemas de Implementação}

\section{Testes realizados e Resultados}
Mostram-se a seguir alguns testes feitos (valores introduzidos) e
os respetivos resultados obtidos:

%\VerbatimInput{teste1.txt}




\chapter*{Conclusao}
\label{concl}

Sintese do Documento.

Estado final do projeto; Analise critica dos resultados.

Trabalho futuro.



\appendix

\chapter{Código do Programa}



Lista-se a seguir o código \java~\cite{citeulike:4677363} do programa
\darius~\cite{FCH11a} que foi desenvolvido.

\begin{verbatim}

public class Aula()

  {

    int n, m;

    int max(int a, int b)

      {

       ......

       return(max);

      }

  }

\end{verbatim}



\begin{verbatim}

llll sanjdb c kjnjcnjnjj mmmmmmmmmmmmm hhhhhhhhhhhhhhhhhhhhhhhh
jjjjjjjjjjjjjjjjjjjjjjjjjjjj kkkkkkkkkkkkkkkkkk

      aqui deve aparecer o código do programa,

      tal como está formato no ficheiro-fonte "darius.java"

      caso indesejável $\varepsilon$

\end{verbatim}



\begin{minted}{c}

#include <stdio.h>
#define N 10
/* Block
 * comment */

 int main()
 {
     int i;
	 
	   // Line comment.
		 puts("Hello world!");
			     
		 for (i = 0; i < N; i++)
		 {
		 puts("LaTeX is also great for programmers!");
		 }
										 
	   return 0;
				
}
\end{minted}


E ainda possível importar diretamente o ficheiro:


\begin{longlisting}
	\inputminted{c}{Exercicio2/exe2_1.l}
	\caption{Ficheiro fonte do exercício 2.1}
	\label{listing:3}
\end{longlisting}





\bibliographystyle{alpha}

\bibliography{./report/bibs/pl}

\end {document}


