\documentclass[pdftex,12pt,a4paper]{report}
\usepackage[square,numbers]{natbib}
\bibliographystyle{abbrvnat}
\usepackage[nottoc]{tocbibind}
\setcounter{tocdepth}{4}
\setcounter{secnumdepth}{4}
\usepackage[dvipsnames]{xcolor}
\usepackage[pdftex]{graphicx}
\usepackage{float}
\usepackage{fancyvrb}
\usepackage{dtklogos}
\fvset{xleftmargin=2em}

\usepackage{pgfplots}
\pgfplotsset{width=10cm,compat=1.9}
\usepackage{tikzscale}
\usepackage{pgfplotstable}
\usepackage{booktabs}
\usepackage[font=small,labelfont=bf,tableposition=top]{caption}

\usepackage[utf8]{inputenc}
\usepackage[portuges]{babel}
\usepackage[T1]{fontenc}
\usepackage{times}
%\usepackage{lmodern}
\usepackage[obeyspaces,spaces]{url}
\usepackage[left=20mm,right=20mm,top=25mm,bottom=25mm]{geometry}
\usepackage{titlesec}
\usepackage{mathtools}
\usepackage{amsfonts}
\usepackage{hologo}
%identa 1º paragrafo de capitulos e secções
\usepackage{indentfirst}
\usepackage{url}
%\usepackage{alltt}
\usepackage[]{hyperref}
\usepackage{xspace}

\hypersetup{
%pdftitle={Trabalho 1 - Gestão de Projeto},
%pdfauthor={Bruno Pereira},
%pdfsubject={Investigação Operacional},
%pdfkeywords={keyword1, keyword2}},
bookmarksnumbered=true,     
bookmarksopen=true,         
bookmarksopenlevel=1,       
colorlinks=true,            
pdfstartview=Fit,           
pdfpagemode=UseOutlines, % this is the option you were lookin for
pdfpagelayout=TwoPageRight
		}

\usepackage{minted}
\usemintedstyle{borland}
\setminted{
%frame=lines,
%framesep=2mm,
baselinestretch=1.2,
fontsize=\footnotesize,
linenos, 
breaklines,
breakautoindent=false,
autogobble
}
\usepackage{caption}

\def\BibTeX{{\rm B\kern-.05em{\sc i\kern-.025em b}\kern-.08em
    T\kern-.1667em\lower.7ex\hbox{E}\kern-.125emX}}

\newenvironment{longlisting}{\captionsetup{type=listing}}{}
\def\darius{\textsf{Darius}\xspace}

\def\java{\texttt{Java}\xspace}

\def\pe{\emph{Publicação Eletrónica}\xspace}

\newcommand{\HRule}{\rule{\linewidth}{0.5mm}}


\begin{document}

\begin{titlepage}


\begin{minipage}{0.3\textwidth}
\begin{flushleft} 
\includegraphics[width=1.1\textwidth]{./report/logo.png}
\end{flushleft}
\end{minipage}
\hfill
\begin{minipage}{0.6\textwidth}
\begin{flushright} 

\fontfamily{pag}\selectfont{\large \textsc{Departamento de Informátoca}\\[0.1cm]
\large \bfseries Mestrado Integrado em Engenharia Informática \\ [0.1cm]
\large \bfseries \textit{Processamento de Linguagens}\\[4mm]
}
\noindent\rule{\textwidth}{0.7mm}
\end{flushright}
\end{minipage}\\[1cm]


\vspace{3cm}


\begin{center}

\fontfamily{pag}\selectfont{\textsc{\Huge Trabalho Prático nº 1}\\[1cm]


{\large \bfseries \emph{Filtros de Texto} \\[2cm] }



\vspace{3cm}

\begin{minipage}{0.4\textwidth}
\begin{flushright} \large
	Bruno Pereira\\
\textbf{Aluno nº 72628} 
\end{flushright}
\end{minipage}


\vfill

\emph{\large Braga, {\large \today}}
}
\end{center}

\end{titlepage}

\begin{abstract}
	O presente documento documenta o trabalho prático em Processamento de
	Linguagens, relativamente a normalizadores de ficheiros \hologo{BibTeX}, dos
	quais um contador de entradas bibliográficas, que transforma essa contagem num
	ficheiro \emph{HTML}, uma ferramenta de normalização para nomes de autores
	e normalização de campos com aspas para chavetas, bem como uma ferramente de
	\emph{pretty-printing} para ficheiros \hologo{BibTeX}. Por último,
	é documentada uma ferramenta para criação de grafos em \emph{Dot} do
	\emph{GraphViz} dado um nome de um autor mostrar todas as publicações em
	comum, bem como respetiva densidade de publicação.



\end{abstract}

\tableofcontents

\chapter*{Introdução}
\addcontentsline{toc}{chapter}{Introdução} 
\label{intro}


O presente documento visa a documentação do processo de aprendizagem de
especificações em \emph{Flex} e criação de filtros para diversos formatos de
texto. No contexto escolhido, criações de filtros para ficheiros
\hologo{BibTeX}. Numa primeira parte, o objetivo é a criação de uma filtro para
contabilizar as entradas bibliográficas e criar um ficheiro \emph{HTML} com
o resultado. Ficheiros deste género podem crescer muito em tamanho, por vezes
uma contabilização pode ajudar a construir repositórios documentais sobre
determinado assunto. De igual modo, a necessidade de normalização de um
documento deste é importante, porque com o passar do tempo, o acrescentar novas
entradas pode criar desvios de formatação, que pode prejudicar uma pesquisa no
documento pelo nome, por exemplo. Um exemplo de um normalizador deste género
figura na segunda parte (capítulo 2) deste documento.  Além do normalizador, na
terceira parte (cap. 3), é apresentado a elaboração de um filtro para fazer um
\emph{pretty-printing} de um documento \hologo{BibTeX}.  Por último, no quarto
capítulo é apresentada uma ferramenta para criar um grafo com os coautores de
determinado autor, onde os pesos das arestas são a densidade de publicação com
cada coautor.


\section*{Metas e objetivos} 

O ambiente Linux é uma das metas deste trabalho. Existem diversas formas de
fazer determinadas coisas para fazer em Linux, onde cada ferramenta tem o seu
lugar e não faltam ferramentas.  O domínio deste sistema operativo é importante,
dado que, um programador, com conhecimento suficiente sobre este sistema
operativo, tem uma liberdade que falta a outros. De igual modo, dado que as
ações das especificações no \emph{Flex} são programadas em linguagem C, um dos
objetivos passa por refinar e aumentar o conhecimento da linguagem, bem como
aprofundar o conhecimento em estruturas mais complexas. De igual modo,
algoritmos e complexidade são serão revisitados, e a otimização será algo
importante durante a elaboração do trabalho. Pretende-se, portanto, soluções
eficientes. Por último, o grande objetivo é desenvolver a capacidade de escrita
de \emph{ER's} e entender como o funcionamento de processadores de linguagens
regulares funcionam, usando geradores de filtros de texto como o \emph{Flex}.
Deste último objetivo será testemunha este documento.  


\section*{Estrutura do Relatório} 
O relatório está organizado em 4 capítulos: o primeiro capítulo é referente
à alínea \emph{a}, o segundo e terceiros capítulos correspondem à alínea
\emph{b} e o quarto capitulo corresponde à alínea \emph{c}.  Cada capítulo
possui três secções: \emph{Análise do Problema}, \emph{Desenho e implementação
da solução} e \emph{Testes e Resultados}.  Na secção \emph{Análise do Problema}
expõe-se informalmente o problema, contendo conteúdo referentes aos dados,
relações e possíveis abordagens à solução. Na secção \emph{Desenho
e implementação da solução} descrevem-se as especificações e respetivas ações do
\emph{Flex}, \emph{START CONDITION}, estruturas de dados e algoritmos e funções
importantes. Em seguida a secção \emph{Testes e Resultados} apresenta os
resultados e acrescenta uma breve discussão sobre problemas de implementação,
alternativas e sugestões.  Note-se que neste documento, para mostrar
a extensibilidade dos testes, muitos deles estão no \emph{Apêndice}. O documento
encerra com a \emph{Conclusão} onde se avaliará a completude dos objetivos, bem
como observações aos resultados obtidos.






\chapter{Contagens de categorias de um ficheiro \hologo{BibTeX}}
\label{chap:a}

\section{Análise do Problema}
Neste primeiro problema, pretende-se analisar um documento \hologo{BibTeX} e fazer as
contagens das respetivas categorias, tais como artigos, teses de mestrado,
manuais, etc. O resultado tem que constar num ficheiro \texttt{HTML}.
\label{sec:ap:a}

\subsection{Especificação dos requisitos}
\label{sec:spec:a}
Existem pelo menos 14 tipos de entradas bibliográficas no \hologo{BibTeX},
podendo haver algumas extensões ou pacotes que possuam outras --- como por exemplo
o \textsc{Bib}\LaTeX{}.
Note-se que, a distinção entre o formato de ficheiro \hologo{BibTeX}
e o programa \hologo{BibTeX} é importante. O pacote \textsc{Bib}\LaTeX{} pode
ser usado tanto pelo programa \hologo{BibTeX} como não, uma vez que
o \emph{backend} por defeito do \textsc{Bib}\LaTeX{} --- o \emph{biber} ---
suporta o formato de ficheiro \hologo{BibTeX} (\texttt{.bib}). Assim, os
utilizadores do \textsc{Bib}\LaTeX{} podem usar o mesmo ficheiro \texttt{.bib} com
poucas alterações. Em consequência pode-se afirmar, que em cada ficheiro
\texttt{.bib}, todos os tipos de entrada começam com \emph{\@}, ora usam
o \hologo{BibTeX}, ora usem o \textsc{Bib}\LaTeX.


\subsection{Dados}

Os 14 tipos de entradas bibliográficas do \hologo{BibTeX} são:

\begin{itemize}
	\item\textbf{Article      } Um artigo de um jornal ou revista.
    
	\item\textbf{Booklet      } Um livro não publicado por uma editora, mas que
		é impresso e encadernado.
       
	\item\textbf{Book         } Um livro publicado por uma editora.
       
	\item\textbf{Conference   } O mesmo que \emph{Inproceedings}. 
	\item\textbf{Inbook       } Uma parte de um livro, o qual pode ser um capítulo (ou secção ou outro qualquer) e/ou uma série de páginas.
      
	\item\textbf{Incollection } Uma parte de um livro que tem o seu próprio título.
     
	\item\textbf{Inproceedings} Um artigo de uma coleção de \emph{papers} académicos de uma conferência.
   
	\item\textbf{Manual       } Documentação técnica. 
  
	\item\textbf{Mastersthesis} Uma tese de mestrado.
	\item\textbf{Misc         } Qualquer outro documento que não se enquadre em nenhuma
		catgoria.
       
	\item\textbf{Phdthesis    } Uma tese de doutoramento.
      
	\item\textbf{Proceedings  } Coleção de \emph{papers} académicos de uma conferência.
     
	\item\textbf{Techreport   } Um relatório publicado por uma escola ou outra instituição.
    
	\item\textbf{Unpublished  }  Um documento com um autor e título, mas não formalmente
		publicado.

\end{itemize}

Para além destes tipo de entradas existem também as entradas \texttt{@STRING},
\texttt{@PREAMBLE} e \texttt{@COMMENT}, onde a primeira serve para definir
abreviaturas para serem usadas no ficheiro \hologo{BibTeX}, a segundá define
como texto especial deve ser formatado, e a última, serve para incluir
comentários que não devem ser tidos em conta pelo \hologo{BibTeX}.



\section{Desenho e implementação da solução}
\label{sec:des:a}

\subsection{Expressões Regulares}
Antes de se iniciar a descrição, note-se que as \emph{ERs} estão ordenadas de
forma a não haver ambiguidade.


Uma entrada de um ficheiro \hologo{BibTeX} começa sempre com \texttt{@}. De
igual modo, os nomes de tipos de entrada podem ser escritos com maiúsculas ou
minúsculas, bem como podem começar por uma maiúscula, seguidas de minúsculas. Em
suma, não é \emph{case sensitive}. Assim, na especificação do \emph{Flex},
podemos definir uma entrada como um conjunto de caracteres, que começa com
\texttt{@} seguido de uma ou mais ocorrências de caracteres, maiúsculos ou
minúsculos.

Todavia, é necessário especializar a especificação, relativamente às entradas
\texttt{@STRING}, \texttt{@PREAMBLE} e \texttt{@COMMENT}. Também, para os tipos
de entrada bibliográficos é necessário especializar a expressão regular para os
tipos comuns de entradas descritos na secção anterior. A razão desta última
especialização justifica-se apenas por motivos de otimização e eficiência do
filtro, que serão respondidos nas secções seguintes. 

Assim, temos as seguintes \emph{ER's}:

\begin{itemize}
	\item Uma expressão regular para capturar ou \texttt{@STRING},
		\texttt{@PREAMBLE} e \texttt{@COMMENT}, da seguinte forma:
\begin{minted}{text}
		\@[Ss][Tt][Rr][Ii][Nn][Gg]
		\@[Pp][Rr][Ee][Aa][Mm][Bb][Ll][Ee]
		\@[Cc][Oo][Mm][Mm][Ee][Nn][Tt]
\end{minted}

A ação nestas expressões regulares é para ignorar.\footnote{Na especificação
	\emph{Flex} estão separadas.}


	\item Uma expressão regular para capturar uma entrada específica, por exemplo:
		\mint{text}|\@[Aa][Rr][Tt][Ii][Cc][Ll][Ee]| para capturar uma ocorrência de
		\texttt{ARTICLE}.

		A ação é contabilizar a ocorrência. Para a contabilização de \emph{ER's}
		deste género, usou-se uma vetor de inteiros, de tamanho 14, em que cada
		posição corresponde a um tipo de entrada.
\newpage

	\item Uma expressão regular para capturar uma entrada genérica, por exemplo:
		\mint{text}|\@[A-Za-z]+| onde o valor capturado é copiado a partir do
		caractere \texttt{@} e inserido numa tabela de
		\emph{hash}, contabilizando repetições.Especificações da tabela de
			\emph{hash} encontram-se na secção seguinte.
	\item Uma expressão regular para ignorar tudo o resto.

\end{itemize}


Por fim, existe a nuance de se fazer a travessia da
tabela de \emph{hash} se houver elementos na tabela.


\subsection{Estruturas de dados}
\label{sec:subsec:es:a}
Escolheu-se uma tabela de \emph{hash} dinâmica que usa o método de
\emph{chaining},
a \texttt{uthash}\footnote{\url{https://troydhanson.github.io/uthash/}} para
utilização neste problema.  Esta possui uma complexidade em termos de tempo
constante na adição, remoção e procura, bem como tem uma melhor gestão de
memória. No entanto, embora haja tabelas mais rápidas, utilizou-se esta tabela
por uma questão de conveniência, dada a simplicidade e ter \emph{performance}
aceitável para este problema.\footnote{\emph{Benchmarking} relativamente
a outras estruturas pode ser encontrado em
\url{http://lh3lh3.users.sourceforge.net/udb.shtml.}}


\section{Testes e Resultados}
\label{sec:ts:a}


\subsection{Resultados}
Para testar o filtro, utilizou-se o ficheiro \hologo{BibTeX} dado como exemplo
em \url{http://www4.di.uminho.pt/~prh/lp.bib}.

O resultado em \emph{HTML} consta no Apêndice~\ref{appendix:a}, na
pág.~\pageref{appendix:a}.

O resultado após ser executado por um \emph{browser} é o que se segue:

\begin{figure}[h!]
	\centering
	\includegraphics[scale=0.5]{./testes/res_html}
	\caption{Resultado visto no \emph{Firefox}}
	\label{fig:res1}
\end{figure}

A partir do documento que adveio da URL na secção \emph{Resultados}, podemos
constar que não existem entradas que difiram do \hologo{BibTeX}, a não ser
a entrada \emph{proceeding}, e que não existem \emph{Booklets}, nem
\emph{Conferences}. 


\subsection{Alternativas, Decisões e Problemas de Implementação}

Numa primeira implementação, existia uma \emph{trie} para guardar os tipos de
entrada que não pertencessem ao \hologo{BibTeX}. A travessia desta estrutura
é recursiva, embora linear no número de nodos, cada nodo podia ter um
\emph{array} de apontadores de tamanho 256, podendo esse \emph{array} ter poucas
posições ocupadas. De igual modo, a \emph{trie} que estava implementada não
estava otimizada e poderia ter \emph{bugs}. Daí a escolha passar a ser uma
tabela de \emph{hash}. Esta última estrutura, à semelhança da \emph{trie}---
ignorando o tamanho da \emph{string} que compõe a chave ---, continua a ter
tempo constante de inserção, e o \emph{overhead} é menor.

À data de redação deste relatório, chegou-se a conclusão que poder-se-ia ter
escolhido uma tabela de \emph{hash} em \emph{open adressing} ou outra estrutura
otimizada, podendo assim ter um filtro mais eficiente. Futuramente, poder-se-á
tentar utilizar uma estrutura diferente e efetuar mais testes.




\chapter{Ferramenta de normalização de um ficheiro \hologo{BibTeX}}
\label{chap:b1}



\section{Análise do Problema}
\label{sec:b1p:b1}
Para além dos tipos de entrada, é necessário especificar o conteúdo da entrada,
como nomes de autor, títulos de obra, editora, etc. O \hologo{BibTeX} possui
propriedades definidas para cada item como campo da entrada.  
Para esta parte do trabalho, é pedido o desenvolvimento de uma ferramenta de
normalização dos nomes dos autores no campo respetivo, no formato \emph{N.(ome)
Apelido}, que de igual modo normalize todos os campos entre aspas, com
chavetas. 

\subsection{Especificação dos requisitos}
\label{sec:spec:b1}

\subsection{Dados}
Os nomes dos autores podem ter muitos formatos. Como por exemplo:

\begin{itemize}
	\item \emph{Donald E. Knuth}
	\item \emph{D. E. Knuth}
	\item \emph{Knuth, Donald E.}
	\item \emph{Knuth, PhD, Donald E.}
	\item \emph{Nicollo Alighieri Franchi-Zanettachi}
	\item \emph{Daniela da Cruz}
\end{itemize}

Assim, há uma necessidade de especializar um conjunto de \emph{ER's} para tratar
cada caso, com especial atenção para os nomes no formato \emph{Apelido, Nome}.

De igual modo, temos que cada campo pode começar com uma chaveta ou aspas,
terminando de igual forma, com a chaveta fechada ou aspas correspondente.

Um conceito importante no \TeX{} em geral, é que um documento está \emph{bem
formado} se todas as chavetas abertas tiverem a chaveta fechada correspondente.
De facto, existem estilos de bibliografias que convertem o primeiro caractere
que compõe o valor do campo em maiúscula e os restantes em minusculas. Esta
funcionalidade ocorre para nomes de um título ou outro campo, que não o do
autor. Por vezes é necessário manter as maiúsculas, dado que existem valores de
campo em que, por exemplo, o primeiro caractere de cada palava está
capitalizado. O \hologo{BibTeX} permite ao utilizador abrir e fechar chavetas em
torno do conjunto de caracteres onde se pretende manter a capitalização.
A relevância deste contexto será explicada na secção seguinte.


\section{Desenho e implementação da solução}
\label{sec:des:b1}


\subsection{Expressões Regulares}

Antes de se iniciar a descrição, note-se que as \emph{ERs} estão ordenadas de
forma a não haver ambiguidade. 

\subsubsection{\emph{INITIAL \emph{START CONDITIONS}}}


Qualquer campo no \hologo{BibTeX} tem sempre o identificador do campo
(\emph{author}, \emph{title}, etc.) seguido de um '\texttt{=}', começando
o valor do campo com a abertura de aspas ou chavetas. Todavia, entre os
o identificador do campo, '\texttt{=}' e \texttt{\{}, pode não haver espaços, ou
pode haver um ou mais espaços. Como a ferramente de normalização tem duas
funcionalidades diferentes conforme os campos, é necessário ter duas
\emph{ER's}: uma que trate de tudo o que é necessário fazer com o campo autor
e outra, genérica, que trate dos restantes no contexto da normalização das
chavetas. Dado que, a captura do campo autor e do processamento das chavetas são
dois contextos diferentes, é necessário recorrer ao uso de \emph{START
CONDITIONS}. Assim, para estes últimos implementou-se a \emph{START 
CONDITION} \texttt{AUT} para tratar do autor, e a \emph{START
CONDITION} \texttt{CHAV} para tratar das chavetas dos outros campos.


Após o exposto atrás, temos duas expressões regulares, tais que:

\begin{itemize}
	\item Captura campo autor
\begin{minted}{text}
			[Aa][Uu][Tt][Hh][Oo][Rr][ ]*"="[ ]*[{"]
\end{minted}
A ação para esta expressão regular é colocar no último caractere capturado uma
chaveta aberta, imprimir no \emph{stdout} e iniciar \texttt{AUT}.
		

	\item Captura qualquer outro campo;
\begin{minted}{text}
		 [A-Za-z]+[ ]*"="[ ]*[{"]
\end{minted}
A ação para esta expressão regular é colocar no último caractere capturado uma
chaveta aberta, imprimir no \emph{stdout} e iniciar \texttt{CHAV}.
\end{itemize}


\subsubsection{\emph{AUT \emph{START CONDITION}}}

Nesta \emph{START CONDITION} faz-se a distinção dos nomes, conforme na
\emph{Secção}~\ref{sec:spec:b1}. No entanto, é necessário distinguir o que é um
nome de um autor. Um nome de uma autor pode-ser seguido de um \emph{and}, se
houver mais que um autor, e também pode ser único e terminar em aspas ou numa
chaveta. Dentro do nome, faz-se a distinção de um nome poder ser uma inicial, ou
uma palavra que inicie com maiúscula, seguida de uma ou mais letras minúsculas
--- por definição, um nome próprio ---, ou pode ser um nome composto (dois nomes
próprios separados por um hífen). De igual modo, os nomes podem conter uma
preposição dentro do nome, e podem ter vários espaços e/ou tabulações antes
e depois do nome. Acresce também a condição especial de que os nomes podem estar
no formato `Apelido, Nome' e neste caso é necessário uma novo contexto para
tratar este caso especial. Para isso, criou-se a \emph{START CONDITION}
\texttt{PREFORMAT}. 



\begin{itemize}
	\item Captura de aspas ou uma chaveta no final campo.
		\mint{text}|[}"]|
		Neste caso, como se poderá ver mais à frente, pode aparentar ser redundante.
		Todavia, por uma questão de coerência e segurança, em caso de uma captura
		para entrar nos estado da \emph{START CONDITION} \texttt{PREFORMAT}, no
		final do processamento do campo nesta condição, volta-se sempre ao estado
		anterior \texttt{AUT}. Assim, o final de campo é sempre processado no mesmo
		contexto.

		A ação nesta \emph{ER} é consumir o valor, imprimindo no \emph{stdout} uma
		chaveta, voltando à \emph{START CONDITION} \texttt{INITIAL}.

	\item Captura de 1 ou mais espaços ou tabulações.
     \mint{text}|[ \t]+|
		 Esta \emph{ER} garante que espaços ou tabulações em torno dos nomes são
		 consumidos, de forma a ter apenas os caracteres correspondentes aos nomes.
		 De igual modo, imprime um espaço no \emph{stdout}.
	
	 \item Captura de um espaço ou tabulação antes e depois de uma
		 preposição, e a própria preposição.
		\mint{text}|[ \t][a-z]+[\t ]|

		 As preposições em nomes com \emph{Daniela da Silva} são ser ignorados.
		 A ação é mesma que na \emph{ER} anterior.

	 \item Captura de uma inicial ou um nome próprio, que não apelido.
    \begin{minted}{text}
		 [A-Z]((\.)?|[a-z]+)
    \end{minted}

	A ação neste caso é obter o primeiro caractere da captura da expressão,
	e imprimir no \emph{stdout} o mesmo caractere seguido de um ponto.


	 \item Captura de um nome próprio, que é apelido, pode ser composto
		 e é o último da listagem ou único.
    \begin{minted}{text}
		((-)?[A-Z][a-z]+)+[ \t]*[}"]
    \end{minted}
 O nome pode ter ou não iniciar com um hífen, seguido do nome próprio. Estes
 dois itens podem ocorrer uma ou mais vezes. Por exemplo, no apelido do nome
 \emph{Nicollo Franchi-Zanettachi}. Neste caso o nome do autor é o último nome
 listado ou o único. Assim, espaços e tabulações antes do final da linha são
 capturados, O ou mais vezes ( apelido pode estar junto das aspas ou da chaveta
 ).

	A ação pretendida aqui é modificar o último caractere para uma chaveta,
	imprimir o resultado para o \emph{stdout} e voltar a \emph{START CONDITION}
	\texttt{INITIAL}.

	 \item Captura de um nome próprio, que é apelido, pode ser composto
		      mas não é o único na listagem de autores.
    \begin{minted}{text}
		 ((-)?[A-Z][a-z]+)+[ \t]+(and)[ \t]+
    \end{minted}

		A ação correspondente é em quase tudo semelhante à ação anterior, no
		entanto, captura todos os espaços e tabulações antes e depois do separador
		\emph{and}, e o separador \emph{and}. Neste caso imprime para o \emph{stdout}
		a captura. Assim o próximo fica livre de espaços no inicio.


	\item Captura de uma linha com a formatação \emph{Apelido, Nome}.
    \begin{minted}{text}
		 ((-)?[A-Z][a-z]+)+[,]+[ \t]
    \end{minted}
A captura deteta se existem pelo menos uma vírgula, depois do nome e que este
seja seguido de espaços. Se a formatação no primeiro nome possui vírgulas, logo
a restante está na mesma formatação. A ação é colocar o apontador de leitura do
\emph{Flex} no inicio da linha, colocar uma variável inteira para um índice de
um vetor a 0, inicializar um vetor de apontadores para \emph{strings} a 
\texttt{NULL}\footnote{Algoritmo e código será apresentado na secção seguinte}.
Em seguida inicia a \emph{START CONDITION} \texttt{PREFORMAT}.


	 \item Captura tudo o resto incluindo \emph{newline}.
    \begin{minted}{text}
		(.|\n)
    \end{minted}
	A ação é ignorar tudo.


\end{itemize}




\subsubsection{\emph{PREFORMAT \emph{START CONDITION}}}

Esta \emph{START CONDITION} tem as \emph{ER's} iguais, exceto a \emph{ER} \verb|[, \t]+| e 
o separador \emph{and} tem um tratamento diferente, bem como o final do valor do
campo. A maior parte das ações foi redefinida para para este novo contexto.




\begin{itemize}
	\item Captura de aspas ou uma chaveta no final campo.
		\mint{text}|[}"]|

		Como foi anteriormente mencionado, quando capturado ou aspas ou uma chaveta,
		obriga-se o \emph{Flex} a colocar a captura no \emph{stdin}
		(\texttt{yyless(0)}). De seguida coloca-se o índice do \emph{array} de
		\emph{strings} já mencionado, na posição inicial e executa-se uma função para
		imprimir os valores contidos no \emph{array}.\footnote{Algoritmo e código
		descrito na secção seguinte}

	\item Captura de 1 ou mais espaços, tabulações ou vírgulas.
     \mint{text}|[, \t]+|
		 A semelhança da anterior \emph{ER} para além de garantir que espaços ou
		 tabulações em torno dos nomes são consumidos, garante,de igual modo, que
		 vírgulas sejam consumidas. No entanto, não imprime um espaço. No seu lugar
		 o apontador para o índice do \emph{array} de \emph{strings} seja
		 incrementado, uma vez que, todas as nomes são rodeados por um ou mais
		 destes caracteres.
		 \emph{stdout}.
	 \item Captura de um espaço ou tabulação antes e depois de uma
		 preposição, e a própria preposição.
		\mint{text}|[ \t]+(and)+[\t ]+|
		A ação definida captura o \emph{and}, caso exista mais que um autor, após
		a ocorrência deste. Note-se que pode encontrar um ou mais mais espaços, bem
		como tabulações antes e depois deste separador. De igial modo, coloca
		o índice do apontador do \emph{array} de \emph{strings} auxiliar na posição
		inicial, imprimindo os nomes próprios do autor pela ordem desejada. Após
		este passo imprime no \emph{stdout} a \emph{string} \emph{and} com um espaço
		de cada lado, iniciando a \emph{START CONDITION} \texttt{AUT}.
		Note-se que o tratamento do conjunto de nomes é diferente da \emph{START
		CONDITION} \texttt{AUT}, uma vez que, em \texttt{AUT}, não se podia saber
		qual seria o último nome, a não ser uma implementação especializada na parte
		da ação em \emph{C}. O intuito deste projeto foi sempre usar ao máximo
		a funcionalidade do \emph{Flex}, e tentar usar ao máximo \emph{ER's}. Aqui,
		como sabemos que o último nome do autor é logo o primeiro a ser listado.,

	
	 \item Captura de um espaço ou tabulação antes e depois de uma
		 preposição, e a própria preposição.
		\mint{text}|[ \t][a-z]+[\t ]|

		 A ação é mesma que na \emph{ER} equivalente, descrita na secção anterior.

	 \item Captura de uma inicial ou um nome próprio, que não apelido.

	 \item Captura de um nome próprio, que é apelido, pode ser composto
		 e é o último da listagem ou único.
    \begin{minted}{text}
		((-)?[A-Z]((\.)|[a-z]+))
    \end{minted}
		A \emph{ER} tem o mesmo intuito que a \emph{ER} equivalente da secção
		anterior, no âmbito da captura. No entanto, a ação é diferente.
		Esta passa por comparar a posição atual do índice do \emph{array} auxiliar
		para identificar a ordem dos nomes. Assim, para cada ocorrência de um nome
		próprio de um autor, se for diferente da posição $0$, é colocado no valor de
		\texttt{yytext} na posição $1$ um um ponto e na posição seguinte,
		o caractere \verb|'\0'|, copiando esse valor para o \emph{array} de
		\emph{strings} auxiliar na posição atual do índice do \emph{array}. Caso
		contrário copia o valor inteiro da \emph{string} para a posição $0$.

	 \item Captura tudo o resto incluindo \emph{newline}.
    \begin{minted}{text}
		(.|\n)
    \end{minted}
	A ação é ignorar tudo.


\end{itemize}


\subsubsection{\emph{CHAV \emph{START CONDITION}}}

No contexto \texttt{CHAV} apenas se faz a captura das aspas ou chavetas de todos
os outros campos. As chavetas ou aspas do campo do autor já estão previstas  no
devido contexto. No entanto, aqui pode ocorrer o novo contexto, já mencionado na
secção \emph{Análise do Problema} deste capítulo. Deste modo, há a necessidade
de se ter uma nova \emph{START CONDITION} \texttt{SPEC}.


\begin{itemize}
  \item Captura uma chaveta aberta.
    \begin{minted}{text}
			[{]
    \end{minted}
		A ação consiste imprimir para o \emph{stdout} a chaveta e iniciar
		a \emph{START CONDITION} \texttt{SPEC}.

  \item Captura do fim de campo, podendo ser uma chaveta ou aspas.
    \begin{minted}{text}
		[}"]
    \end{minted}
		Ação: imprimir a chaveta de fecho, e voltar para as \emph{START CONDITION}
		\texttt{INITIAL}

  \item Captura qualquer outro campo;
    \begin{minted}{text}
			(.|\n)
    \end{minted}
\end{itemize}
Ação: imprimir tudo o resto, incluindo \emph{newlines} para \emph{stdout}.



\subsubsection{\emph{SPEC \emph{START CONDITION}}}

Há apenas acrescentar sobre esta \emph{START CONDITION}, que apenas é um
\emph{workaroud} para evitar de capturar uma chaveta de a meio do valor do
campo, e ser passível de ser considerada fim do valor de campo. 


\begin{itemize}
  \item Captura o fecho de chavetas.
    \begin{minted}{text}
		[}]
    \end{minted}
		Imprime para o \emph{stdout} a mesma chaveta.


  \item Captura qualquer outro campo;
    \begin{minted}{text}
			(.|\n)
    \end{minted}
		Imprime tudo o resto no \emph{stdout}.
	
\end{itemize}
\subsection{Algoritmos}
\begin{Verbatim}
     void print_array (char ** array)
     {
         int i;
     	
         for (i = 1; i < ARRAY_SIZE&&array[i]; i++)
             {
     						
     	            printf("%s ", array[i]);
             }
         printf("%s", array[0]);
     										
     															
     }

\end{Verbatim}

A função acima corresponde à função de impressão dos nomes próprios dos autores
na \emph{START CONDITION} \texttt{PREFORMAT}. Note-se que apenas valores
existentes no \emph{array} são imprimidos no \emph{stdout}, começando pela
segunda posição. A primeira posição é imprimida no fim.

\begin{Verbatim}
	void clean_array (char ** array)
	{
	    int i;
		
			for (i = 0; i < ARRAY_SIZE&&array[i]; 
			           free(array[i]), array[i++]=NULL);
			
			
	}

\end{Verbatim}

A função acima corresponde à função de inicialização do \emph{array} dos nomes
próprios dos autores na \emph{START CONDITION} \texttt{PREFORMAT}. Note-se que
liberta a memória e inicializa valores previamente existentes.


\section{Testes e Resultados}
\subsection{Resultados}
\label{sec:ts:b1}

O ficheiro de usado para este teste pode ser visto no
\emph{Apêndice}~\ref{appendix:a1} na pág.~\pageref{appendix:a1}. 

O resultado pode ser consultado no \emph{Apêndice}~\ref{appendix:b} na
pág.~\pageref{appendix:b}

\subsection{Alternativas, Decisões e Problemas de Implementação}

Adicionalmente à solução descrita neste capitulo, poder-se-ia ter implementado
ou mais uma \emph{START CONDITION} ou possivelmente mais algumas \emph{ER's}
que tratassem de nomes de sufixo como em  \emph{Knuth, PhD, Donald E.}.

Assumiu-se uma codificação \emph{ASCII}, pelo que não foram tratados caracteres
em \emph{UTF-8} ou \emph{ISO 8859-1}. Para tal ter-se-ia que tratar os
caracteres com o tamanho de dois \emph{bytes} e capturar sequências de escape
para caracteres especiais em determinado ficheiro \hologo{BibTeX} e guardá-los
como caracteres de dois \emph{bytes}. 








\chapter{Ferramenta \emph{pretty-printing} de um ficheiro \hologo{BibTeX}}
\label{chap:b2}

\section{Análise do Problema}
\label{sec:b2p:b2}
Nesta parte, o desafio é elaborar uma ferramenta de \emph{pretty-printing} que
indente corretamente cada campo, escreva um autor por linha e coloque sempre no
início os campos autor e título.

\section{Especificação dos requisitos}
\label{sec:spec:b2}

\subsection{Dados}

No \hologo{BibTeX}, cada tipo de entrada bibliográfica tem campos opcionais
e obrigatórios. Na resolução deste problema apenas se centrou nos obrigatórios,
dado que se quiser uma ferramenta genérica, que inclua outros pacotes, o número
de campos é elevado. De igual modo, os campos obrigatórios podem ser opcionais
para algum tipo de entrada, e vice---versa. Em extensões, como no
\textsc{Bib}\LaTeX{}, alguns campos são são comuns, por uma questão de compatibilidade.

Os campos considerados são:

\begin{itemize}


\item Organização
\item \emph{How Published}
\item Instituição
\item Publicação
\item Titulo do livro
\item Jornal
\item Edição
\item Capitulo
\item Morada
\item Volume
\item Serie
\item Escola
\item Numero
\item Editor
\item Autor
\item Titulo
\item Págs
\item Mês
\item Ano
\item Tipo

\end{itemize}

Note-se que estes campos podem ser tanto obrigatórios como opcionais.
A ordem com que os campos estão colocados não tem a obrigatoriedade de uma
sequência. Por exemplo o campo autor pode ser colocado no final da enumeração
dos campos, sem nenhum efeito colateral. O que decide a ordem dos elementos
é sempre o estilo de bibliografia.


\section{Desenho e implementação da solução}
\label{sec:des:b2}

A sugestão de solução apresentada, possui quatro \emph{START CONDITIONS}:
a \emph{SC} \texttt{ENTRY}, \emph{SC} \texttt{AUT}, a \emph{SC} \texttt{FIELD}
e, por último, a \emph{SPEC}. A necessidade de ter as primeiras \emph{SC}
deve-se à necessidade de criar um contexto para tratar os autores, bem como
o título, sendo estas ativadas dentro \emph{SC} \texttt{ENTRY}. O intuito
\emph{SC} \texttt{SPEC} já foi descrita em secções anteriores --- capturar par
de chavetas, para evitar conflitos com as outras \emph{ER's}. De igual modo,
usaram-se três variáveis globais do tipo inteiro: uma para guardar o estado das
\emph{SC} no caso da \texttt{SPEC}, dois contadores, um para um índice de uma
\emph{string}, outro para guardar o índice de um \emph{array} de \emph{strings}
multidimensional. Além, destas variáveis de valor inteiro, criaram-se, como já
mencionado, uma \emph{string} e um \emph{array} multidimensional para guardar
o resultado das ocorrências dos campos. Criou-se este último, com o intuito de
guardar nas primeiras posições os valores do campos autor e título, seguido de
tudo o resto. Assim, após a captura do fim da entrada bibliográfica, o resultado
é passado ao \emph{stdout} pela ordem correta, iniciando o índice do
\emph{array} multidimensional de cada vez que é encontrada uma entrada
bibliográfica. Também, o \emph{array} é multidimensional dado que intenciona
guardar na primeira linha a legenda do campo e na segunda o valor do campo.
A especificação de cada \emph{SC}, que implementa estes conceitos segue-se nas
seguintes secções.


\subsubsection{\emph{START CONDITION} \texttt{INITIAL}}

\begin{itemize}

	\item Captura o início de uma entrada, ou seja, um \texttt{@} seguido de uma
		ou mais caracteres maiúsculos ou minúsculos.

\begin{verbatim}
\@[A-Za-z]+
\end{verbatim}

Como já foi anteriormente mencionado, a captura deste valor através da
\emph{ER} descrita, despoleta uma ação, tal que a posição do \emph{array}
multidimensional é inciada logo na terceira posição, uma vez as duas primeiras
estão reservadas para o autor e o título. De igual modo, inicia a \emph{SC}
\texttt{ENTRY}, imprimindo no \emph{stdout} uma pequeno separador composto por
cardinais.

\item Captura tudo o resto, incluindo o caractere \emph{newline}.

\begin{verbatim}
[(.|\n)                            
\end{verbatim}

A ação é ignorar o restante.



\end{itemize}

%%%%%%%%%%%%%%%%%%%%%%%%%%%%%%%%%%%%%%


\subsubsection{\emph{START CONDITION} \texttt{ENTRY}}

A \emph{SC} \texttt{ENTRY} está no contexto da entrada bibliográfica, onde os
vários campos são capturados. 

\begin{itemize}
\item Captura chaveta correspondente ao final da entrada bibliográfica.

\begin{verbatim}
[}] 
\end{verbatim}
Aqui é invocada a função de impressão dos campos pela ordem desejada, voltando
a \emph{SC} \texttt{INITIAL}.
\item Captura uma chaveta seguida de 0 ou mais espaços, com uma vírgula
	a seguir.

\begin{verbatim}
[}][ ]*[,] {;} 
\end{verbatim}

A ação aqui definida é ignorar a captura. Este \emph{ER} serve de artifício
para não haver ambiguidade com o final dos campos, dado que a expressão
é maior. Note-se que, caso não se recorresse a este artifício, a captura dos
campos poderia terminar a meio.

\item Captura o campo \emph{Organization}, \emph{case-insensitive}, seguido 0 ou
	mais espaços, um \emph{=}, com 0 ou mais espaços de seguida, podendo terminar
	ou não numa chaveta ou aspas.
\begin{verbatim}
[Oo][Rr][Gg][Aa][Nn][Ii][Zz][Aa][Tt][Ii][Oo][Nn][ ]*"="[ ]*[{"]? 
\end{verbatim}

Após a captura , o índice da \emph{string} auxiliar para armazenar
temporariamente o valor dos caracteres capturados é inicializado, sendo guardada
a legenda do campo na primeira linha, na coluna correspondente à posição da
ocorrência. Em seguida é inicializado a \emph{SC} \texttt{FIELD}, correspondente
ao contexto de manipulação de qualquer outro campo, que não o autor e o título.
                            
\item Captura similar a anterior, exceto a aspas ou chavetas de início do valor
	do campo são obrigatórias. Neste caso a captura é do campo autor.
\begin{verbatim}
ENTRY>[Aa][Uu][Tt][Hh][Oo][Rr][ ]*"="[ ]*[{"] 
\end{verbatim}

A ação é igual à de \emph{ER} anterior, com a exceção da posição do
\emph{array} que é utilizada diretamente, e é iniciada a \emph{SC} \texttt{AUT}.



\item Captura igual à anterior, só que neste caso, a captura é do campo do
	título. A \emph{SC} iniciada é a \texttt{TITLE}.
\begin{verbatim}
<ENTRY>[Tt][Ii][Tt][Ll][Ee][ ]*"="[ ]*[{"] 
\end{verbatim}

Estas \emph{ER's} estão aqui a título exemplar, dado que existem \emph{ER's}
para cada campo obrigatório mencionado na secção \emph{Análise do Problema}.


\item Captura tudo o resto incluindo o caractere \emph{newline}, sendo a ação
	ignorar a captura.
\begin{verbatim}
(.|\n)
\end{verbatim}



\end{itemize}

%%%%%%%%%%%%%%%%%%%%%%%%%%%%%%%%%%%%%%

\subsubsection{\emph{START CONDITION} \texttt{AUT}}

\begin{itemize}
\item 
\begin{verbatim}
[ \t]+"and"[ \t]+
\end{verbatim}
 {strcpy(value+i, "\n\t\t\t "); i+=5;}

\item 
\begin{verbatim}
<AUT>[}"][ ]*[ ]? 
\end{verbatim}
                                   strcpy(value+i,"\n"); i++; 
                                   fields[1][0]=strdup(value); 
				                   BEGIN ENTRY;}

\item 
\begin{verbatim}
(.|\n)
\end{verbatim}

{value[i++]=yytext[0];}


\end{itemize}

%%%%%%%%%%%%%%%%%%%%%%%%%%%%%%%%%%%%%%
\subsubsection{\emph{START CONDITION} \texttt{TITLE}}

\begin{itemize}
\item Captura dois ou mais espaços.
\begin{verbatim}
[ ]{2,}
\end{verbatim}

Apenas serve para consumir espaços a mais, imprimindo no \emph{stdout} um espaço.

\item Captura qualquer chaveta aberta, que não a do fim de campo.
\begin{verbatim}
[{]
\end{verbatim}

A ação desta captura é devido ao que foi exposto em capítulos anteriores. No entanto, como foi anteriormente explicado, vários contextos utilizam a \emph{START CONDTION} \texttt{SPEC}. Assim, como em anteriores \emph{START CONDTIONS} o valor do estado é guardado numa varrável global, com a macro \texttt{YYSTATE}. O chaveta é consumida.

\item Captura o final do campo caso seja aspas ou o fecho de chavetas.
\begin{verbatim}
[}"][ ]*[,]? 
\end{verbatim}

\item Captura tudo o resto.
\begin{verbatim}
(.)
\end{verbatim}
{value[i++]=yytext[0];}

\item Captura do caractere \emph{newline}
\begin{verbatim}
(\n)
\end{verbatim}

A ação é não fazer nada. Apenas serve para consumir eliminar quebras de linha no campo do título.


\end{itemize}

%%%%%%%%%%%%%%%%%%%%%%%%%%%%%%%%%%%%%%
\subsubsection{\emph{START CONDITION} \texttt{FIELD}}

\begin{itemize}
\item 
\begin{verbatim}
"\\url{"
\end{verbatim}
 { state_caller = YYSTATE; BEGIN SPEC;}

\item 
\begin{verbatim}
[ ]{2,}
\end{verbatim}

\item 
\begin{verbatim}
[{]
\end{verbatim}

\item 
\begin{verbatim}
[}"]?[ ]*[,]
\end{verbatim}
strcpy(value+i,"\n"); i++;
				                   fields[1][pos++]=strdup(value);
				                   BEGIN ENTRY;}

\item 
\begin{verbatim}
(.)
\end{verbatim}
 {value[i++]=yytext[0];}

\item 
\begin{verbatim}
(\n)
\end{verbatim}

\end{itemize}
%%%%%%%%%%%%%%%%%%%%%%%%%%%%%%%%%%%%%
\subsubsection{\emph{START CONDITION} \texttt{SPEC}}

\begin{itemize}
\item 

\begin{verbatim}
[}]
\end{verbatim}
yyless(0);BEGIN state_caller;

\item 
\begin{verbatim}
(.|\n)
\end{verbatim}
{value[i++]=yytext[0];}

\end{itemize}



\subsection{Estruturas de dados}

\subsection{Algoritmos}

\begin{verbatim}
void print_campos()
{
    int j;

    for(j = 0; j < pos; j++){
        printf("%s %s", fields[0][j], fields[1][j]);
    	free(fields[0][j]);
    	free(fields[1][j]);
    	fields[0][j]=NULL;
    	fields[1][j]=NULL;
    }

}
\end{verbatim}


\begin{verbatim}
int j;

    yylex();
 for(j = 0; j < MAX_ENTRIES; j++)
        {
            if(fields[0][j])
                free(fields[0][j]);
            else if (fields[1][j])
                free(fields[1][j]);

        }

\end{verbatim}

\section{Codificação e Testes}
\label{sec:ts:b2}

\subsection{Alternativas, Decisões e Problemas de Implementação}

\subsection{Testes e Resultados}
%Mostram-se a seguir alguns testes feitos (valores introduzidos) e
%%os respetivos resultados obtidos:


\chapter{Grafo de associações de um autor em \emph{Dot} de um ficheiro \BibTeX}
\label{chap:c}

\section{Análise do Problema}
\label{sec:cp:c}
Nesta parte, requere-se que se faça o grafo de associações para um dado autor,
dos autores que usualmente publicam com ele. A linguagem a ser usada é linguagem
\emph{Dot} do \emph{GraphViz}, para renderizar o grafo, tendo que
o filtro em \emph{Flex} gerar no \emph{stdout} o grafo nessa
linguagem.   

\section{Especificação dos requisitos}
\label{sec:spec:c}

\subsection{Dados}

Em relação aos dados do problema, pouco mais há a acrescentar do que
já se mencionou em capítulos anteriores, relativamente ao
\hologo{BibTeX} --- o campo autor podes ter um ou mais autores, cada
um separado por \emph{and}, podendo o campo estar dentro de chavetas
ou aspas.

Em relação ao \emph{Dot} existem algumas considerações a ter,
nomeadamente no desenho do grafo através da linguagem. Pretende-se um
grafo, de preferência orientado, onde numa única página seja possível
mostrar todas as associações. Cada associação pode ter um peso,
correspondente à densidade de publicação e existe uma, e uma só
associação entre o autor escolhido e cada um dos seus co-autores. Além
do mais, podem ocorrer os seguintes casos: o autor tem vários
co-autores, o autor nunca publicou com ninguém, e o autor escolhido
não existe. Acresce-se que o grafo é sempre produzido, estando vazio
no caso de o autor não existir ou haver apenas um nodo com o nome do
autor escolhido, no caso de autor ter sempre publicado sozinho.

\subsection{Relações}

Relativamente às associações, existem relações subjacentes que
é preciso interpretar. Pretendendo-se todas as associações entre cada autor que
ocorre no campo homólogo no ficheiro do \hologo{BibTeX}, é necessário
construir um grafo com todas as associações entre todos os autores.
Por exemplo, temos 2 entradas bibliográficas tal que: 

\begin{itemize}
	\item C and B 


O autor C publica com B, por sua vez, B publica com C,
	\item A and B and C


O autor A publica com B e com C, e B publica com A e com C,
e C publica com A e com B.
\end{itemize}

Ou seja, para uma linha com N autores, para cada autor
exitem N-1 associações. Para manter a consistência do grafo, e necessário inserir
essas N -1 associações mais o nome do autor, N vezes.



\section{Desenho e implementação da solução}
\label{sec:des:c}
Na implementação usou-se para criar o grafo, uma tabela de \emph{hash}
multi-nível, ou seja, uma tabela de \emph{hash} de tabelas de \emph{hash}, onde
cada ocorrência de autor está presente no primeiro nível, e para cada ocorrência
estão todas as ocorrências sem repetidos, com um contador associado. Além da
tabela existem uma \emph{string} auxiliar, para guardar os caracteres capturados
dos nomes dos autores. Cada nome é inserido num \emph{array} de \emph{strings}
sendo depois inseridas todas as associações na tabela de \emph{hash}. No fim,
é efetuada um procura na tabela de primeiro nível com o valor do autor
escolhido, sendo depois percorrida a tabela de segundo nível desse autor,
imprimindo no \emph{stdout} as associações já no formato \emph{Dot}. De igual
modo, existem duas variáveis inteiras correspondentes a cada índice de cada
\emph{array}. Por último, existe uma \emph{SC} com o nome \texttt{AUT} para
manipulação do valor do campo autor, nesse contexto.  



\subsection{Expressões Regulares}
Antes de se iniciar a descrição, note-se que as \emph{ERs} estão ordenadas de
forma a não haver ambiguidade.


\subsubsection{\emph{START CONDITION} \texttt{INITIAL}}

\begin{itemize}
	\item Captura campo autor (\emph{case insensitive}), de forma similar ao que
		se fez em capítulos anteriores. 
\begin{verbatim}
[Aa][Uu][Tt][Hh][Oo][Rr][ ]*"="[ ]*[{"][ ]*
\end{verbatim}

A captura despoleta a inicialização dos índices na posição 0 e inicia
\emph{SC} \texttt{AUT}.  

\item Captura tudo o resto, incluindo o caractere \emph{newline}.  
\begin{verbatim}
(.|\n)
\end{verbatim}

A ação é ignorar as capturas da \emph{ER}.

\end{itemize}


\subsubsection{\emph{START CONDITION} \texttt{AUT}}

\begin{itemize}
	\item Captura fim de campo, a semelhança com \emph{ER's} de capítulos
		anteriores.  

\begin{verbatim}
[ ]*[}"] 
\end{verbatim}

O procedimento associado a esta \emph{ER} é colocar o caractere \texttt{NULL}
na última posição da \emph{string} auxiliar, e copiar o valor para
o \emph{array} de \emph{strings} auxiliar. O índice deste último é incrementado,
sendo de seguida invocada a função \texttt{permute} que tratara da reorganização
do \emph{array} de \emph{strings} auxiliar, e da respetiva inserção na tabela de
\emph{hash}. No fim, volta à \emph{SC} \texttt{INITIAL}.  

\item Captura o separador \emph{and} rodeado de um ou mais espaços e tabulações.
\begin{verbatim}
[ /t]+(and)[ /t]+
\end{verbatim}

Aqui é colocado o caractere \texttt{NULL}  na última posição da \emph{string}
auxiliar, sendo copiada para a posição $j$ do \emph{array} de \emph{strings}
auxiliar.  A posição $j$ é total de ocorrências de autores até ao momento.
O índice da \emph{string} volta a primeira posição, pronto para a leitura de um
novo nome. 

\item Captura tudo o resto, incluindo o caractere \emph{newline}.  
\begin{verbatim}
(.|\n)
\end{verbatim}

Aqui é copiada a captura para a \emph{string} auxiliar, incrementado a sua
posição. 

\end{itemize}

\subsection{Estruturas de dados}

A estrutura de dados, como já foi mencionado atrás é uma tabela de \emph{hash}
multi-nível. A estrutura de base é a mesma que foi utilizada no primeiro
capítulo --- \emph{uthash}.  

\subsection{Algoritmos}

O código que se segue é o utilizado na inserção dos autores na tabela de
\emph{hash}. 

\begin{Verbatim}
void insertAuthors ( int length )
{int i;
 add_autor(authors[0]);
 for ( i=1; i < length; i++ )
     add_coautor(authors[0], authors[i] );
 				
}
\end{Verbatim}

O código seguinte é a versão da função \texttt{swap} utilizada na reorganização
do \emph{array} de \emph{strings}. 
\begin{Verbatim}
void swap(char **ap_str1, char **ap_str2)
{char *temp = *ap_str1;
*ap_str1 = *ap_str2;
*ap_str2 = temp;
}
\end{Verbatim}

A função \texttt{permute}, caso haja apenas um autor, insere apenas o mesmo. Se
houver mais para cada autor calcula-se uma permutação, tal que, o nome do autor
para ser inserido no primeiro nível é colocado à cabeça do \emph{array},
trocando de lugar com o autor seguinte da iteração. Assim obtém-se a permutação
pretendida.
\begin{Verbatim}
void permute ( int length )
{
    int j, i;
    if(length==1)
        {
            insertAuthors ( 1 );
            return;

        }


    for ( i = 0; i < length ; i++ )
        {
            swap( &authors[0], &authors[i] );
            insertAuthors ( length );
        }

    return;
}
\end{Verbatim}

\section{Testes e Resultados}
\label{sec:ts:c}


\subsection{Resultados}

Utilizando o nome de autor `Tim Teitelbaum', os resultados estão no
apêndice~\ref{appendix:d1}, na pág.~\pageref{appendix:d1}. O ficheiro
\hologo{BibTeX} utilizado está no apêndice~\ref{appendix:a1},
pág~\pageref{listing:a2}.

A imagem foi obtida através do comando \verb|dot -Tpng res_dot.dot -o out.png| 

A escolha do preâmbulo para o \emph{Dot}:

\begin{itemize}
	\item \texttt{strict directed graph} 

		Evitar múliplas arestas para o mesmo objeto.

	\item \texttt{ratio=fill} e \texttt{size x, y}. 
		
		Dado não caber na página é uniformemente reduzido, e escala para otimizar
		a razão y/x.

	\item \texttt{ranfir=LR} para colocar o autor do lado esquerdo.
\end{itemize}


\subsection{Alternativas, Decisões e Problemas de Implementação}

Numa primeira abordagem, tentou-se implementar uma solução que utilizava
o \texttt{strstr} do \texttt{libc}. A solução funcionava, no entanto, não era
eficiente. Era necessário ler a linha três vezes: uma para verificar se o autor
existia, outra para processar a linha, caso o autor existisse,
e o \texttt{strstr}. Optou-se pela solução implementada, dado que percorre
o documento uma vez, podendo demorar na função \texttt{permute} se houver muitos
autores. Uma nota final: para poder utilizar esta ferramenta com eficiência,
recomenda-se que se normalize primeiro o ficheiro com a ferramenta da parte 2,
uma vez que o mesmo autor pode ter o seu nome escrito de diferentes formas.





\chapter*{Conclusão}
\label{concl}

%Síntese do Documento.

%Estado final do projeto; Analise critica dos resultados.

%Trabalho futuro.


\nocite{*}

%\bibliographystyle{alpha}
%
\bibliography{./report/bibs/pl}

\appendix

\chapter{Código do \emph{HTML} da Parte 1}
\label{appendix:a}

\begin{longlisting}
	\inputminted{html}{testes/res_html.html}
	\caption{Ficheiro fonte do exercício 2.1}
	\label{listing:1}
\end{longlisting}




\begin{minted}{c}

#include <stdio.h>
#define N 10
/* Block
 * comment */

 int main()
 {
     int i;
	 
	   // Line comment.
		 puts("Hello world!");
			     
		 for (i = 0; i < N; i++)
		 {
		 puts("LaTeX is also great for programmers!");
		 }
							 
	   return 0;
				
}
\end{minted}


E ainda possível importar diretamente o ficheiro:










\end {document}


