\chapter{Analise e Especificacao}
\label{ae}
\section{Descricao Informal do Problema}



\section{Especificacao dos Requisitos}

\subsection{Dados}

\subsection{Pedidos}

\subsection{Relacoes}



2     Enunciados
Para sistematizar o trabalho que se pede em cada uma das propostas seguintes, considere que deve, em qualquer um
dos casos, realizar a seguinte lista de tarefas:

    1. Especificar os padroes de frases que quer encontrar no texto-fonte, atraves de ERs.
    2. Identificar as accoes semanticas a realizar como reaccao ao reconhecimento de cada um desses padroes.
    3. Identificar as Estruturas de Dados globais que possa eventualmente precisar para armazenar temporariamente a
       informacao que vai extraindo do texto-fonte ou que vai construindo a medida que o processamento avanca.
    4. Desenvolver um Filtro de Texto para fazer o reconhecimento dos padroes identificados e proceder a transformacao
       pretendida, com recurso ao Gerador Flex.

2.2    Normalizador de ficheiros BibTeX
BibTeX e uma ferramenta de formatacao de citacoes bibliograficas em documentos LATEX, criada com o objectivo de
facilitar a separacao da base de dados com a bibliografia consultada da sua apresentacao no fim do documento LATEX
em edicao. BibTeX foi criada por Oren Patashnik e Leslie Lamport em 1985, tendo cada entrada nessa base de dados
textual o aspecto que se ilustra a seguir:
@InProceedings{CPBFH07e,
  author =    {Daniela da Cruz and Maria Joao Varanda Pereira
               and Mario Beron and Ruben Fonseca and
               Pedro Rangel Henriques},
  title =     {Comparing Generators for Language-based Tools},
  booktitle = {Proceedings of the 1.st Conference on Compiler
               Related Technologies and Applications, CoRTA'07
               --- Universidade da Beira Interior, Portugal},
  year =      {2007},
  editor =    {},
  month =     {Jul},
}

De modo a familiarizar-se com o formato do BibTeX podera consultar o ficheiro lp.bib dispon?vel em http://www.
di.uminho.pt/~prh/lp.bib e ainda a pagina oficial do formato referido (http://www.bibtex.org/), devendo para
ja saber que a primeira palavra (logo a seguir ao caracter "@") designa a categoria da referencia (havendo em BibTeX
pelo menos 14 diferentes).
As tarefas que devera executar neste trabalho pratico sao:

a) Analise o documento BibTeX referido acima e faca a contagem das categorias (phDThesis, Misc, InProceeding,
    etc.), que ocorrem no documento. No final, devera produzir um documento em formato HTML com o nome das
    categorias encontradas e respectivas contagens.
b) Desenvolva uma ferramenta de normalizacao (sempre que um campo esta entre aspas, troque para chavetas e
    escreva o nome dos autores no formato "N. Apelido") e faca uma ferramenta de pretty-printing que indente
    corretamente cada campo, escreva um autor por linha e coloque sempre no in?cio os campos autor e t?tulo.
c) Construa um Grafo que mostre, para um dado autor (escolhido pelo utilizador) todos os autores que publicam
     normalmente com o autor em causa.
     Recorrendo a linguagem Dot do GraphViz2 , gere um ficheiro com esse grafo de modo a que possa, posteriormente,
     usar uma das ferramentas que processam Dot 3 para desenhar o dito grafo de associacoes de autores.

2.3    From treebanks to probablilistic grammar
