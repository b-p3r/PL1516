\chapter{Concepção/Desenho da Resolução}
\section{Estruturas de Dados}
\section{Algoritmos}

2.2    Normalizador de ficheiros BibTeX
BibTeX é uma ferramenta de formatação de citações bibliográficas em documentos LATEX, criada com o objectivo de
facilitar a separação da base de dados com a bibliografia consultada da sua apresentação no fim do documento LATEX
em edição. BibTeX foi criada por Oren Patashnik e Leslie Lamport em 1985, tendo cada entrada nessa base de dados
textual o aspecto que se ilustra a seguir:
@InProceedings{CPBFH07e,
  author =    {Daniela da Cruz and Maria Joao Varanda Pereira
               and Mario Beron and Ruben Fonseca and
               Pedro Rangel Henriques},
  title =     {Comparing Generators for Language-based Tools},
  booktitle = {Proceedings of the 1.st Conference on Compiler
               Related Technologies and Applications, CoRTA’07
               --- Universidade da Beira Interior, Portugal},
  year =      {2007},
  editor =    {},
  month =     {Jul},
}

De modo a familiarizar-se com o formato do BibTeX poderá consultar o ficheiro lp.bib disponı́vel em http://www.
di.uminho.pt/~prh/lp.bib e ainda a página oficial do formato referido (http://www.bibtex.org/), devendo para
já saber que a primeira palavra (logo a seguir ao caracter ”@”) designa a categoria da referência (havendo em BibTeX
pelo menos 14 diferentes).
As tarefas que deverá executar neste trabalho prático são:

a) Analise o documento BibTeX referido acima e faça a contagem das categorias (phDThesis, Misc, InProceeding,
    etc.), que ocorrem no documento. No final, deverá produzir um documento em formato HTML com o nome das
    categorias encontradas e respectivas contagens.
b) Desenvolva uma ferramenta de normalização (sempre que um campo está entre aspas, troque para chavetas e
    escreva o nome dos autores no formato ”N. Apelido”) e faça uma ferramenta de pretty-printing que indente
    corretamente cada campo, escreva um autor por linha e coloque sempre no inı́cio os campos autor e tı́tulo.
c) Construa um Grafo que mostre, para um dado autor (escolhido pelo utilizador) todos os autores que publicam
     normalmente com o autor em causa.
     Recorrendo à linguagem Dot do GraphViz2 , gere um ficheiro com esse grafo de modo a que possa, posteriormente,
     usar uma das ferramentas que processam Dot 3 para desenhar o dito grafo de associações de autores.

2.3    From treebanks to probablilistic grammar
