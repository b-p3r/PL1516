\chapter{Grafo de associações de um autor em \emph{Dot} de um ficheiro \BibTeX}
\label{chap:c}

\section{Análise do Problema}
\label{sec:cp:c}
Nesta parte, requere-se que se faça o grafo de associações para um dado autor,
dos autores que usualmente publicam com ele. A linguagem a ser usada é linguagem
\emph{Dot} do \emph{GraphViz}, para renderizar o grafo, tendo que
o filtro em \emph{Flex} gerar no \emph{stdout} o grafo nessa
linguagem.   

\section{Especificação dos requisitos}
\label{sec:spec:c}

\subsection{Dados}

Em relação aos dados do problema, pouco mais há a acrescentar do que
já se mencionou em capítulos anteriores, relativamente ao
\hologo{BibTeX} --- o campo autor podes ter um ou mais autores, cada
um separado por \emph{and}, podendo o campo estar dentro de chavetas
ou aspas.

Em relação ao \emph{Dot} existem algumas considerações a ter,
nomeadamente no desenho do grafo através da linguagem. Pretende-se um
grafo, de preferência orientado, onde numa única página seja possível
mostrar todas as associações. Cada associação pode ter um peso,
correspondente à densidade de publicação e existe uma, e uma só
associação entre o autor escolhido e cada um dos seus co-autores. Além
do mais, podem ocorrer os seguintes casos: o autor tem vários
co-autores, o autor nunca publicou com ninguém, e o autor escolhido
não existe. Acresce-se que o grafo é sempre produzido, estando vazio
no caso de o autor não existir ou haver apenas um nodo com o nome do
autor escolhido, no caso de autor ter sempre publicado sozinho.

\subsection{Relações}

Relativamente às associações, existem relações subjacentes que
é preciso interpretar. Pretendendo-se todas as associações entre cada autor que
ocorre no campo homólogo no ficheiro do \hologo{BibTeX}, é necessário
construir um grafo com todas as associações entre todos os autores.
Por exemplo, temos 2 entradas bibliográficas tal que: 

\begin{itemize}
	\item C and B 
O autor C publica com B, por sua vez, B publica com C,
	\item A and B and C
O autor A publica com B e com C, e B publica com A e com C,
e C publica com A e com B.
\end{itemize}

Ou seja, para uma linha com N autores, para cada autor
exitem N associações. Para manter a consistência do grafo, e necessário inserir
essas N associações, N vezes.



\section{Desenho e implementação da solução}
\label{sec:des:c}
Na implementação usou-se para criar o grafo, uma tabela de \emph{hash}
multi-nível, ou seja, uma tabela de \emph{hash} de tabelas de \emph{hash}, onde
cada ocorrência de autor está presente no primeiro nível, e para cada ocorrência
estão todas as ocorrências sem repetidos, com um contador associado. Além da
tabela existem uma \emph{string} auxiliar, para guardar os caracteres capturados
dos nomes dos autores. Cada nome é inserido num \emph{array} de \emph{strings}
sendo depois inseridas todas as associações na tabela de \emph{hash}. No fim,
é efetuada um procura na tabela de primeiro nível com o valor do autor
escolhido, sendo depois percorrida a tabela de segundo nível desse autor,
imprimindo no \emph{stdout} as associações já no formato \emph{Dot}. De igual
modo, existem duas variáveis inteiras correspondentes a cada índice de cada
\emph{array}. Por último, existe uma \emph{SC} com o nome \texttt{AUT} para
manipulação do valor do campo autor, nesse contexto.  



\subsection{Expressões Regulares}
Antes de se iniciar a descrição, note-se que as \emph{ERs} estão ordenadas de
forma a não haver ambiguidade.


\subsubsection{\emph{START CONDITION} \texttt{INITIAL}}

\begin{itemize}
	\item Captura campo autor (\emph{case insensitive}), de forma similar ao que
		se fez em capítulos anteriores. 
\begin{verbatim}
[Aa][Uu][Tt][Hh][Oo][Rr][ ]*"="[ ]*[{"][ ]*
\end{verbatim}

A captura despoleta a inicialização dos índices na posição 0 e inicia
\emph{SC} \texttt{AUT}.  

\item Captura tudo o resto, incluindo o caractere \emph{newline}.  
\begin{verbatim}
(.|\n)
\end{verbatim}

A ação é ignorar as capturas da \emph{ER}.

\end{itemize}


\subsubsection{\emph{START CONDITION} \texttt{AUT}}

\begin{itemize}
	\item Captura fim de campo, a semelhança com \emph{ER's} de capítulos
		anteriores.  

\begin{verbatim}
[ ]*[}"] 
\end{verbatim}

O procedimento associado a esta \emph{ER} é colocar o caractere \texttt{NULL}
na última posição da \emph{string} auxiliar, e copiar o valor para
o \emph{array} de \emph{strings} auxiliar. O índice deste último é incrementado,
sendo de seguida invocada a função \texttt{permute} que tratara da reorganização
do \emph{array} de \emph{strings} auxiliar, e da respetiva inserção na tabela de
\emph{hash}. No fim, volta à \emph{SC} \texttt{INITIAL}.  

\item Captura o separador \emph{and} rodeado de um ou mais espaços e tabulações.
\begin{verbatim}
[ /t]+(and)[ /t]+
\end{verbatim}

Aqui é colocado o caractere \texttt{NULL}  na última posição da \emph{string}
auxiliar, sendo copiada para a posição $j$ do \emph{array} de \emph{strings}
auxiliar.  A posição $j$ é total de ocorrências de autores até ao momento.
O índice da \emph{string} volta a primeira posição, pronto para a leitura de um
novo nome. 

\item Captura tudo o resto, incluindo o caractere \emph{newline}.  
\begin{verbatim}
(.|\n)
\end{verbatim}

Aqui é copiada a captura para a \emph{string} auxiliar, incrementado a sua
posição. 

\end{itemize}

\subsection{Estruturas de dados}

A estrutura de dados, como já foi mencionado atrás é uma tabela de \emph{hash}
multi-nível. A estrutura de base é a mesma que foi utilizada no primeiro
capítulo --- \emph{uthash}.  

\subsection{Algoritmos}

O código que se segue é o utilizado na inserção dos autores na tabela de
\emph{hash}. 

\begin{Verbatim}
void insertAuthors ( int length )
{int i;
 add_autor(authors[0]);
 for ( i=1; i < length; i++ )
     add_coautor(authors[0], authors[i] );
 				
}
\end{Verbatim}

O código seguinte é a versão da função \texttt{swap} utilizada na reorganização
do \emph{array} de \emph{strings}. 
\begin{Verbatim}
void swap(char **ap_str1, char **ap_str2)
{char *temp = *ap_str1;
*ap_str1 = *ap_str2;
*ap_str2 = temp;
}
\end{Verbatim}

A função \texttt{permute}, caso haja apenas um autor, insere apenas o mesmo. Se
houver mais para cada autor calcula-se uma permutação, tal que, o nome do autor
para ser inserido no primeiro nível é colocado à cabeça do \emph{array},
trocando de lugar com o autor seguinte da iteração. Assim obtém-se a permutação
pretendida.
\begin{Verbatim}
void permute ( int length )
{
    int j, i;
    if(length==1)
        {
            insertAuthors ( 1 );
            return;

        }


    for ( i = 0; i < length ; i++ )
        {
            swap( &authors[0], &authors[i] );
            insertAuthors ( length );
        }

    return;
}
\end{Verbatim}

\section{Testes e Resultados}
\label{sec:ts:c}


\subsection{Resultados}

Utilizando o nome de autor `Tim Teitelbaum', os resultados estão no
apêndice~\ref{appendix:d1}, na pág.~\pageref{appendix:d1}. O ficheiro
\hologo{BibTeX} utilizado está no apêndice~\ref{appendix:a1},
pág~\pageref{listing:a2}.

A imagem foi obtida através do comando \verb|dot -Tpng res_dot.dot -o out.png| 


\subsection{Alternativas, Decisões e Problemas de Implementação}

Numa primeira abordagem, tentou-se implementar uma solução que utilizava
o \texttt{strstr} do \texttt{libc}. A solução funcionava, no entanto, não era
eficiente. Era necessário ler a linha três vezes: uma para verificar se o autor
existia, outra para processar a linha, caso o autor existisse,
e o \texttt{strstr}. Optou-se pela solução implementada, dado que percorre
o documento uma vez, podendo demorar na função \texttt{permute} se houver muitos
autores. Uma nota final: para poder utilizar esta ferramenta com eficiência,
recomenda-se que se normalize primeiro o ficheiro com a ferramenta da parte 2,
uma vez que o mesmo autor pode ter o seu nome escrito de diferentes formas.



