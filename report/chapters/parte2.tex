\chapter{Ferramenta de normalização de um ficheiro \hologo{BibTeX}}
\label{chap:b1}



\section{Análise do Problema}
\label{sec:b1p:b1}
Para além dos tipos de entrada, é necessário especificar o conteúdo da entrada,
como nomes de autor, títulos de obra, editora, etc. O \hologo{BibTeX} possui
propriedades definidas para cada item como campo da entrada.  
Para esta parte do trabalho, é pedido o desenvolvimento de uma ferramenta de
normalização dos nomes dos autores no campo respetivo, no formato "N."(ome)
"Apelido", que de igual modo normalize todos os campos entre aspas, com
chavetas. 

\subsection{Especificação dos requisitos}
\label{sec:spec:b1}

\subsection{Dados}
Os nomes dos autores podem ter muitos formatos. Como por exemplo:

\begin{itemize}
	\item \emph{Donald E. Knuth}
	\item \emph{D. E. Knuth}
	\item \emph{Knuth, Donald E.}
	\item \emph{Knuth, PhD, Donald E.}
	\item \emph{Nicollo Alighieri Franchi-Zanettachi}
	\item \emph{Daniela da Cruz}
\end{itemize}

Assim, há uma necessidade de especializar um conjunto de \emph{ER's} para tratar
cada caso, com especial atenção para os nomes no formato \emph{Apelido, Nome}.

De igual modo, temos que cada campo pode começar com uma chaveta ou aspas,
terminando de igual forma, com a chaveta fechada ou aspas correspondente.

Um conceito importante no \TeX em geral, é que um documento está \emph{bem
formado} se todas as chavetas abertas tiverem a chaveta fechada correspondente.
De facto, existem estilos de bibliografias que convertem o primeiro caractere
que compõe o valor do campo em maiúscula e os restantes em minusculas. Esta
funcionalidade ocorre para nomes de um título ou outro campo, que não o do
autor. Por vezes é necessário manter as maiúsculas, dado que existem valores de
campo em que, por exemplo, o primeiro caractere de cada palava está
capitalizado. O \hologo{BibTeX} permite ao utilizador abrir e fechar chavetas em
torno do conjunto de caracteres onde se pretende manter a capitalização.
A relevância deste contexto será explicada na secção seguinte.


\section{Desenho e implementação da solução}
\label{sec:des:b1}

\subsection{\emph{INITIAL \emph{START CONDITIONS}}}
\subsubsection{Expressões Regulares e Ações}
\subsection{\emph{AUT \emph{START CONDITION}}}
\subsubsection{Expressões Regulares e Ações}
\subsection{\emph{PREFORMAT \emph{START CONDITION}}}
\subsubsection{Expressões Regulares e Ações}
\subsection{\emph{CHAV \emph{START CONDITION}}}
\subsubsection{Expressões Regulares e Ações}
\subsection{\emph{SPEC \emph{START CONDITION}}}
\subsubsection{Expressões Regulares e Ações}

\subsection{Algoritmos}

\section{Testes e Resultados}
\subsection{Resultados}
\label{sec:ts:b1}

\subsection{Alternativas, Decisões e Problemas de Implementação}

Adicionalmente à solução descrita neste capitulo, poder-se-ia ter implementado
ou mais uma \emph{START CONDITION} ou possivelmente mais algumas \emph{ER's}
que tratassem de nomes de sufixo como em  \emph{Knuth, PhD, Donald E.}.

Assumiu-se uma codificação \emph{ASCII}, pelo que não foram tratados caracteres
em \emph{UTF-8} ou \emph{ISO 8859-1}. Para tal ter-se-ia que tratar os
caracteres com o tamanho de dois \emph{bytes} e capturar sequências de escape
para caracteres especiais em determinado ficheiro \hologo{BibTeX} e guardá-los
como caracteres de dois \emph{bytes}. 

\subsection{Testes e Resultados}
%Mostram-se a seguir alguns testes feitos (valores introduzidos) e
%%os respetivos resultados obtidos:

