\chapter{Ferramenta de normalização de um ficheiro \hologo{BibTeX}}
\label{chap:b1}



\section{Análise do Problema}
\label{sec:b1p:b1}
Para além dos tipos de entrada, é necessário especificar o conteúdo da entrada,
como nomes de autor, títulos de obra, editora, etc. O \hologo{BibTeX} possui
propriedades definidas para cada item como campo da entrada.  
Para esta parte do trabalho, é pedido o desenvolvimento de uma ferramenta de
normalização dos nomes dos autores no campo respetivo, no formato \emph{N. (ome)
Apelido}, que de igual modo normalize todos os campos entre aspas, com
chavetas. 

\subsection{Especificação dos requisitos}
\label{sec:spec:b1}

\subsection{Dados}
Os nomes dos autores podem ter muitos formatos. Como por exemplo:

\begin{itemize}
  \item \emph{Donald E. Knuth}
  \item \emph{D. E. Knuth}
  \item \emph{Knuth, Donald E.}
  \item \emph{Knuth, PhD, Donald E.}
  \item \emph{Nicollo Alighieri Franchi-Zanettachi}
  \item \emph{Daniela da Cruz}
\end{itemize}

Assim, há uma necessidade de especializar um conjunto de \emph{ER's} para tratar
cada caso, com especial atenção para os nomes no formato \emph{Apelido, Nome}.

De igual modo, temos que cada campo pode começar com uma chaveta ou aspas,
terminando de igual forma, com a chaveta fechada ou aspas correspondente.

Um conceito importante no \TeX{} em geral, é que um documento está \emph{bem
formado} se todas as chavetas abertas tiverem a chaveta fechada correspondente.
De facto, existem estilos de bibliografias que convertem o primeiro caractere
que compõe o valor do campo em maiúscula e os restantes em minusculas. Esta
funcionalidade ocorre para nomes de um título ou outro campo, que não o do
autor. Por vezes é necessário manter as maiúsculas, dado que existem valores de
campo em que, por exemplo, o primeiro caractere de cada palava está
capitalizado. O \hologo{BibTeX} permite ao utilizador abrir e fechar chavetas em
torno do conjunto de caracteres onde se pretende manter a capitalização.
A relevância deste contexto será explicada na secção seguinte.


\section{Desenho e implementação da solução}
\label{sec:des:b1}


\subsection{Expressões Regulares}

Antes de se iniciar a descrição, note-se que as \emph{ERs} estão ordenadas de
forma a não haver ambiguidade. 

\subsubsection{\emph{INITIAL \emph{START CONDITIONS}}}


Qualquer campo no \hologo{BibTeX} tem sempre o identificador do campo
(\emph{author}, \emph{title}, etc.) seguido de um '\texttt{=}', começando
o valor do campo com a abertura de aspas ou chavetas. Todavia, entre os
o identificador do campo, '\texttt{=}' e \texttt{\{}, pode não haver espaços, ou
pode haver um ou mais espaços. Como a ferramente de normalização tem duas
funcionalidades diferentes conforme os campos, é necessário ter duas
\emph{ER's}: uma que trate de tudo o que é necessário fazer com o campo autor
e outra, genérica, que trate dos restantes no contexto da normalização das
chavetas. Dado que, a captura do campo autor e do processamento das chavetas são
dois contextos diferentes, é necessário recorrer ao uso de \emph{START
CONDITIONS}. Assim, para estes últimos implementou-se a \emph{START 
CONDITION} \texttt{AUT} para tratar do autor, e a \emph{START
CONDITION} \texttt{CHAV} para tratar das chavetas dos outros campos.


Após o exposto atrás, temos duas expressões regulares, tais que:

\begin{itemize}
  \item Captura campo autor
\begin{verbatim}
    [Aa][Uu][Tt][Hh][Oo][Rr][ ]*"="[ ]*[{"]
\end{verbatim}
A ação para esta expressão regular é colocar no último caractere capturado uma
chaveta aberta, imprimir no \emph{stdout} e iniciar \texttt{AUT}.
    

  \item Captura qualquer outro campo;
\begin{verbatim}
     [A-Za-z]+[ ]*"="[ ]*[{"]
\end{verbatim}
A ação para esta expressão regular é colocar no último caractere capturado uma
chaveta aberta, imprimir no \emph{stdout} e iniciar \texttt{CHAV}.
\end{itemize}


\subsubsection{\emph{AUT \emph{START CONDITION}}}
\label{subsubsection:des:b1}

Nesta \emph{START CONDITION} faz-se a distinção dos nomes, conforme na
\emph{Secção}~\ref{sec:spec:b1}. No entanto, é necessário distinguir o que é um
nome de um autor. Um nome de uma autor pode-ser seguido de um \emph{and}, se
houver mais que um autor, e também pode ser único ou o último, e terminar em
aspas ou numa chaveta. Dentro do nome, faz-se a distinção de um nome poder ser
uma inicial, ou uma palavra que inicie com maiúscula, seguida de uma ou mais
letras minúsculas --- por definição, um nome próprio ---, ou pode ser um nome
composto (dois nomes próprios separados por um hífen). De igual modo, os nomes
podem conter uma preposição dentro do nome, e podem ter vários espaços e/ou
tabulações antes e depois do nome. Acresce também a condição especial de que os
nomes podem estar no formato \emph{Apelido, Nome} e neste caso é necessário uma novo
contexto para tratar este caso especial. Para isso, criou-se a \emph{START
CONDITION} \texttt{PREFORMAT}. 



\begin{itemize}
  \item Captura de aspas ou uma chaveta no final campo.
		\begin{verbatim}  
    [}"]
    \end{verbatim}
    Neste caso, como se poderá ver mais à frente, pode aparentar ser redundante.
    Ou seja, já existem \emph{ER's} que tratam da chaveta ou aspas como fim de
    campo.  Todavia, por uma questão de coerência e segurança, em caso de uma
    captura para entrar nos estado da \emph{START CONDITION} \texttt{PREFORMAT},
    no final do processamento do campo nesta condição, volta-se sempre ao estado
    anterior \texttt{AUT}. Assim, o final de campo é sempre processado no mesmo
    contexto.

    A ação nesta \emph{ER} é consumir o valor, imprimindo no \emph{stdout} uma
    chaveta, voltando à \emph{START CONDITION} \texttt{INITIAL}.

  \item Captura de 1 ou mais espaços ou tabulações.
		\begin{verbatim}  
     [ \t]+
    \end{verbatim}

     Esta \emph{ER} garante que espaços ou tabulações em torno dos nomes são
     consumidos, de forma a ter apenas os caracteres correspondentes aos nomes.
     De igual modo, imprime um espaço no \emph{stdout}.
  
   \item Captura de um espaço ou tabulação antes e depois de uma
     preposição, e a própria preposição.
		\begin{verbatim}  
    [ \t][a-z]+[\t ]
    \end{verbatim}


     As preposições em nomes com \emph{Daniela da Silva} são ignorados.
     A ação é mesma que na \emph{ER} anterior.

   \item Captura de uma inicial ou um nome próprio, que não apelido.
    \begin{verbatim}
     [A-Z]((\.)?|[a-z]+)
    \end{verbatim}

  A ação neste caso é obter o primeiro caractere da captura da expressão,
  e imprimir no \emph{stdout} o mesmo caractere seguido de um ponto.


   \item Captura de um nome próprio, que é apelido, pode ser composto
     e é o último da listagem ou único.
    \begin{verbatim}
    ((-)?[A-Z][a-z]+)+[ \t]*[}"]
    \end{verbatim}
 O nome pode ter ou não iniciar com um hífen, seguido do nome próprio. Estes
 dois itens podem ocorrer uma ou mais vezes. Por exemplo, no apelido do nome
 \emph{Nicollo Franchi-Zanettachi}. Neste caso, o nome do autor é o último nome
 listado ou o único. Assim, espaços e tabulações antes do final da linha são
 capturados, 0 ou mais vezes (apelido pode estar junto das aspas ou da chaveta).

  A ação pretendida aqui é modificar o último caractere para uma chaveta,
  imprimir o resultado para o \emph{stdout} e voltar a \emph{START CONDITION}
  \texttt{INITIAL}.

   \item Captura de um nome próprio, que é apelido, pode ser composto
          mas não é o único na listagem de autores.
    \begin{verbatim}
     ((-)?[A-Z][a-z]+)+[ \t]+(and)[ \t]+
    \end{verbatim}

    A ação correspondente é em quase tudo semelhante à ação anterior, no
    entanto, captura todos os espaços e tabulações, antes e depois do separador
    \emph{and}, e o separador \emph{and}. Neste caso imprime para o \emph{stdout}
    a captura. Assim o próximo fica livre de espaços no inicio.


  \item Captura de uma linha com a formatação \emph{Apelido, Nome}.
    \begin{verbatim}
     ((-)?[A-Z][a-z]+)+[,]+[ \t]
    \end{verbatim}
    A \emph{ER} captura, a além do nome, pelo menos uma vírgula depois do nome,
    estando este seguido de espaços. Assume-se que, se a formatação no primeiro
    nome possui vírgulas, logo a restante está na mesma formatação. A ação
    é colocar o apontador de leitura do \emph{Flex} no inicio da linha, colocar
    uma variável inteira para um índice de um vetor a 0, inicializar
    \emph{array} de \emph{strings} a \texttt{NULL}\footnote{Algoritmo e código
      será apresentado na \emph{Subsecção~\ref{subsec:des:algol} na
      pág.~\pageref{subsec:des:algol}}}.  Em seguida inicia a \emph{START
      CONDITION} \texttt{PREFORMAT}.


   \item Captura tudo o resto incluindo \emph{newline}.
    \begin{verbatim}
    (.|\n)
    \end{verbatim}
    A ação é ignorar tudo, que seja capturado por esta \emph{ER}.
\end{itemize}




\subsubsection{\emph{PREFORMAT \emph{START CONDITION}}}

Esta \emph{START CONDITION} tem as \emph{ER's} iguais, exceto a \emph{ER} \verb|[, \t]+| e 
o separador \emph{and} tem um tratamento diferente, bem como o final do valor do
campo. A maior parte das ações foi redefinida para para este novo contexto.




\begin{itemize}
  \item Captura de aspas ou uma chaveta no final campo.

    \begin{verbatim}
    [}"]
    \end{verbatim}

		Como foi anteriormente mencionado, quando capturado ou aspas ou uma chaveta,
		obriga-se o \emph{Flex} a colocar a captura no \emph{stdin}
		(\texttt{yyless(0)}). De seguida coloca-se o índice do \emph{array} de
		\emph{strings}, mencionado anteriormente, na posição inicial e executa-se
		uma função para imprimir os valores contidos no
		\emph{array}.\footnote{Algoritmo e código descrito na
			\emph{Subsecção~\ref{subsec:des:algol}, na
			pág.~\pageref{subsec:des:algol}}}

  \item Captura de 1 ou mais espaços, tabulações ou vírgulas.
     \begin{verbatim}
     [, \t]+|
     \end{verbatim}
     A semelhança da anterior \emph{ER}, para além de garantir que espaços ou
     tabulações em torno dos nomes são consumidos, garante de igual modo que
     vírgulas sejam consumidas. No entanto, não imprime um espaço. No seu lugar
     da ação de impressão de um espaço, a variável do tipo inteiro para o índice
     do \emph{array} de \emph{strings} é incrementada por uma unidade. Ou seja,
     por cada nova ocorrência de um novo nome, o índice do \emph{array} já se
     encontra na posição correta.

   \item Captura de um espaço ou tabulação antes e depois de uma
     preposição, e a própria preposição.
    \begin{verbatim}
     [ \t]+(and)+[\t ]+
    \end{verbatim}
    A ação definida captura o separador \emph{and}, caso exista mais que um autor, após
    a ocorrência deste. Note-se que pode encontrar um ou mais mais espaços, bem
    como tabulações antes e depois deste separador. Além disso, coloca
    o índice do apontador do \emph{array} de \emph{strings} auxiliar na posição
    inicial, imprimindo os nomes próprios do autor pela ordem desejada ---
    \emph{N. Apelido} em troca de \emph{Apelido, Nome}. Após
    este passo, imprime no \emph{stdout} a \emph{string} \emph{and} com um espaço
    de cada lado, iniciando a \emph{START CONDITION} \texttt{AUT}.
    Note-se que, o tratamento do conjunto de nomes é aqui diferente da \emph{START
    CONDITION} \texttt{AUT}, uma vez que, em \texttt{AUT}, não se podia saber
    qual seria o último nome, a não ser uma implementação especializada na parte
    da ação em linguagem \emph{C}. O intuito deste projeto foi sempre usar ao máximo
    a funcionalidade do \emph{Flex}, e tentar usar ao máximo \emph{ER's}. Aqui,
    como sabemos que o último nome do autor é logo o primeiro a ser listado.,

  
   \item Captura de um espaço ou tabulação antes e depois de uma
     preposição, e a própria preposição.
    \begin{verbatim}
    [ \t][a-z]+[\t ]
    \end{verbatim}
    A ação é mesma que na \emph{ER} equivalente, descrita na
    \emph{Subsubsecção~\ref{subsubsection:des:b1}} na
    pág.~\pageref{subsubsection:des:b1}, que é imediatamente antes desta.


   \item Captura de uma inicial ou um nome próprio, que  pode ser composto.
    \begin{verbatim}
    ((-)?[A-Z]((\.)|[a-z]+))
    \end{verbatim}
    A \emph{ER} tem o mesmo intuito que a \emph{ER} equivalente da
    \emph{Subsubsecção~\ref{subsubsection:des:b1}} na
    pág.~\pageref{subsubsection:des:b1}, no âmbito da captura. No entanto,
    a ação é diferente.  Esta passa por comparar a posição atual do índice do
    \emph{array} auxiliar para identificar a ordem dos nomes. Assim, para cada
    ocorrência de um nome próprio de um autor, se for diferente da posição $0$,
    é colocado no valor de \texttt{yytext} na posição $1$ um um ponto e na
    posição seguinte, o caractere \verb|'\0'|, copiando esse valor para
    o \emph{array} de \emph{strings} auxiliar, na posição atual do índice do
    \emph{array}. Caso contrário, copia o valor completo da \emph{string} para
    a posição $0$, do mesmo \emph{array} auxiliar.  

   \item Captura tudo o resto incluindo \emph{newline}.
    \begin{verbatim}
    (.|\n)
    \end{verbatim}
    A ação é ignorar tudo o que seja capturado por esta \emph{ER}.


\end{itemize}


\subsubsection{\emph{CHAV \emph{START CONDITION}}}

No contexto \texttt{CHAV}, apenas se faz a captura das aspas ou chavetas de todos
os outros campos. As chavetas ou aspas do campo do autor já estão previstas  no
devido contexto. No entanto, aqui pode ocorrer o novo contexto, já mencionado na
secção \emph{Análise do Problema} deste capítulo. Deste modo, há a necessidade
de se ter uma nova \emph{START CONDITION} \texttt{SPEC}. 


\begin{itemize}
  \item Captura uma chaveta aberta.
    \begin{verbatim}
      [{]
    \end{verbatim}
    A ação consiste imprimir para o \emph{stdout} a chaveta e iniciar
    a \emph{START CONDITION} \texttt{SPEC}.

  \item Captura do fim de campo, podendo ser uma chaveta ou aspas.
    \begin{verbatim}
    [}"]
    \end{verbatim}
    Ação: imprimir a chaveta de fecho, e voltar para as \emph{START CONDITION}
    \texttt{INITIAL}

  \item Captura qualquer outro campo;
    \begin{verbatim}
      (.|\n)
    \end{verbatim}
\end{itemize}
Ação: imprimir tudo o resto, incluindo \emph{newlines} para \emph{stdout}.



\subsubsection{\emph{SPEC \emph{START CONDITION}}}

Há apenas acrescentar sobre esta \emph{START CONDITION}, que apenas é um
\emph{workaroud} para evitar de capturar uma chaveta de a meio do valor do
campo, e ser passível de ser considerada fim do valor de campo. 


\begin{itemize}
  \item Captura o fecho de chavetas.
    \begin{verbatim}
    [}]
    \end{verbatim}
    Imprime para o \emph{stdout} a mesma chaveta.


  \item Captura qualquer outro campo;
    \begin{verbatim}
      (.|\n)
    \end{verbatim}
    Imprime tudo o resto no \emph{stdout}.
  
\end{itemize}
\subsection{Algoritmos}
\label{subsec:des:algol}
\begin{verbatim}
     void print_array (char ** array)
     {
         int i;
       
         for (i = 1; i < ARRAY_SIZE&&array[i]; i++)
             {
                 
                   printf("%s ", array[i]);
             }
         printf("%s", array[0]);
                         
                                   
     }

\end{verbatim}

A função acima corresponde à função de impressão dos nomes próprios dos autores
na \emph{START CONDITION} \texttt{PREFORMAT}. Note-se que apenas valores
existentes no \emph{array} são imprimidos no \emph{stdout}, começando pela
segunda posição. A primeira posição é imprimida no fim.

\begin{verbatim}
  void clean_array (char ** array)
  {
      int i;
    
      for (i = 0; i < ARRAY_SIZE&&array[i]; 
                 free(array[i]), array[i++]=NULL);
      
      
  }

\end{verbatim}

A função acima corresponde à função de inicialização do \emph{array} dos nomes
próprios dos autores na \emph{START CONDITION} \texttt{PREFORMAT}. Note-se que
liberta a memória e inicializa valores previamente existentes.

\newpage
\section{Testes e Resultados}
\label{sec:ts:b1}
\subsection{Resultados}


O ficheiro de usado para este teste pode ser visto no
\emph{Apêndice}~\ref{appendix:a1} na pág.~\pageref{appendix:a1}. 

O resultado pode ser consultado no \emph{Apêndice}~\ref{appendix:b} na
pág.~\pageref{appendix:b}

\subsection{Alternativas, Decisões e Problemas de Implementação}

Adicionalmente à solução descrita neste capitulo, poder-se-ia ter implementado
ou mais uma \emph{START CONDITION} ou possivelmente mais algumas \emph{ER's}
que tratassem de nomes de sufixo como em  \emph{Knuth, PhD, Donald E.}.

Assumiu-se uma codificação \emph{ASCII}, pelo que não foram tratados caracteres
em \emph{UTF-8} ou \emph{ISO 8859-1}. Para tal ter-se-ia que tratar os
caracteres com o tamanho de dois \emph{bytes} e capturar sequências de escape
para caracteres especiais em determinado ficheiro \hologo{BibTeX} e guardá-los
como caracteres de dois \emph{bytes}. 






