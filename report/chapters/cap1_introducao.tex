\chapter*{Introducao} \label{intro}

\begin{description}
  \item [Enquadramento] \textbf{bla bla} bla bla
  \item [Conteudo documento] \textsf{ble ble} \texttt{ble} ble
  \item [Resultados -- pontos a evidenciar] \textit{bli bli bli bli}
  \item [Estrutura do documento] \underline{blo blo blo}
\end{description}

Letras gregas sao estas $ \alpha \beta \gamma \delta $ que aqui demonstro

Exemplo simples de fracao \[ \frac{\frac{a * b + c}{4-3}}{3*5} \] simples

\section*{Estrutura do Relatorio} 


Explicar como esta? organizado o documento, referindo os capitulos existentes
em~\cite{yu09} e a sua articulacao explicando o conteudo de cada um.  No
capitulo \ref{ae} faz-se uma anA?lise detalhada do problema proposto de modo
a poder-se especificar  as entradas, resultados e formas de transformaA?A?o.\\
etc. \ldots


No capitulo~\ref{concl} termina-se o relatorio com uma sintese do que foi dito,
as conclusoes e o trabalho futuro

1     Objectivos e Organizacao
Este trabalho pratico tem como principais objectivos:

    o aumentar a experiencia de uso do ambiente Linux, da linguagem imperativa C (para codificacao das estruturas
      de dados e respectivos algoritmos de manipulacao), e de algumas ferramentas de apoio a programacao;
    o aumentar a capacidade de escrever Expressoes Regulares (ER) para descricao de padroes de frases;
    o desenvolver, a partir de ERs, sistematica e automaticamente Processadores de Linguagens Regulares, que filtrem
      ou transformem textos;
    o utilizar geradores de filtros de texto, como o Flex

Para o efeito, esta folha contem varios enunciados, dos quais devera resolver pelo menos um.

