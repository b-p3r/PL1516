\chapter*{Introdução} \label{intro}

\begin{description}
  \item [Enquadramento] \textbf{bla bla} bla bla
  \item [Conteúdo documento] \textsf{ble ble} \texttt{ble} ble
  \item [Resultados -- pontos a evidenciar] \textit{bli bli bli bli}
  \item [Estrutura do documento] \underline{blo blo blo}
\end{description}

Letras gregas são estas $ \alpha \beta \gamma \delta $ que aqui demonstro

Exemplo simples de fração \[ \frac{\frac{a * b + c}{4-3}}{3*5} \] simples

\section*{Estrutura do Relatório} 


Explicar como está¡ organizado o documento, referindo os capítulos existentes
em~\cite{yu09} e a sua articulação explicando o conteúdo de cada um.  No
capítulo \ref{ae} faz-se uma análise detalhada do problema proposto de modo
a poder-se especificar  as entradas, resultados e formas de transformação.\\
etc. \ldots


No capítulo~\ref{concl} termina-se o relatório com uma síntese do que foi dito,
as conclusões e o trabalho futuro

1     Objectivos e Organização
Este trabalho prático tem como principais objectivos:

    • aumentar a experiência de uso do ambiente Linux, da linguagem imperativa C (para codificação das estruturas
      de dados e respectivos algoritmos de manipulação), e de algumas ferramentas de apoio à programação;
    • aumentar a capacidade de escrever Expressões Regulares (ER) para descrição de padrões de frases;
    • desenvolver, a partir de ERs, sistemática e automaticamente Processadores de Linguagens Regulares, que filtrem
      ou transformem textos;
    • utilizar geradores de filtros de texto, como o Flex

Para o efeito, esta folha contém vários enunciados, dos quais deverá resolver pelo menos um.

