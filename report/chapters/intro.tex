\chapter*{Introdução}\label{intro}

\begin{description}
  \item [Enquadramento] 
  \item [Conteudo documento]
  \item [Resultados --- pontos a evidenciar]
  \item [Estrutura do documento]
\end{description}


\section*{Estrutura do Relatório} 


%Explicar como está organizado o documento, referindo os capítulos existentes
%em~\cite{yu09} e a sua articulação explicando o conteúdo de cada um.  No
%capitulo~\ref{ae} faz-se uma análise detalhada do problema proposto de modo
%a poder-se especificar  as entradas, resultados e formas de transformação.
%Etc. \ldots
%
%
%No capitulo~\ref{concl} termina-se o relatório com uma síntese do que foi dito,
%as conclusões e o trabalho futuro

1     Objetivos e Organização
Este trabalho pratico tem como principais objetivos:

		Aumentar a experiencia de uso do ambiente Linux, da linguagem imperativa
		C (para codificação das estruturas de dados e respetivos algoritmos de
		manipulação), e de algumas ferramentas de apoio a programação;
		Aumentar a capacidade de escrever Expressões Regulares (ER) para descrição
		de padrões de frases;
		Desenvolver, a partir de ERs, sistemática e automaticamente Processadores
		de Linguagens Regulares, que filtrem ou transformem textos;
    Utilizar geradores de filtros de texto, como o Flex


