\chapter*{Introdução}
\addcontentsline{toc}{chapter}{Introdução} 
\label{intro}


O presente documento visa a documentação do processo de aprendizagem de
especificações em \emph{Flex} e criação de filtros para diversos formatos de
texto. No contexto escolhido, criações de filtros para ficheiros
\hologo{BibTeX}. Numa primeira parte, o objetivo é a criação de uma filtro para
contabilizar as entradas bibliográficas e criar um ficheiro \emph{HTML} com
o resultado. Ficheiros deste género podem crescer muito em tamanho, por vezes
uma contabilização pode ajudar a construir repositórios documentais sobre
determinado assunto. De igual modo, a necessidade de normalização de um
documento deste é importante, porque com o passar do tempo, o acrescentar novas
entradas pode criar desvios de formatação, que pode prejudicar uma pesquisa no
documento pelo nome, por exemplo. Um exemplo de um normalizador deste género
figura na segunda parte (capítulo 2) deste documento.  Além do normalizador, na
terceira parte (cap. 3), é apresentado a elaboração de um filtro para fazer um
\emph{pretty-printing} de um documento \hologo{BibTeX}.  Por último, no quarto
capítulo é apresentada uma ferramenta para criar um grafo com os coautores de
determinado autor, onde os pesos das arestas são a densidade de publicação com
cada coautor.


\section*{Metas e objetivos} 

O ambiente Linux é uma das metas deste trabalho. Existem diversas formas de
fazer determinadas coisas para fazer em Linux, onde cada ferramenta tem o seu
lugar e não faltam ferramentas.  O domínio deste sistema operativo é importante,
dado que, um programador, com conhecimento suficiente sobre este sistema
operativo, tem uma liberdade que falta a outros. De igual modo, dado que as
ações das especificações no \emph{Flex} são programadas em linguagem C, um dos
objetivos passa por refinar e aumentar o conhecimento da linguagem, bem como
aprofundar o conhecimento em estruturas mais complexas. De igual modo,
algoritmos e complexidade são serão revisitados, e a otimização será algo
importante durante a elaboração do trabalho. Pretende-se, portanto, soluções
eficientes. Por último, o grande objetivo é desenvolver a capacidade de escrita
de \emph{ER's} e entender como o funcionamento de processadores de linguagens
regulares funcionam, usando geradores de filtros de texto como o \emph{Flex}.
Deste último objetivo será testemunha este documento.  


\section*{Estrutura do Relatório} 
O relatório está organizado em 4 capítulos: o primeiro capítulo é referente
à alínea \emph{a}, o segundo e terceiros capítulos correspondem à alínea
\emph{b} e o quarto capitulo corresponde à alínea \emph{c}.  Cada capítulo
possui três secções: \emph{Análise do Problema}, \emph{Desenho e implementação
da solução} e \emph{Testes e Resultados}.  Na secção \emph{Análise do Problema}
expõe-se informalmente o problema, contendo conteúdo referentes aos dados,
relações e possíveis abordagens à solução. Na secção \emph{Desenho
e implementação da solução} descrevem-se as especificações e respetivas ações do
\emph{Flex}, \emph{START CONDITION}, estruturas de dados e algoritmos e funções
importantes. Em seguida a secção \emph{Testes e Resultados} apresenta os
resultados e acrescenta uma breve discussão sobre problemas de implementação,
alternativas e sugestões.  Note-se que neste documento, para mostrar
a extensibilidade dos testes, muitos deles estão no \emph{Apêndice}. O documento
encerra com a \emph{Conclusão} onde se avaliará a completude dos objetivos, bem
como observações aos resultados obtidos.




