\chapter{Ferramenta \emph{pretty-printing} de um ficheiro \hologo{BibTeX}}
\label{chap:b2}

\section{Análise do Problema}
\label{sec:b2p:b2}
Nesta parte, o desafio é elaborar uma ferramenta de \emph{pretty-printing} que
indente corretamente cada campo, escreva um autor por linha e coloque sempre no
início os campos autor e título.

\section{Especificação dos requisitos}
\label{sec:spec:b2}

\subsection{Dados}

No \hologo{BibTeX}, cada tipo de entrada bibliográfica tem campos opcionais
e obrigatórios. Na resolução deste problema apenas se centrou nos obrigatórios,
uma vez que, se se quiser uma ferramenta genérica que inclua outros pacotes,
o número de campos é elevado. De igual modo, os campos obrigatórios podem ser
opcionais para algum tipo de entrada, e vice-versa. Em extensões, como no
\textsc{Bib}\LaTeX{}, alguns campos são comuns a outros campos, por uma questão de
compatibilidade. Por exemplo, campo autor e título são comuns à maioria dos
tipos de entrada, no caso da entrada \texttt{MISC} são opcionais. Omitiu-se, de
igual modo, os campos \texttt{note}, \texttt{abstract} entre outros, que podem
ter demasiada informação, para um \emph{pretty-printing}.

Os campos considerados foram:

\begin{itemize}


\item Organização
\item Forma de publicação\footnote{\emph{How Published}}
\item Instituição
\item Publicação
\item Titulo do livro
\item Jornal
\item Edição
\item Capitulo
\item Morada
\item Volume
\item Serie
\item Escola
\item Numero
\item Editor
\item Autor
\item Titulo
\item Págs
\item Mês
\item Ano
\item Tipo

\end{itemize}

Note-se que estes campos podem ser tanto obrigatórios como opcionais.
A ordem com que os campos estão colocados não tem a obrigatoriedade de uma
sequência. Por exemplo o campo autor pode ser colocado no final da enumeração
dos campos, sem nenhum efeito colateral. O que decide a ordem dos elementos
é sempre o estilo de bibliografia. 


\section{Desenho e implementação da solução}
\label{sec:des:b2}

A sugestão de solução apresentada, possui quatro \emph{START CONDITIONS}:
a \emph{SC} \texttt{ENTRY}, \emph{SC} \texttt{AUT}, a \emph{SC} \texttt{FIELD}
e, por último, a \texttt{SPEC}. A necessidade de ter as primeiras \emph{SC}
deve-se à necessidade de criar um contexto para tratar os autores, bem como
o título, sendo estas ativadas dentro \emph{SC} \texttt{ENTRY}. O intuito
\emph{SC} \texttt{SPEC} já foi descrita em secções anteriores --- capturar par
de chavetas, para evitar conflitos com as outras \emph{ER's}. De igual modo,
usaram-se três variáveis globais do tipo inteiro: uma para guardar o estado das
\emph{SC} no caso da \texttt{SPEC}, dois contadores, um para um índice de uma
\emph{string}, outro para guardar o índice de um \emph{array} de \emph{strings}
bi-dimensional. Além, destas variáveis de valor inteiro, criaram-se, como já
mencionado, uma \emph{string} e um \emph{array} bi-dimensional para guardar
o resultado das ocorrências dos campos. Criou-se este último, com o intuito de
guardar nas primeiras posições os valores do campos autor e título, seguido de
tudo o resto. Assim, após a captura do fim da entrada bibliográfica, o resultado
é passado ao \emph{stdout} pela ordem correta, iniciliazidando o índice do
\emph{array} bi-dimensional, de cada vez que é encontrada uma entrada
bibliográfica. Pr último, o \emph{array} é bi-dimensional dado que se pretende
guardar na primeira linha a legenda do campo e na segunda o valor do campo.
A especificação de cada \emph{SC}, que implementa estes conceitos segue-se nas
seguintes secções. Note-se que, que todas a \emph{START CONDITIONS} estão
ordenadas dentro do seu contexto, de  forma a não haver ambiguidade.

\subsection{Expressões Regulares}


\subsubsection{\emph{START CONDITION} \texttt{INITIAL}}

\begin{itemize}

	\item Captura o início de uma entrada, ou seja, um \texttt{@} seguido de uma
		ou mais caracteres maiúsculos ou minúsculos.

\begin{verbatim}
\@[A-Za-z]+
\end{verbatim}

Como já foi anteriormente mencionado, a captura deste valor através da
\emph{ER} descrita, despoleta uma ação, tal que a posição do \emph{array}
bi-dimensional é inciada logo na terceira posição, uma vez as duas primeiras
estão reservadas para o autor e o título. De igual modo, inicia a \emph{SC}
\texttt{ENTRY}, imprimindo no \emph{stdout} uma pequeno separador composto por
cardinais.

\item Captura tudo o resto, incluindo o caractere \emph{newline}.

\begin{verbatim}
(.|\n)                            
\end{verbatim}

A ação é ignorar tudo o que é capturado por esta \emph{ER}.



\end{itemize}

%%%%%%%%%%%%%%%%%%%%%%%%%%%%%%%%%%%%%%


\subsubsection{\emph{START CONDITION} \texttt{ENTRY}}

A \emph{SC} \texttt{ENTRY} está no contexto da entrada bibliográfica, onde os
vários campos são capturados. 

\begin{itemize}
\item Captura chaveta correspondente ao final da entrada bibliográfica.

\begin{verbatim}
[}] 
\end{verbatim}
Aqui é invocada a função de impressão dos campos pela ordem desejada, voltando
a \emph{SC} \texttt{INITIAL}.
\item Captura uma chaveta seguida de 0 ou mais espaços, com uma vírgula
	a seguir.

\begin{verbatim}
[}][ ]*[,]  
\end{verbatim}

A ação aqui definida é ignorar a captura. Este \emph{ER} serve de artifício
para não haver ambiguidade com o final dos campos, dado que a expressão
é maior. Note-se que, caso não se recorresse a este artifício, a captura dos
campos poderia terminar a meio.

\item Captura o campo \emph{Organization}, \emph{case-insensitive}, seguido 0 ou
	mais espaços, um \emph{"="}, com 0 ou mais espaços de seguida, podendo terminar
	ou não numa chaveta ou aspas.
\begin{verbatim}
[Oo][Rr][Gg][Aa][Nn][Ii][Zz][Aa][Tt][Ii][Oo][Nn][ ]*"="[ ]*[{"]? 
\end{verbatim}

Após a captura, o índice da \emph{string} auxiliar,  para armazenar
temporariamente o valor dos caracteres capturados, é inicializado, sendo guardada
a legenda do campo na primeira linha na coluna correspondente à posição da
ocorrência. Em seguida é inicializada a \emph{SC} \texttt{FIELD}, correspondente
ao contexto de manipulação de qualquer outro campo, que não o autor e o título.
                            
\item Captura similar a anterior, exceto a aspas ou chavetas de início do valor
	do campo são obrigatórias. Neste caso a captura é do campo autor.
\begin{verbatim}
[Aa][Uu][Tt][Hh][Oo][Rr][ ]*"="[ ]*[{"] 
\end{verbatim}

A ação é igual à de \emph{ER} anterior, com a exceção da posição do
\emph{array} que é utilizada diretamente, e é iniciada a \emph{SC} \texttt{AUT}.



\item Captura igual à anterior, só que neste caso, a captura é do campo do
	título. A \emph{SC} iniciada é a \texttt{TITLE}.
\begin{verbatim}
[Tt][Ii][Tt][Ll][Ee][ ]*"="[ ]*[{"] 
\end{verbatim}

Estas \emph{ER's} estão aqui a título exemplar, dado que existem \emph{ER's}
para cada campo obrigatório mencionado na secção \emph{Análise do Problema}.


\item Captura tudo o resto incluindo o caractere \emph{newline}, sendo a ação
	ignorar a captura.
\begin{verbatim}
(.|\n)
\end{verbatim}



\end{itemize}

%%%%%%%%%%%%%%%%%%%%%%%%%%%%%%%%%%%%%%

\subsubsection{\emph{START CONDITION} \texttt{AUT}}
Nesta \emph{SC} trata-se do campo do autor, de forma que cada autor apareça um
por linha e tenha a devida indentação.
\begin{itemize}
\item Captura um ou mais espaços ou tabulações, seguidas do separador
	\emph{and}, sendo seguidas novamente, por um ou mais espaços ou tabulações.
\begin{verbatim}
[ \t]+"and"[ \t]+
\end{verbatim}

Nesta captura é copiado para a \emph{string} auxiliar, uma \emph{string} com
o caractere \emph{newline}, para colocar um autor por linha, e três tabulações
e um espaço para dar a indentação pretendida. O valor do índice é incrementado
pelo o número de caracteres que é copiado, ou seja incrementado por 5.


\item Captura aspas ou chavetas que correspondem ao final do valor do campo do
	autor.
\begin{verbatim}
[}"] 
\end{verbatim}

Adiciona-se um \emph{newline} à \emph{string} auxiliar --- incrementando por um
o valor do seu índice --- e o valor da \emph{string}, já com todos os autores,
é copiada para a segunda linha, na coluna $0$ do \emph{array}
bi-dimensional. Após este passo volta para a \emph{SC} que gerou a \emph{SC}
\texttt{AUT}, neste caso \texttt{ENTRY}, 

\item Captura tudo o resto, incluindo o caractere \emph{newline}.
\begin{verbatim}
(.|\n)
\end{verbatim}

Nesta captura, o valor contido na variável do \emph{Flex}, \texttt{yytext}
é copiado para a \emph{string} auxiliar. Como a captura é caractere a caractere,
copia-se apenas o primeiro caractere desta \emph{string}. Como em anteriores
adições à \emph{string} auxiliar, a posição é incrementada.


\end{itemize}

%%%%%%%%%%%%%%%%%%%%%%%%%%%%%%%%%%%%%%
\subsubsection{\emph{START CONDITION} \texttt{TITLE}}

\begin{itemize}
\item Captura dois ou mais espaços.
\begin{verbatim}
[ ]{2,}
\end{verbatim}

Apenas serve para consumir espaços a mais, imprimindo no \emph{stdout} um espaço.

\item Captura qualquer chaveta aberta, que não a do fim de campo.
\begin{verbatim}
[{]
\end{verbatim}

A ação desta captura é devido ao que foi exposto em capítulos anteriores. No
entanto, como foi anteriormente explicado, vários contextos utilizam
a \emph{START CONDTION} \texttt{SPEC}. Assim, como em anteriores \emph{START
CONDTIONS} o valor do estado é guardado numa variável global, com a macro
\texttt{YYSTATE}. O chaveta é consumida.

\item Captura o final do campo caso seja aspas ou o fecho de chavetas.
\begin{verbatim}
[}"] 
\end{verbatim}
A ação é copiar um caractere \emph{newline} para a \emph{string} auxiliar, para
depois o valor desta \emph{string} ser copiado para a segunda linha e segunda
coluna do \emph{array} bi-dimensional. Após este passo, regressa à \emph{SC}
\texttt{ENTRY}.


\item Captura tudo o resto.
\begin{verbatim}
(.)
\end{verbatim}

Copia o valor da ocorrência para a \emph{string} auxiliar.

\item Captura do caractere \emph{newline}
\begin{verbatim}
(\n)
\end{verbatim}

A ação é ignorar a captura. Apenas serve para consumir eliminar quebras de linha no
campo do título.


\end{itemize}

%%%%%%%%%%%%%%%%%%%%%%%%%%%%%%%%%%%%%%
\subsubsection{\emph{START CONDITION} \texttt{FIELD}}

Na \emph{SC} \texttt{FIELD} é processado o conteúdo de todos os outros campos
mencionado na primeira secção deste capítulo, que não o autor e o título.
\begin{itemize}
	\item Captura campo com \emph{URL}.  
\begin{verbatim}
"\\url{"
\end{verbatim}

À semelhança do descrito noutros capítulos, a \emph{tag}, começa com aspas tendo
que terminar em aspas. Daí que se tenha usado o conceito da do capítulo anterior
para capturar esta \emph{tag}. A ação tem por base consumir o valor desta
captura, e iniciar a \emph{SC} \texttt{SPEC}, guardando o estado da \emph{SC} no
processo da ação. 

\item Captura dois ou mais espaços.
\begin{verbatim}
[ ]{2,}
\end{verbatim}

O procedimento associado a esta captura é simplesmente acrescentar um espaço
à \emph{string} auxiliar, consumindo os espaços em excesso. 

\item Captura a abertura de chavetas.
\begin{verbatim}
[{]
\end{verbatim}

À semelhança da \emph{ER} equivalente na \emph{SC} \texttt{TITLE}, na captura
desta \emph{ER}, guarda-se o estado da \emph{SC} atual para iniciar
a \emph{SC} \texttt{SPEC}, pelas mesmas razões.     

\item Captura opcional de um fecho de chavetas ou aspas, seguido de 0 ou mais
	espaços, terminados por uma vírgula. 
\begin{verbatim}
[}"]?[ ]*[,]
\end{verbatim}

Existem campos cujo valor não está delimitado nem por aspas nem por chavetas.
Assim, é necessário definir o fim do valor de um campo de outra forma, ou seja,
o campo tem que terminar por vírgula. No entanto, isto nem sempre acontece,
porque, caso o campo seja o último há definir outra forma de o campo acabar.
Esse é o intuito da \emph{ER} seguinte. O processo associado a esta \emph{ER}
é uma cópia de caractere \emph{newline} para a \emph{string} auxiliar, sendo
feita a cópia da \emph{string} auxiliar para o \emph{array} bi-dimensional, na
segunda linha, na índice equivalente à contagem das ocorrências dos campos. No
fim volta para a \emph{SC} \texttt{ENTRY}.  

\item Captura opcional de um fecho de chavetas ou aspas, seguido de 0 ou mais
	espaços, com um \emph{newline} opcional, seguido 0 ou mais espaços, terminado
	por uma chaveta. 
\begin{verbatim}
[}"]?[ ]*(\n)?[ ]*[}]
\end{verbatim}

Esta \emph{ER} define a outra forma de de um campo terminar. Isto é quando
o campo é o último. Note-se que, a chaveta que termina esta \emph{ER} tem um
tratamento no contexto da \emph{ER} \texttt{ENTRY} --- o fim da entrada
bibliográfica. Deste modo, é necessário uma manobra adicional: colocar
o \emph{Flex} na posição para reler a chaveta no contexto correto. Para isso
usa-se o \texttt{yyless (yyleng-1)}. Esta função é invocada antes de voltar para
\emph{SC} \texttt{ENTRY}. O resto do processo é igual ao da especificação
anterior.  

\item Captura tudo, exceto o caractere \emph{newline}. 
\begin{verbatim}
(.)
\end{verbatim}

Ação: copiar a ocorrência do caractere da captura para \emph{string} auxiliar. 

\item Captura o caractere \emph{newline}.  
\begin{verbatim}
(\n)
\end{verbatim}

A operação associada é ignorar, ou seja, a \emph{string} auxiliar não terá
quebras de linha. 

\end{itemize}
%%%%%%%%%%%%%%%%%%%%%%%%%%%%%%%%%%%%%
\subsubsection{\emph{START CONDITION} \texttt{SPEC}}

\begin{itemize}
\item Captura uma chaveta aberta. 

\begin{verbatim}
[}]
\end{verbatim}

A ação é similar a contextos descritos em capítulos anteriores --- caso o fecho
de uma chaveta seja detetado neste contexto, coloca o \emph{Flex} na posição de
releitura dessa chaveta para ser processado pelo contexto correspondente.
Chama-se à atenção de que se usou a variável global de estado descrita
anteriormente, para iniciar um novo contexto. Isto acontece, uma vez que tanto
títulos e outros campos podem ter uma abertura de chavetas no meio do valor do
campo, daí do uso de uma variável global.

\item Captura tudo o resto.
\begin{verbatim}
(.)
\end{verbatim}

Na captura, ação é guardar o valor na \emph{string} auxiliar, ignorando
a chaveta. 


\item Captura o caractere \emph{newline}. 
\begin{verbatim}
(\n)
\end{verbatim}

Mantém-se a intenção de não ter quebras de linhas dentro dos campos, que não
sejam autor. 

\end{itemize}


\subsection{Algoritmos}

A função abaixo é a função utilizada para imprimir os valores guardados no
\emph{array} bi-dimensional. A impressão vai até ao total de posições
incrementadas na captura dos campos. Assim, não é necessário alocar memória para
o \emph{array} 2D. 

\begin{verbatim}
void print_campos()
{
    int j;

    for(j = 0; j < pos; j++){
        printf("%s %s", fields[0][j], fields[1][j]);
      free(fields[0][j]);
      free(fields[1][j]);
      fields[0][j]=NULL;
      fields[1][j]=NULL;
    }

}
\end{verbatim}

Na \texttt{main} todas as entradas não nulas são libertadas da memória. 
\begin{verbatim}
int j;

    yylex();
 for(j = 0; j < MAX_ENTRIES; j++)
        {
            if(fields[0][j])
                free(fields[0][j]);
            else if (fields[1][j])
                free(fields[1][j]);

        }

\end{verbatim}

\section{Testes e Resultados}
\label{sec:ts:b2}

\subsection{Resultados}

Os resultados da aplicação deste filtro ao ficheiro no
apêndice~\ref{appendix:a1} na pág.~\pageref{appendix:a1}, estão no
apêndice~\ref{appendix:c} na pág.~\pageref{appendix:c}. 

\subsection{Alternativas, Decisões e Problemas de Implementação}

Houve diversos problemas na implementação dado à quantidade de exceções que
podem ocorrer num ficheiro \hologo{BibTeX} e a intenção de usar uma solução
completamente "nativa" do \emph{Flex}, isto é, apenas com \emph{ER's} e funções do
\emph{Flex}. Logo, de forma a simplificar o problema, não se está a considerar
caracteres especiais, nem escape de caracteres. Até à data de redação deste
documento
desconhece-se uma solução "nativa" que vá de encontro à especificação do enunciado
--- ordem pré-definida dos campos na apresentação. No entanto houve várias
tentativas, onde uma solução em particular fazia o \emph{pretty-printing}, no
entanto, ter-se-ia que assumir que o utilizador do ficheiro \hologo{BibTeX} colocava os campos
autor e titulo em primeiro lugar. Esta solução não vai de encontro
à especificação do enunciado.

Numa melhoria futura, acrescentar suporte para caracteres especiais à semelhança
de capítulos anteriores.

