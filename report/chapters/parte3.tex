\chapter{Ferramenta \emph{pretty-printing} de um ficheiro \hologo{BibTeX}}
\label{chap:b2}

\section{Análise do Problema}
\label{sec:b2p:b2}
Nesta parte, o desafio é elaborar uma ferramenta de \emph{pretty-printing} que
indente corretamente cada campo, escreva um autor por linha e coloque sempre no
início os campos autor e título.

\section{Especificação dos requisitos}
\label{sec:spec:b2}

\subsection{Dados}

No \hologo{BibTeX}, cada tipo de entrada bibliográfica tem campos opcionais
e obrigatórios. Na resolução deste problema apenas se centrou nos obrigatórios,
dado que se quiser uma ferramenta genérica, que inclua outros pacotes, o número
de campos é elevado. De igual modo, os campos obrigatórios podem ser opcionais
para algum tipo de entrada, e vice---versa. Em extensões, como no
\textsc{Bib}\LaTeX{}, alguns campos são são comuns, por uma questão de compatibilidade.

Os campos considerados são:

\begin{itemize}


\item Organização
\item \emph{How Published}
\item Instituição
\item Publicação
\item Titulo do livro
\item Jornal
\item Edição
\item Capitulo
\item Morada
\item Volume
\item Serie
\item Escola
\item Numero
\item Editor
\item Autor
\item Titulo
\item Págs
\item Mês
\item Ano
\item Tipo

\end{itemize}

Note-se que estes campos podem ser tanto obrigatórios como opcionais.
A ordem com que os campos estão colocados não tem a obrigatoriedade de uma
sequência. Por exemplo o campo autor pode ser colocado no final da enumeração
dos campos, sem nenhum efeito colateral. O que decide a ordem dos elementos
é sempre o estilo de bibliografia.


\section{Desenho e implementação da solução}
\label{sec:des:b2}

A sugestão de solução apresentada, possui quatro \emph{START CONDITIONS}:
a \emph{SC} \texttt{ENTRY}, \emph{SC} \texttt{AUT}, a \emph{SC} \texttt{FIELD}
e, por último, a \emph{SPEC}. A necessidade de ter as primeiras \emph{SC}
deve-se à necessidade de criar um contexto para tratar os autores, bem como
o título, sendo estas ativadas dentro \emph{SC} \texttt{ENTRY}. O intuito
\emph{SC} \texttt{SPEC} já foi descrita em secções anteriores --- capturar par
de chavetas, para evitar conflitos com as outras \emph{ER's}. De igual modo,
usaram-se três variáveis globais do tipo inteiro: uma para guardar o estado das
\emph{SC} no caso da \texttt{SPEC}, dois contadores, um para um índice de uma
\emph{string}, outro para guardar o índice de um \emph{array} de \emph{strings}
multidimensional. Além, destas variáveis de valor inteiro, criaram-se, como já
mencionado, uma \emph{string} e um \emph{array} multidimensional para guardar
o resultado das ocorrências dos campos. Criou-se este último, com o intuito de
guardar nas primeiras posições os valores do campos autor e título, seguido de
tudo o resto. Assim, após a captura do fim da entrada bibliográfica, o resultado
é passado ao \emph{stdout} pela ordem correta, iniciando o índice do
\emph{array} multidimensional de cada vez que é encontrada uma entrada
bibliográfica. Também, o \emph{array} é multidimensional dado que intenciona
guardar na primeira linha a legenda do campo e na segunda o valor do campo.
A especificação de cada \emph{SC}, que implementa estes conceitos segue-se nas
seguintes secções.


\subsubsection{\emph{START CONDITION} \texttt{INITIAL}}

\begin{itemize}

	\item Captura o início de uma entrada, ou seja, um \texttt{@} seguido de uma
		ou mais caracteres maiúsculos ou minúsculos.

\begin{verbatim}
\@[A-Za-z]+
\end{verbatim}

Como já foi anteriormente mencionado, a captura deste valor através da
\emph{ER} descrita, despoleta uma ação, tal que a posição do \emph{array}
multidimensional é inciada logo na terceira posição, uma vez as duas primeiras
estão reservadas para o autor e o título. De igual modo, inicia a \emph{SC}
\texttt{ENTRY}, imprimindo no \emph{stdout} uma pequeno separador composto por
cardinais.

\item Captura tudo o resto, incluindo o caractere \emph{newline}.

\begin{verbatim}
[(.|\n)                            
\end{verbatim}

A ação é ignorar o restante.



\end{itemize}

%%%%%%%%%%%%%%%%%%%%%%%%%%%%%%%%%%%%%%


\subsubsection{\emph{START CONDITION} \texttt{ENTRY}}

A \emph{SC} \texttt{ENTRY} está no contexto da entrada bibliográfica, onde os
vários campos são capturados. 

\begin{itemize}
\item Captura chaveta correspondente ao final da entrada bibliográfica.

\begin{verbatim}
[}] 
\end{verbatim}
Aqui é invocada a função de impressão dos campos pela ordem desejada, voltando
a \emph{SC} \texttt{INITIAL}.
\item Captura uma chaveta seguida de 0 ou mais espaços, com uma vírgula
	a seguir.

\begin{verbatim}
[}][ ]*[,] {;} 
\end{verbatim}

A ação aqui definida é ignorar a captura. Este \emph{ER} serve de artifício
para não haver ambiguidade com o final dos campos, dado que a expressão
é maior. Note-se que, caso não se recorresse a este artifício, a captura dos
campos poderia terminar a meio.

\item Captura o campo \emph{Organization}, \emph{case-insensitive}, seguido 0 ou
	mais espaços, um \emph{=}, com 0 ou mais espaços de seguida, podendo terminar
	ou não numa chaveta ou aspas.
\begin{verbatim}
[Oo][Rr][Gg][Aa][Nn][Ii][Zz][Aa][Tt][Ii][Oo][Nn][ ]*"="[ ]*[{"]? 
\end{verbatim}

Após a captura , o índice da \emph{string} auxiliar para armazenar
temporariamente o valor dos caracteres capturados é inicializado, sendo guardada
a legenda do campo na primeira linha, na coluna correspondente à posição da
ocorrência. Em seguida é inicializado a \emph{SC} \texttt{FIELD}, correspondente
ao contexto de manipulação de qualquer outro campo, que não o autor e o título.
                            
\item Captura similar a anterior, exceto a aspas ou chavetas de início do valor
	do campo são obrigatórias. Neste caso a captura é do campo autor.
\begin{verbatim}
ENTRY>[Aa][Uu][Tt][Hh][Oo][Rr][ ]*"="[ ]*[{"] 
\end{verbatim}

A ação é igual à de \emph{ER} anterior, com a exceção da posição do
\emph{array} que é utilizada diretamente, e é iniciada a \emph{SC} \texttt{AUT}.



\item Captura igual à anterior, só que neste caso, a captura é do campo do
	título. A \emph{SC} iniciada é a \texttt{TITLE}.
\begin{verbatim}
<ENTRY>[Tt][Ii][Tt][Ll][Ee][ ]*"="[ ]*[{"] 
\end{verbatim}

Estas \emph{ER's} estão aqui a título exemplar, dado que existem \emph{ER's}
para cada campo obrigatório mencionado na secção \emph{Análise do Problema}.


\item Captura tudo o resto incluindo o caractere \emph{newline}, sendo a ação
	ignorar a captura.
\begin{verbatim}
(.|\n)
\end{verbatim}



\end{itemize}

%%%%%%%%%%%%%%%%%%%%%%%%%%%%%%%%%%%%%%

\subsubsection{\emph{START CONDITION} \texttt{AUT}}

\begin{itemize}
\item 
\begin{verbatim}
[ \t]+"and"[ \t]+
\end{verbatim}
 {strcpy(value+i, "\n\t\t\t "); i+=5;}

\item 
\begin{verbatim}
<AUT>[}"][ ]*[ ]? 
\end{verbatim}
                                   strcpy(value+i,"\n"); i++; 
                                   fields[1][0]=strdup(value); 
				                   BEGIN ENTRY;}

\item 
\begin{verbatim}
(.|\n)
\end{verbatim}

{value[i++]=yytext[0];}


\end{itemize}

%%%%%%%%%%%%%%%%%%%%%%%%%%%%%%%%%%%%%%
\subsubsection{\emph{START CONDITION} \texttt{TITLE}}

\begin{itemize}
\item Captura dois ou mais espaços.
\begin{verbatim}
[ ]{2,}
\end{verbatim}

Apenas serve para consumir espaços a mais, imprimindo no \emph{stdout} um espaço.

\item Captura qualquer chaveta aberta, que não a do fim de campo.
\begin{verbatim}
[{]
\end{verbatim}

A ação desta captura é devido ao que foi exposto em capítulos anteriores. No entanto, como foi anteriormente explicado, vários contextos utilizam a \emph{START CONDTION} \texttt{SPEC}. Assim, como em anteriores \emph{START CONDTIONS} o valor do estado é guardado numa varrável global, com a macro \texttt{YYSTATE}. O chaveta é consumida.

\item Captura o final do campo caso seja aspas ou o fecho de chavetas.
\begin{verbatim}
[}"][ ]*[,]? 
\end{verbatim}

\item Captura tudo o resto.
\begin{verbatim}
(.)
\end{verbatim}
{value[i++]=yytext[0];}

\item Captura do caractere \emph{newline}
\begin{verbatim}
(\n)
\end{verbatim}

A ação é não fazer nada. Apenas serve para consumir eliminar quebras de linha no campo do título.


\end{itemize}

%%%%%%%%%%%%%%%%%%%%%%%%%%%%%%%%%%%%%%
\subsubsection{\emph{START CONDITION} \texttt{FIELD}}

\begin{itemize}
\item 
\begin{verbatim}
"\\url{"
\end{verbatim}
 { state_caller = YYSTATE; BEGIN SPEC;}

\item 
\begin{verbatim}
[ ]{2,}
\end{verbatim}

\item 
\begin{verbatim}
[{]
\end{verbatim}

\item 
\begin{verbatim}
[}"]?[ ]*[,]
\end{verbatim}
strcpy(value+i,"\n"); i++;
				                   fields[1][pos++]=strdup(value);
				                   BEGIN ENTRY;}

\item 
\begin{verbatim}
(.)
\end{verbatim}
 {value[i++]=yytext[0];}

\item 
\begin{verbatim}
(\n)
\end{verbatim}

\end{itemize}
%%%%%%%%%%%%%%%%%%%%%%%%%%%%%%%%%%%%%
\subsubsection{\emph{START CONDITION} \texttt{SPEC}}

\begin{itemize}
\item 

\begin{verbatim}
[}]
\end{verbatim}
yyless(0);BEGIN state_caller;

\item 
\begin{verbatim}
(.|\n)
\end{verbatim}
{value[i++]=yytext[0];}

\end{itemize}



\subsection{Estruturas de dados}

\subsection{Algoritmos}

\begin{verbatim}
void print_campos()
{
    int j;

    for(j = 0; j < pos; j++){
        printf("%s %s", fields[0][j], fields[1][j]);
    	free(fields[0][j]);
    	free(fields[1][j]);
    	fields[0][j]=NULL;
    	fields[1][j]=NULL;
    }

}
\end{verbatim}


\begin{verbatim}
int j;

    yylex();
 for(j = 0; j < MAX_ENTRIES; j++)
        {
            if(fields[0][j])
                free(fields[0][j]);
            else if (fields[1][j])
                free(fields[1][j]);

        }

\end{verbatim}

\section{Codificação e Testes}
\label{sec:ts:b2}

\subsection{Alternativas, Decisões e Problemas de Implementação}

\subsection{Testes e Resultados}
%Mostram-se a seguir alguns testes feitos (valores introduzidos) e
%%os respetivos resultados obtidos:
