\chapter*{Conclusão}
\addcontentsline{toc}{chapter}{Conclusão}
\refstepcounter{chapter}
\label{concl}
Neste documento apresentaram-se as propostas de solução para quatro filtros: um
contador de entradas bibliográficas, um filtro para normalizar um documento,
formatando os nomes dos autores em \emph{N. Apelido} e remoção de chavetas, como
também uma ferramenta de \emph{pretty-printing} e o gerador de grafos em
\emph{Dot} para coautores de um determinado autor e respetiva densidade de
publicação. 

Em relação à aprendizagem do Linux, usaram-se diversas ferramentas para
obtenção de dados e editores de texto, tal como \texttt{grep}, \texttt{sort},
\emph{vim}, \texttt{cat}, \texttt{tac}, \texttt{sed}, entre outros. Um ponto
comum em todos eles foram as \emph{ER} que ampliaram a visão sobre como
trabalhar em Linux.

De igual modo, a utilização das especificações do \emph{Flex} permitiu perceber
melhor as \emph{ERs} e a revisita da linguagem C ajudou a limar algumas arestas.  

A utilização das estruturas poderia ter sido melhor. A escolha da \emph{uthash}
para utilização neste projeto é adequada à sua dimensão, no entanto, existem
estruturas mais eficientes para determinados filtros criados.

De um modo geral, os resultados foram animadores, uma vez que se conseguiu
cumprir com os requisitos do trabalho, de uma forma eficiente. Muito
provavelmente existem melhorias a ser feitas ou algo está parcialmente correta,
ou não o está de todo, no entanto, não se tem conhecimento, dados os testes
efetuados de situações em que os filtros falharam.

Futuramente, poder-se-á estender o projeto a caracteres especiais e tratamento
de caracteres com escape. Uma análise rigorosa poderá por em evidência
alternativas de solução mais eficientes ou mais simples. Testes com outras
estruturas poderão ser efetuados bem como \emph{benchmarking} de cada estrutura
nova aplicada.



