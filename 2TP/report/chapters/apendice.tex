\appendix

\part*{ANEXOS}
\addcontentsline{toc}{part}{ANEXOS}
\refstepcounter{part} 

\chapter{Gramática para linguagem criada}\footnote{A gramática está na notação
	\emph{Bauk-Naur Form}}
\label{appendix:a}
\begin{grammar}

<Program> ::= <Declarations> <Body> 

<Body> ::= `BEGIN' <InstructionList> `END'

<Declaration> ::= <id>
\alt <id> `[' <num> `]'
\alt <id> `[' <num> `]' `[' <num> `]' 

<Declarations> ::= `VAR' <DeclarationsList> `;' 

<DeclarationsList> ::= <Declaration> 
\alt <DeclarationsList> `,' <Declaration> 

<Constant> ::= <num>


<Factor> ::= <Constant>
\alt <Variable>
\alt <id> `[' <ExpAdditiv> `]' `[' <ExpAdditiv> `]'
\alt `(' <Exp> `)'
\alt `NOT' <Exp>

<Variable> ::= <id>
\alt <id> `[' <ExpAdditiv> `]'
\alt <id> `[' <ExpAdditiv> `]' `[' <ExpAdditiv> `]' 
 
<Term> ::= <Factor>
\alt <Term> `*'  <Factor>
\alt <Term> `/' <Factor>
\alt <Term> `%' <Factor>
\alt <Term> `AND' <Factor>

<ExpAdditiv> ::= <Term> 
\alt <ExpAdditiv> `+' <Term>
\alt <ExpAdditiv> `-' <Term> 
\alt <ExpAdditiv> `OR' <Term> 


<Exp> ::= <ExpAdditiv>             
\alt  <ExpAdditiv> `>'  <ExpAdditiv> 
\alt  <ExpAdditiv> `<'  <ExpAdditiv> 
\alt  <ExpAdditiv> `>=' <ExpAdditiv> 
\alt  <ExpAdditiv> `<=' <ExpAdditiv> 
\alt  <ExpAdditiv> `==' <ExpAdditiv> 
\alt  <ExpAdditiv> `!=' <ExpAdditiv> 

<Else> ::= <>
\alt `ELSE' '{'<InstructionsList> '}'

<Atribution> ::=  <Variable> `=' <ExpAdditiv> 

<Instruction> ::= <Atribution> `;' 
\alt `READ'  <Variable> `;'
\alt `WRITE' <ExpAdditiv> `;'                      
\alt `WRITE' <string> `;'
\alt `IF' `(' <Exp> `)' `{' <InstructionsList> `}' <Else>
\alt `WHILE `(' <Exp> `)' `{' <InstructionsList> `}' 
\alt `DO'`{' <InstructionsList> `}'`WHILE `(' <Exp> `)' `;' 

<InstructionList> ::= <Instruction>
\alt <InstructionList> <Instruction>  

\end{grammar}

\chapter{Código do analisador léxico do \emph{Flex}}
\label{appendix:b}
%\begin{longlisting}
%	\inputminted{html}{testes/res_html.html}
%	\caption{Resultado do \emph{output} da aplicação do filtro na Parte 1}
%	\label{listing:b}
%\end{longlisting}

\chapter{Código do analisador sintático e tradutor do \emph{YACC}}
\label{appendix:c}
%\begin{longlisting}
%	\inputminted{html}{testes/res_html.html}
%	\caption{Resultado do \emph{output} da aplicação do filtro na Parte 1}
%	\label{listing:c}
%\end{longlisting}

\chapter{Código gerado a partir do tradutor do \emph{YACC}, para os exemplos
pedidos}
\label{appendix:d}
\section{Maior de três números}
\label{appendix:d:sec:d1}
VAR x, y, max;


BEGINNING

	WRITE "escreva x:";
	READ x;
	WRITE "escreva y:";
	READ y;
	
	IF ( x > y )
	{
	  max = x;
	
	}
	ELSE
	{
	  max = y;
	
	}
	
	WRITE "O maior numero e:";
	WRITE max;
	
END



pushi 0
pushi 0
pushi 0
start
jz l1level1	
pushs "escreva x:"
writes
pushg 0
pushg 0
read
pushs "escreva y:"
writes
pushg 1
pushg 1
read
pushg 2
pushg 0
storeg 2
jump l2level1
l1level1:nop
pushg 2
pushg 1
storeg 2
l2level1:nop
pushs "O maior numero e:"
writes
pushg 2
writei
stop


\section{Somatório de \texttt{N} números}
\label{appendix:d:sec:d2}
VAR n, current, sum, counter;

BEGINNING 

	WRITE "Escreva o total de numeros :";
	READ n;

	current = 0;
	sum = 0;
	counter = 0;

	WHILE (counter < n)
	{
	WRITE "Escreva um numero:";
	READ current;
	sum = sum + current;

	counter = counter +1;
	
	}
	WRITE "O valor total da soma e:"
	WRITE sum;

END




\section{Sequência de pares de \texttt{N} números dados}
\label{appendix:d:sec:d3}
VAR current, counter;

BEGINNING

	current = 1;
	counter = 0;


	WHILE (current != 0)
	{

	WRITE "Escreva um numero:";
	READ current;

	  IF(current % 2 == 0)
	  {
	   WRITE current;
	   counter = counter +1;
	  }




	}

	WRITE "Total de numeros pares lidos:";
	WRITE counter;

END



pushi 0
pushi 0
start
whileloop1:nop
pushg 0
pushi 0
equal
not
jz whiledone1
jz l1level1	
pushg 0
pushi 1
storeg 0
pushg 1
pushi 0
storeg 1
pushs "Escreva um numero:"
writes
pushg 0
pushg 0
read
pushg 0
writei
pushg 1
pushg 1
pushi 1
add  
storeg 1
l1level1:nop
jump whileloop1
whiledone1:nop
pushs "Total de numeros pares lidos:"
writes
pushg 1
writei
stop


\section{Ordenação de \emph{array} de tamanho \texttt{N}--\emph{Insertion Sort}}
\label{appendix:d:sec:d4}
VAR a[20], pos, n, x;

BEGINNING
	pos = 0;
	READ n;

	WHILE (pos < n)
	{
	READ x;
	a[pos] = x;

	pos = pos +1;
	
	}

	pos = 1;


	WHILE ( pos < n )
	{
	IF (a[pos - 1] <= a[pos])
	{
	pos = pos +1;
	
	}
	ELSE
	{
	   x = a[pos];
	   a[pos] = a[pos - 1];
	   a[pos - 1] = x;

	   pos = pos - 1;  
              
    	IF( pos == 0)
        {
     		pos = 1;
        }
  
	}
	
	}
        pos = 0;
	WHILE ( pos < n )
	{
        WRITE a[pos];	

	}
END


\section{Média e máximo de uma matriz \texttt{[N][M]}}
\label{appendix:d:sec:d5}

VAR matriz [10] [5], media, restomedia, total, max, i, j, n , m, tmp, cont;

BEGINNING
	max = 0;
	media = 0;
	total = 0;
	tmp = 0;
	cont = 0;
        WRITE "escreva o numero de linhas :"
	READ n;
        WRITE "escreva o numero de colunas :"
	READ m;

	i = 0;

	WHILE (i < n)
	{
	
	        j = 0;
		WHILE(j < m)
		{
	         READ tmp;
                 
		 matriz[i][j] = tmp;
		 total = total + tmp;
		 cont = cont + 1;

		IF(tmp > max)
		{
		max  = tmp;
				       
		}
			
                j = j + 1;		
		}
	
	i = i + 1;
	}

	media = total/cont;
	restomedia = total \% cont;

	WRITE "A media tem um valor de :";
	WRITE media;
	WRITE "o resto tem um valor de :";
	WRITE restomedia;
	WRITE "O maximo tem um valor de :";
	WRITE max;
END


%\begin{longlisting}
%	\inputminted{html}{testes/res_html.html}
%	\caption{Resultado do \emph{output} da aplicação do filtro na Parte 1}
%	\label{listing:d5}
%\end{longlisting}


\section{Cálculo de uma potência com determinada base e expoente--função
	\texttt{potencia(Base,Exp)}}
\label{appendix:d:sec:d6}

%\begin{longlisting}
%	\inputminted{html}{testes/res_html.html}
%	\caption{Resultado do \emph{output} da aplicação do filtro na Parte 1}
%	\label{listing:d6}
%\end{longlisting}


