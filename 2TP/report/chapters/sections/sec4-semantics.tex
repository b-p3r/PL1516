\subsection{Análise Semântica Estática}
\label{subsec:semantica:desenho}

A análise semântica estática complementa a análise lexical e sintática, pois
existem situações em que a apesar de a análise léxica e sintática estarem
corretas, as sequências de símbolos não têm sentido. A título de exemplo, a data
\texttt{2016-45-00} está sintaticamente correta pois segue o formato
\texttt{aaaa-mm-dd}, mas não tem sentido do ponto de vista semântico.

Na \emph{LPIS} existem situações em que a gramática e análise léxica não são
suficientes, nomeadamente na verificação de tipos, ou seja, se estes são
consistentes na sua definição.  Anteriormente, é mencionada a inferência de
tipos em fatores, variáveis, arrays, aditivos, termos, expressões e expressões
relacionais. Assim é necessário não só verificar os elementos de cada operação
binária, como inferir o tipo do seu resultado.  Como já foi mencionado, esta
\emph{LPIS} apenas permite valores inteiros, mas no entanto o resultado das
expressões lógicas e relacionais são do tipo booleano. Por uma questão de
consistência, considera-se que as instruções terão o tipo \texttt{Any}.

Assim, as instruções assumem o tipo \texttt{Any}, variáveis atómicas e arrays
o tipo \texttt{Integer} (exceto subprogramas sem valor de retorno),
e o resultado de uma operação aditiva entre fatores como variáveis e arrays
assumirá o tipo \texttt{Integer}, bem como o tipo de cada de cada membro da
operação binária aditiva. O resultado de uma operação multiplicativa entre
termos (resultantes da operação aditiva) também tomará o tipo \texttt{Integer},
bem como ambos os membros da operação binária em questão. Seguidamente,
o resultado de expressões relacionais deverá ser do tipo \texttt{Boolean}. No
entanto, os membros desta operação binária assumirão o tipo \texttt{Integer}.
Para concluir, o resultado de uma expressão lógica assumirá o tipo
\texttt{Boolean} e ambos os membros da operação binária assumem também o tipo
\texttt{Boolean}.

Para além da verificação de tipos, a análise semântica deve assegurar
a existência de etiquetas, ou \texttt{labels}, de referência no resultado da
geração de código. Dado que a linguagem em causa tem quatro tipos de estruturas
de controlo de fluxo, existirá um mecanismo que cria as \texttt{labels} para
cada tipo de estrutura de controlo e que permita o aninhamento das mesmas. Ou
seja, as etiquetas devem ter a referência do nível em que estão, e devem ser
colocadas no seu devido lugar.  Posteriormente, será descrito o algoritmo
e a implementação do mecanismo de criação de \texttt{labels}.

Adicionalmente, as estruturas de controlo devem ser verificadas para confirmar
a sua devida utilização.  Por exemplo, se existem \texttt{breaks} fora de um
\texttt{loop} ou \texttt{switch}. 

Finalmente, é necessário considerar outro tipo de análise semântica: a análise
semântica dinâmica.  Ao contrário da anterior, este tipo de análise não
é efetuada em tempo de compilação, mas sim em tempo de execução. Existem
exemplos retirados da gramática da linguagem alvo deste projeto, que não serão
considerados, visto que a máquina virtual VM já os inclui, como é o caso da
divisão por zero, em que os valores retirados da pilha não têm o tipo esperado,
ou acessos indefinidos a uma ? de código. Na especificação léxica da
linguagem, definiu-se que todos os números tomarão valores não negativos,
e previu-se a existência de números negativos na gramática  a uma operação
unária?.  Deste modo não é necessária a especificação semântica estática de
declarações de arrays com tamanho zero, resultados negativos em cálculos de
índices e divisões por zero, uma vez que estes casos fazem parte da análise
semântica dinâmica.



\subsection{Análise Semântica Estática}
\label{subsec:semantica:desenho}


Program : Declarations Body   

Nesta produção é imprimido no \emph{stdout}, o resultado dos blocos que vêm de
baixo  da árvore sintática, através de \emph{Body}. 



Body : BEGINNING 

Esta produção tem uma regra intermédia, em que imprime no \emph{stdout}
a \emph{string} `start'. A ação é para ser executada de imediato, aquando do
reconhecimento da palavra reservada.


InstructionsList END      

O restante da produção, junta tudo dos blocos construídos a partir de baixo da
árvore sintática, concatenando a \emph{string} `stop' no fim.\ O resultado vai
para cima da árvore sintática.


\subsection{Declarações}
Declaration : id              
Nesta produção é verificado se o identificador não existe. Se não existe
o identificador é adicionado à tabela de identificadores, com o o tipo, classe
(neste caso \emph{Variable}) e com o nível \emph{Program}. A \emph{string}
`pushi 0' é impressa no \emph{stdout}.   



| id '[' num ']'              
Nesta produção é verificado se o identificador não existe. Se não existe
o identificador é adicionado à tabela de identificadores, com o o tipo, classe
(neste caso \emph{Array}) e com o nível \emph{Program} e com o seu tamanho. A \emph{string}
`pushn' concatenada com o tamanho é impressa no \emph{stdout}.   

   
| id '[' num ']' '[' num ']'  
Nesta produção é verificado se o identificador não existe. Se não existe
o identificador é adicionado à tabela de identificadores, com o o tipo, classe
(neste caso \emph{Matrix}) e com o nível \emph{Program} e com o seu tamanho,
e o número de linhas. A \emph{string}
`pushn' concatenada com o tamanho é impressa no \emph{stdout}.   

  

Declarations : VAR DeclarationsList ';'  

Nesta produção o valor do bloco construído de baixo da árvore é passado para
cima.


  

DeclarationsList : Declaration             

Nesta produção o valor do bloco construído de baixo da árvore é passado para
cima.



| DeclarationsList ',' Declaration         

Nesta produção o valor do bloco construído de baixo da árvore é passado para
cima, após juntar o que vem do lado da mão direita com o que vem do lado da mão
esquerda.


Variable : id                 

Nesta produção é verificado se o identificador existe. Se existe
o endereço é obtido da tabela de identificadores. A \emph{string}
`pushg' concatenada com o endereço é impressa é juntada ao bloco.
O tipo inteiro é passado para o nível acima da árvore sintática.
Caso contrário, é lançado um erro.


| id '[' ExpAdditiv ']'       

Nesta produção é verificado se o identificador existe. Se existe
o endereço é obtido da tabela de identificadores. A \emph{string}
`pushgp' concatenada com o enedereço, mais a \emph{string} `padd' é junta ao
bloco..
Caso contrário, é lançado um erro.


| id '[' ExpAdditiv ']' '[' ExpAdditiv ']' 

Nesta produção é verificado se o identificador existe. Se existe
o endereço é obtido da tabela de identificadores. A \emph{string}
`pushgp' concatenada com o enedereço, mais as \emph{strings} `padd', onde o que
vem das espressões é adicionado ao bloco, seguido de `pushi'com o número de
linhas seguido de `mul' e `add' seguindo o algoritmo de calculo de endereços em
\emph{row major}. 
O tipo inteiro é passado para o nível acima da árvore sintática, juntamente com
 o apontador para o valor da entrada da tabela de identificadores. 
Caso contrário, é lançado um erro.



Constant : num 

Nesta produção é criada a \emph{string} `pushi' concatenada com o valor numérico
para cima na árvore  sintática. O tipo também é passado para cima, como um tipo
inteiro.



\subsubsection{Fator} 

Factor : Constant              

Nesta produção os valores das instruções geradas são passadas para
\emph{Factor}, herdando assim as propriedades de uma constante (tipo
e instruções geradas).


| Variable

Nesta produção, tal como nas constantes os valores são passados para
\emph{Variable}. Note-se que o tipo de \emph{Variable} é um triplo, ou seja
passa as instruções geradas, o tipo e um apontador para a entrada resolvida no
nível abaixo da árvore sintática.  
 
| '('  '-'Exp')'               

Para a operação unária negativa de números inteiros, é necessário verificar se
o valor da expressão é do tipo inteiro. Caso contrário, há uma erro. A ação
semântica passa por obter da árvore sintática, todas as instruções geradas,
e depois gerar a \emph{string} `pushi -1sub'. O tipo da operação é passado
para o nível superior da árvore sintática.

| '(' Exp ')'                

Nesta produção, tanto os valores de tipo e instruções são passadas de baixo para
cima da árvore sintática.

                         
| NOT '(' Exp ')'            


Para a operação unária negativa de números inteiros, é necessário verificar se
o valor da expressão é do tipo booleano. Caso contrário, há uma erro. A ação
semântica passa por obter da árvore sintática, todas as instruções geradas,
e depois gerar a \emph{string} `not'. O tipo da operação é passado
para o nível superior da árvore sintática.
     


\subsubsection{Termo} 


Term : Factor            

Á semelhança de outras produções, os valores do tipo e instruções são passados
de baixo para cima da árvore sintática. Portanto um termo herda as propriedades
de um fator.

|  Term '*' Factor       

Nesta  produção, em ambos os membros da operação de multiplicação, os tipos são
verificados, ou seja, se ambos são do tipo inteiro. Caso contrário, há um erro.
As instruções que vem de ambos os membros geradas, logo antes de se gerar
a \emph{string} `mul', juntando esta ao restante bloco de instruções. 
O tipo que é passado para o nível acima é um  inteiro.
  
|  Term '/' Factor       

Nesta  produção, em ambos os membros da operação de multiplicação, os tipos são
verificados, ou seja, se ambos são do tipo inteiro. Caso contrário, há um erro.
As instruções que vem de ambos os membros geradas, logo antes de se gerar
a \emph{string} `div', juntando esta ao restante bloco de instruções. 
O tipo que é passado para o nível acima é um  inteiro.
 
|  Term '%' Factor       


Nesta  produção, em ambos os membros da operação de multiplicação, os tipos são
verificados, ou seja, se ambos são do tipo inteiro. Caso contrário, há um erro.
As instruções que vem de ambos os membros geradas, logo antes de se gerar
a \emph{string} `mod', juntando esta ao restante bloco de instruções. 
O tipo que é passado para o nível acima é um inteiro.
|  Term AND Factor       


Nesta  produção, em ambos os membros da operação de multiplicação, os tipos são
verificados, ou seja, se ambos são do tipo inteiro. Caso contrário, há um erro.
As instruções que vem de ambos os membros geradas, logo antes de se gerar
a \emph{string} `mul', juntando esta ao restante bloco de instruções. 
O tipo que é passado para o nível acima é um  booleano.
-> EXP ADITIVA

ExpAdditiv : Term      


Nesta produção os valores das instruções geradas são passadas para
\emph{ExpAdditiv}, herdando assim as propriedades de uma constante (tipo
e instruções geradas).

| ExpAdditiv '+' Term  

Nesta  produção, em ambos os membros da operação de multiplicação, os tipos são
verificados, ou seja, se ambos são do tipo inteiro. Caso contrário, há um erro.
As instruções que vem de ambos os membros geradas, logo antes de se gerar
a \emph{string} `add', juntando esta ao restante bloco de instruções. 
O tipo que é passado para o nível acima é um  inteiro.

| ExpAdditiv '-' Term  

Nesta  produção, em ambos os membros da operação de multiplicação, os tipos são
verificados, ou seja, se ambos são do tipo inteiro. Caso contrário, há um erro.
As instruções que vem de ambos os membros geradas, logo antes de se gerar
a \emph{string} `sub', juntando esta ao restante bloco de instruções. 
O tipo que é passado para o nível acima é um inteiro..

  
| ExpAdditiv OR  Term  

Nesta  produção, em ambos os membros da operação de multiplicação, os tipos são
verificados, ou seja, se ambos são do tipo inteiro. Caso contrário, há um erro.
As instruções que vem de ambos os membros geradas, logo antes de se gerar
a \emph{string} `add', juntando esta ao restante bloco de instruções. 
O tipo que é passado para o nível acima é um  booleano.

\subsubsection{Expressão}

Exp : ExpAdditiv            

Nesta produção, tanto os valores de tipo e instruções são passadas de baixo para
cima da árvore sintática.
 
|  ExpAdditiv L   ExpAdditiv

Nesta  produção, em ambos os membros da operação de multiplicação, os tipos são
verificados, ou seja, se ambos são do tipo inteiro. Caso contrário, há um erro.
As instruções que vem de ambos os membros geradas, logo antes de se gerar
a \emph{string} `inf', juntando esta ao restante bloco de instruções. O tipo que
é passado para o nível acima é um  booleano.

|  ExpAdditiv G   ExpAdditiv
Nesta  produção, em ambos os membros da operação de multiplicação, os tipos são
verificados, ou seja, se ambos são do tipo inteiro. Caso contrário, há um erro.
As instruções que vem de ambos os membros geradas, logo antes de se gerar
a \emph{string} `sup', juntando esta ao restante bloco de instruções. O tipo que
é passado para o nível acima é um  booleano.


  
|  ExpAdditiv GEQ ExpAdditiv
Nesta  produção, em ambos os membros da operação de multiplicação, os tipos são
verificados, ou seja, se ambos são do tipo inteiro. Caso contrário, há um erro.
As instruções que vem de ambos os membros geradas, logo antes de se gerar
a \emph{string} `supeq', juntando esta ao restante bloco de instruções. O tipo que
é passado para o nível acima é um  booleano.


|  ExpAdditiv LEQ ExpAdditiv
Nesta  produção, em ambos os membros da operação de multiplicação, os tipos são
verificados, ou seja, se ambos são do tipo inteiro. Caso contrário, há um erro.
As instruções que vem de ambos os membros geradas, logo antes de se gerar
a \emph{string} `infeq', juntando esta ao restante bloco de instruções. O tipo que
é passado para o nível acima é um  booleano.


|  ExpAdditiv EQ  ExpAdditiv
Nesta  produção, em ambos os membros da operação de multiplicação, os tipos são
verificados, ou seja, se ambos são do tipo inteiro. Caso contrário, há um erro.
As instruções que vem de ambos os membros geradas, logo antes de se gerar
a \emph{string} `equal', juntando esta ao restante bloco de instruções. O tipo que
é passado para o nível acima é um  booleano.

|  ExpAdditiv NEQ ExpAdditiv
Nesta  produção, em ambos os membros da operação de multiplicação, os tipos são
verificados, ou seja, se ambos são do tipo inteiro. Caso contrário, há um erro.
As instruções que vem de ambos os membros geradas, logo antes de se gerar as
\emph{string} com as \emph{strings} `equal' e `not' concatenadas, juntando esta
ao restante bloco de instruções. O tipo que é passado para o nível acima é um
booleano.

\subsubsection{Atribuição}
                                    
Atribution :  Variable '=' ExpAdditiv  

Nesta produção são verificados os tipos, tanto da expressão inteira como da
variável. Se forem ambos do tipo inteiro executam a ação, caso contrário gera-se
uma erro.\ Após a verificação é verificada a categoria da variável, que vem na
informação da entrada por apontador, vinda de baixo da árvore sintática. Se
a categoria for um \emph{array} ou uma matriz, junta ao bloco a \emph{string}
`storen'. Caso contrário  obtém o endereço do valor da entrada e junta ao bloco
a \emph{string} `storen' concatenada com o endereço da variável da \emph{stack}.  



\subsubsection{Instruções}

InstructionsList : Instruction           

Nesta produção, o bloco da instrução é passado para cima da árvore sintática.


| InstructionsList Instruction          

Nesta produção, o bloco da instrução é construído com base do que vem do lado da
mão direita, com o lado da mão esquerda, passando para cima da árvore sintática,
o resultado.



Instruction : Atribution ';'           
Nesta produção, o bloco da atribuição é passado para cima da árvore sintática.

| READ  Variable ';'      

Nesta produção, é obtido de \emph{Variable}, o endereço da \emph{stack} de uma
variável. Após obter de cima da árvore sintática, as instruções geradas a 
\emph{string} `pushg \%d read'  com o endereço
da \emph{stack} virtual é adcionada ao bloco.

| WRITE ExpAdditiv ';'    

Nesta produção é verificado o tipo da expressão inteira. Caso for do tipo
inteiro, executa a ação. Caso contrário, lança um erro.
Após obter de cima da árvore sintática o bloco de intruções anteriores,
concatena a este a \emph{string} `writei'. 

| WRITE string ';'        

A ação nesta produção, é concatenar a \emph{string} `writes' ao bloco.


| IF '('  Exp ')' 

Esta produção tem uma regra intermédia. O tipo da expressão é avaliado se é
booleano. Caso contrário é lançado um erro. A ação é juntar o que vem de
\emph{Exp}, juntar ao que vem de cima, e depois criar a \emph{label} `jz
l1level' concatenada com a \emph{string} que vem da intrução \texttt{add\_label}. 

'{' InstructionsList '}' Else          

No final desta produção, o que vem da \emph{InstructioonList}  e que vem do
\emph{Else} é passado para cima da árvore sintática. 

| WHILE '(' Exp ')' 

Esta produção tem uma regra intermédia. O tipo da expressão é avaliado se é
booleano. Caso contrário é lançado um erro. A ação é 
imprimir a \emph{string} `whileloop' conancatenado com o resultado da função
\texttt{add\_label},  juntar o que vem de \emph{Exp}, juntar ao que vem de cima,
e depois criar a \emph{label} `jz whiledone' concatenada com a \emph{string} que
vem da instrução \texttt{get\_label}. 


'{' InstructionsList '}'  
No final desta produção, obtido o valor atual da \emph{stack} de contadores,
guardado numa variável temporária, sendo removida o \emph{label} da
\emph{string}. Depois é junto o que o vem de \emph{InstructionsList}, mais
o valor guardado da variável temporária, concatenado com \emph{jump whileloop}
e o \emph{whileloop}. Em seguida é removido um nível da \emph{stack}       


| DO '{' InstructionsList '}' WHILE '(' Exp ')' ';' 

Else :         


|            
ELSE '{' InstructionsList '}'          








