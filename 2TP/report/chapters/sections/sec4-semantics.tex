\subsection{Análise Semântica Estática}
\label{subsec:semantica:desenho}

A análise semântica estática complementa a análise lexical e sintática, pois
existem situações em que a apesar de a análise léxica e sintática estarem
corretas, as sequências de símbolos não têm sentido. A título de exemplo, a data
\texttt{2016-45-00} está sintaticamente correta pois segue o formato
\texttt{aaaa-mm-dd}, mas não tem sentido do ponto de vista semântico.

Na \emph{LPIS} existem situações em que a gramática e análise léxica não são
suficientes, nomeadamente na verificação de tipos, ou seja, se estes são
consistentes na sua definição.  Anteriormente, é mencionada a inferência de
tipos em fatores, variáveis, arrays, aditivos, termos, expressões e expressões
relacionais. Assim é necessário não só verificar os elementos de cada operação
binária, como inferir o tipo do seu resultado.  Como já foi mencionado, esta
\emph{LPIS} apenas permite valores inteiros, mas no entanto o resultado das
expressões lógicas e relacionais são do tipo booleano. Por uma questão de
consistência, considera-se que as instruções terão o tipo \texttt{Any}.

Assim, as instruções assumem o tipo \texttt{Any}, variáveis atómicas e arrays
o tipo \texttt{Integer} (exceto subprogramas sem valor de retorno),
e o resultado de uma operação aditiva entre fatores como variáveis e arrays
assumirá o tipo \texttt{Integer}, bem como o tipo de cada de cada membro da
operação binária aditiva. O resultado de uma operação multiplicativa entre
termos (resultantes da operação aditiva) também tomará o tipo \texttt{Integer},
bem como ambos os membros da operação binária em questão. Seguidamente,
o resultado de expressões relacionais deverá ser do tipo \texttt{Boolean}. No
entanto, os membros desta operação binária assumirão o tipo \texttt{Integer}.
Para concluir, o resultado de uma expressão lógica assumirá o tipo
\texttt{Boolean} e ambos os membros da operação binária assumem também o tipo
\texttt{Boolean}.

Para além da verificação de tipos, a análise semântica deve assegurar
a existência de etiquetas, ou \texttt{labels}, de referência no resultado da
geração de código. Dado que a linguagem em causa tem quatro tipos de estruturas
de controlo de fluxo, existirá um mecanismo que cria as \texttt{labels} para
cada tipo de estrutura de controlo e que permita o aninhamento das mesmas. Ou
seja, as etiquetas devem ter a referência do nível em que estão, e devem ser
colocadas no seu devido lugar.  Posteriormente, será descrito o algoritmo
e a implementação do mecanismo de criação de \texttt{labels}.

Adicionalmente, as estruturas de controlo devem ser verificadas para confirmar
a sua devida utilização.  Por exemplo, se existem \texttt{breaks} fora de um
\texttt{loop} ou \texttt{switch}. Na gramática desta linguagem não existem
formas de //..

Finalmente, é necessário considerar outro tipo de análise semântica: a análise
semântica dinâmica.  Ao contrário da anterior, este tipo de análise não
é efetuada em tempo de compilação, mas sim em tempo de execução. Existem
exemplos retirados da gramática da linguagem alvo deste projeto, que não serão
considerados, visto que a máquina virtual VM já os inclui, como é o caso da
divisão por zero, em que os valores retirados da pilha não têm o tipo esperado,
ou acessos indefinidos a uma //? de código. Na especificação léxica da
linguagem, definiu-se que todos os números tomarão valores não negativos,
e previu-se a existência de números negativos na gramática // a uma operação
unária?.  Deste modo não é necessária a reespecificação? semântica estática de
declarações de arrays com tamanho zero, resultados negativos em cálculos de
índices e divisões por zero, uma vez que estes casos fazem parte da análise
semântica dinâmica.


\subsection{Gramática Independente de Contexto}
\label{subsec:gramatica:desenho}

\subsubsection{Axioma}
\label{subsec:subsubsec:axioma:desenho}

\begin{grammar}

<Program> ::= <Declarations> <Body>

<Body> ::= `BEGIN' <InstructionList> `END'

<Declaration> ::= <id>
\alt <id> `[' <num> `]'
\alt <id> `[' <num> `]' `[' <num> `]'

<Declarations> ::= 'VAR' <DeclarationsList> `;' 

<DeclarationsList> ::= <Declaration> 
\alt <DeclarationsList> `,' <Declaration> 

<Term> ::= <id>
\alt <num>
\alt <id> `[' <ExpAdditiv> `]'
\alt <id> `[' <ExpAdditiv> `]' `[' <ExpAdditiv> `]'
\alt `(' <Exp> `)'
\alt `NOT' <Exp>

\end{grammar}


\subsubsection{Declarações de variáveis}
\label{subsec:subsubsec:declaracoes:desenho}


\subsubsection{Expressões}
\label{subsec:subsubsec:expressoes:desenho}


\subsubsection{Instruções}
\label{subsec:subsubsec:instrucoes:desenho}




\section{Analisador léxico}
\label{sec:lexico:desenho}

\subsection{Expressões Regulares}
\label{sec:subsec:ers:desenho}

\begin{minted}{text}
		\@[Ss][Tt][Rr][Ii][Nn][Gg]
		\@[Pp][Rr][Ee][Aa][Mm][Bb][Ll][Ee]
		\@[Cc][Oo][Mm][Mm][Ee][Nn][Tt]
\end{minted}

\subsection{Ficheiro \emph{Flex} com a especificação}
\label{sec:subsection:flex:desenho}


\section{Analisador sintático e tradutor}
\label{sec:sintatico:desenho}


\subsection{Algoritmos}
\label{sec:subsection:algoritmos:desenho}


\subsection{Estruturas de dados}
\label{sec:subsec:estruturas:desenho}

\url{http://lh3lh3.users.sourceforge.net/udb.shtml.}








