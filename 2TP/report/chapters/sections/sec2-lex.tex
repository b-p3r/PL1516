\subsection{Analisador léxico}
\label{subsec:lexico:desenho}


Na fase de construir o analisador léxico, tomou-se em conta a análise anterior
da gramática e definiu-se para cada simbolo terminal uma expressão regular,
e respetiva ação associada.


\subsubsection{Expressões Regulares}
\label{subsec:subsubsec:ers:desenho}

Em primeira instância definiu-se o que era uma letra, um digito e, tudo o que
será para ser rejeitado da seguinte forma:


\begin{minted}{text}
letra [A-Za-z]
digito [0\-9]
lixo \.|\\n
\end{minted}


Em seguida definiram-se as expressões para captura dos símbolos terminais, para
serem usadas como \emph{tokens} no ficheiro \emph{YACC}, definiram-se da
seguinte forma:

\begin{itemize}
	\item Captura de operadores de apenas um caractere; 
\begin{minted}{text}
[-/;,\[\]+\(\)\{\}\%=]  
\end{minted}

Com esta expressão regular são capturados os símbolos referentes às operações
aditivas e multiplicativas, bem como caracteres usados para delimitar expressões
e na construção de estruturas de controlo (chavetas e parêntesis). De igual
modo, são capturados o símbolo unário de um número negativo, delimitadores de
final de instruções (ponto e vírgula), separadores de declarações (vírgula), elementos de declaração
de \emph{arrays} (parêntesis retos) e o sinal de atribuição, que nesta
linguagem é o sinal de igual. 

A ação para esta expressão regular é devolver o valor do código ASCII-estendido
de cada caractere capturado.


	\item Captura de operadores de operações lógicas e relacionais.
	\item 
\begin{minted}{text}
(OR)                      {return OR;}
(<)                       {return L;}
(>)                       {return G;}
(<=)                      {return LEQ;}
(>=)                      {return GEQ;}
(==)                      {return EQ;}
(!=)                      {return NEQ;}
\end{minted}
	\item Captura de palavra reserva para início e fim de programa. 
\begin{minted}{text}
(BEGINNING)               {return BEGINNING;}
(END)                     {return END;}
\end{minted}

A ação será retornar um valor numérico, que será atribuído aquando
o \emph{linking} do ficheiro \emph{Flex} e do ficheiro \emph{YACC}, no
\texttt{yy.tab.c}.    

	\item Captura das palavras reservadas para funções de leitura e escrita. 
\begin{minted}{text}
(READ)                    {return READ;}
(WRITE)                   {return WRITE;}
\end{minted}

Ação a mesma que a anterior.

	\item Captura das palavras reservadas usadas para declarações e estruturas de
		controlo.
\begin{minted}{text}
(VAR)                     {return VAR;}
(WHILE)                   {return WHILE;}
(IF)                      {return IF;}
(ELSE)                    {return ELSE;}
(DO)                      {return DO;}
\end{minted}

Ação: \emph{ibidem}. 

	\item Captura de uma \emph{string}  
  
		\verb|\"[^"]+\"|                 

Neste caso é atribuído ao valor da variável global \texttt{yylval} no elemento
da união \texttt{char * val\_string} uma cópia da \emph{string}, sendo retornado
uma valor que será atribuído pelo \emph{YACC}. 

	\item Captura de um dígito
\begin{minted}{text}
{digito}+                 
\end{minted}

A ação é atribuir ao valor da variável global \texttt{yylval} no elemento
da união \texttt{int val\_nro} o valor do digito capturado., sendo retornado 
uma valor que será atribuído pelo \emph{YACC}. 


	\item Captura de um identificador. 
\begin{minted}{text}
{letra}+                  
\end{minted}
Neste caso é atribuído ao valor da variável global \texttt{yylval} no elemento
da união \texttt{char * val\_id} uma cópia da palavra capturada, sendo retornado
uma valor que será atribuído pelo \emph{YACC}. 

	\item Outros 
\begin{minted}{text}
{lixo}                    {;}
\end{minted}

Ação: ignorar.

\end{itemize}
















