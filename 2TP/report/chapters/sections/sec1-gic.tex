\subsection{Gramática Independente de Contexto}
\label{subsec:gramatica:desenho}

O conjunto dos símbolos terminais da gramática é o que se segue:

T = {id, num, string, BEGINNING, END, VAR, NOT, AND, OR, READ, WRITE, IF,
	WHILE, DO, ELSE,`[', `]', `;', `,', `(', `)', `*', `/', `\%', `{', `}', `+',
`-', `<', `>', `>=', `<=', `==', `!=', `='}


O conjunto dos símbolos não-terminais da gramática é o que se segue:

NT = {Program, Declarations, Body, InstructionsList, Declaration,
DeclarationsList, Factor, ExpAdditiv, Exp, Variable, Term, Atribution,
Instruction, Else, Constant }
\subsubsection{Axioma}
\label{subsec:subsubsec:axioma:desenho}

Nesta linguagem, um programa é composto por declarações e um corpo.

\begin{grammar}
<Program> ::= <Declarations> <Body> 
\end{grammar}

O corpo do programa terá sempre que ter as palavras reservadas \emph{BEGIN},
para iniciar a execução do programa, e \emph{END}, para terminar a execução do
programa. Entre estas duas palavras reservadas estará um conjunto de instruções.  

\begin{grammar}
<Body> ::= `BEGIN' <InstructionList> `END'
\end{grammar}


\subsubsection{Declarações de variáveis}
\label{subsec:subsubsec:declaracoes:desenho}

Assumiu-se que as variáveis seriam todas do tipo inteiro, tendo estas um
identificador, podendo ser variáveis, \emph{arrays} unidimensionais, ou
\emph{arrays} bidimensionais. O tamanho dos \emph{arrays} será sempre um valor
não negativo.   

\begin{grammar}
<Declaration> ::= <id>
\alt <id> `[' <num> `]'
\alt <id> `[' <num> `]' `[' <num> `]' 
\end{grammar}

Uma ou mais declarações formam um conjunto de declarações. Note-se que
é mandatório pelo menos uma declaração.

\begin{grammar}
<DeclarationsList> ::= <Declaration> 
\alt <DeclarationsList> `,' <Declaration> 
\end{grammar}

As declarações devem começar sempre pela palavra reservada \emph{VAR}.

\begin{grammar}
<Declarations> ::= `VAR' <DeclarationsList> `;' 
\end{grammar}


\subsubsection{Expressões}
\label{subsec:subsubsec:expressoes:desenho}

Uma constante é um número não negativo.

\begin{grammar}
<Constant> ::= <num>
\end{grammar}

Uma variável será sempre um identificador, um \emph{array}, com uma expressão
inteira no seu índice, ou índices, se for multidimensional.

\begin{grammar}
<Variable> ::= <id>
\alt <id> `[' <ExpAdditiv> `]'
\alt <id> `[' <ExpAdditiv> `]' `[' <ExpAdditiv> `]' 
\end{grammar} 

Um fator pode ser uma constante, uma variável, uma expressão, uma expressão
negativa, ou a negação de uma expressão.
\begin{grammar}
<Factor> ::= <Constant>
\alt <Variable>
\alt `(' <Exp> `)'
\alt `(' `-' <Exp> `)'
\alt `NOT' <Exp>
\end{grammar}


Um termo será sempre um conjunto de um ou mais fatores, em que as operações que o compõem
serão sempre multiplicativas. Note-se que dado não haver instruções lógicas na
VM, o \emph{AND} terá que ser uma multiplicação entre valores inteiros entre
0 e 1. 

\begin{grammar}
<Term> ::= <Factor>
\alt <Term> `*'  <Factor>
\alt <Term> `/' <Factor>
\alt <Term> `%' <Factor>
\alt <Term> `AND' <Factor>
\end{grammar}


Uma expressão aditiva será sempre um conjunto de um ou mais termos, em que as operações que o compõem
serão sempre aditivas. Note-se que dado não haver instruções lógicas na
VM, o \emph{OR} terá que ser uma soma entre valores inteiros não negativos.

\begin{grammar}
<ExpAdditiv> ::= <Term> 
\alt <ExpAdditiv> `+' <Term>
\alt <ExpAdditiv> `-' <Term> 
\alt <ExpAdditiv> `OR' <Term> 
\end{grammar}

Uma expressão será uma expressão aditiva ou duas expressões aditivas com
determinada relação.

\begin{grammar}
<Exp> ::= <ExpAdditiv>             
\alt  <ExpAdditiv> `>'  <ExpAdditiv> 
\alt  <ExpAdditiv> `<'  <ExpAdditiv> 
\alt  <ExpAdditiv> `>=' <ExpAdditiv> 
\alt  <ExpAdditiv> `<=' <ExpAdditiv> 
\alt  <ExpAdditiv> `==' <ExpAdditiv> 
\alt  <ExpAdditiv> `!=' <ExpAdditiv> 
\end{grammar}


\subsubsection{Instruções}
\label{subsec:subsubsec:instrucoes:desenho}

Uma atribuição será sempre uma variável a tomar o valor de uma expressão
inteira.
\begin{grammar}
<Atribution> ::=  <Variable> `=' <ExpAdditiv> 
\end{grammar}


Uma instrução pode ser uma atribuição, a leitura de uma variável de
\emph{stdin}, a escrita de um valor inteiro ou uma \emph{string} no
\emph{stdout}, ou uma estrutura de controlo com a avaliação de uma expressão
booleana, com um conjunto de instruções associados a cada condição.

\begin{grammar}
<Instruction> ::= <Atribution> `;' 
\alt `READ'  <Variable> `;'
\alt `WRITE' <ExpAdditiv> `;'                      
\alt `WRITE' <string> `;'
\alt `IF' `(' <Exp> `)' `{' <InstructionsList> `}' <Else>
\alt `WHILE `(' <Exp> `)' `{' <InstructionsList> `}' 
\alt `DO'`{' <InstructionsList> `}'`WHILE `(' <Exp> `)' `;' 
\end{grammar}


A bloco seguinte representa a existência ou não de instruções alternativas a um
\emph{if}. 

\begin{grammar}
<Else> ::= <>
\alt `ELSE' '{'<InstructionsList> '}'
\end{grammar}

Uma lista de instruções é um conjunto de uma ou mais instruções. Note-se que
a existência de instruções é mandatória, isto é, um programa ou uma estrutura de
controlo terá sempre pelo menos uma instrução.

\begin{grammar}
<InstructionList> ::= <Instruction>
\alt <InstructionList> <Instruction>  
\end{grammar}











