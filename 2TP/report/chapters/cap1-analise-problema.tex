\chapter{Análise do Problema}
\label{cap:analise}


\section{Especificação dos requisitos}
\label{sec:especificacao:analise}
%Especificação dos requisitos (ver enunciado)

O desafio deste projeto consiste na criação de uma linguagem de programação
imperativa simples (LPIS) e respetivo compilador.  Para tal é necessário criar
uma gramática em \emph{Bauk-Naur Form}--BNF--, definir símbolos terminais e não
terminais, e desenvolver o analisador léxico, seguido do desenvolvimento do
analisador sintático, com base nas regras da gramática, tendo, de igual modo, em conta
a análise léxica. O compilador da linguagem irá incorporar ambas as análises
supramencionadas, e procederá a uma análise de ações semânticas e á geração do
código.  O código gerado será pseudo-código da maquina virtual VM, o analisador
léxico será elaborado no \emph{Flex}, e o YACC será usado para a geração do
código e análises sintática e semântica.   


\section{Pedidos}

Para a linguagem alvo deste projeto, pede-se que, no mínimo, permita:


\begin{itemize}
\item Declaração e manuseamento de variáveis atómicas do tipo inteiro que
	permitam a realização de operações aritméticas, relacionais e lógicas;
\item Declaração e manuseamento variáveis estruturadas do tipo \emph{array} de
	inteiros, com 1 ou 2 dimensões, que permitam apenas operações de indexação;
\item Realização de instruções algorítmicas básicas como a atribuição de
	expressões a variáveis;
\item Leitura do \emph{stdin} e escrita no \emph{stdout};
\item Realização de instruções de controlo do fluxo de execução que permitam
	aninhamento;
\end{itemize}

Opcionalmente, a linguagem definida deve ser capaz de definir e invocar
subprogramas sem parâmetros, mas que possam retornar um resultado atómico.

Pede-se também que, qualquer variável não pode ser redeclarada.

\section{Dados}
\label{sec:dados:analise}

Uma linguagem imperativa completa deve de permitir pelo menos duas estruturas de controlo:

\begin{itemize}
	\item Estruturas de fluxo condicional (\emph{if then else});
	\item Estruturas cíclicas (\emph{while});
\end{itemize}

Por uma questão de simplificação do código, poderá adicionalmente permitir
a estrutura \emph{if then} e a estrutura \emph{do while}.

De igual modo, permitindo a linguagem acesso a estruturas do tipo \emph{array},
é necessário ter em conta que qualquer \emph{array}, independentemente da dimensão,
é representado em memória como um único \emph{array} de uma dimensão. Contudo,
é necessário estabelecer regras para o acesso a \emph{\emph{array}s} uni e bidimensionais. Para
garantir a eficácia da linguagem, considerar-se-á que o acesso será feito em
\emph{row major} para \emph{\emph{array}s} bidimensionais.

Genericamente, o acesso a um \emph{array} pode ser representado por a fórmula $A[i]
= b + w * (i - lb)$, sendo:

\begin{itemize}
	\item \texttt{b} o endereço base;
\item \texttt{w} o tamanho do elemento;
\item \texttt{i} o índice do elemento;
\item \texttt{lb} o limite inferior na memória;
\end{itemize}

No caso do acesso em questão, $w = 1$ e $lb = 0$, logo $A[i] = b + i$.

Para um \emph{array} bidimensional temos $A[i][j] = b + w [N(i - lr) + (j
- lc)]$, onde:

\begin{itemize}
	\item \texttt{b} é o endereço base;
\item \texttt{i} é o índice da linha do elemento;
\item \texttt{j} é o índice da coluna do elemento;
\item \texttt{w} é o tamanho do elemento em bytes;
\item \texttt{lr} é o limite inferior da linha;
\item \texttt{lc} é o limite inferior da coluna;
\item \texttt{N} é o número máximo de linhas;
\end{itemize}

Assumindo $w = 1$ e $lr = lc = 0$, temos $A[i][j] = b + N*i + j$.

\section{Relações}
\label{sec:relacoes:analise}

Para o cálculo das expressões é necessário ter em conta algumas propriedades de
cada tipo de expressão. Ou seja, é necessário verificar os tipos atómicos
(variáveis, constantes, elementos de \emph{arrays}), que assumimos como
inteiros, e os resultados das expressões por inferência. Assim:
\begin{itemize}
\item Para as expressões aritméticas (soma, subtração, multiplicação, divisão
	inteira e módulo), bem como os elementos constituintes da expressão, o tipo
	deverá ser um inteiro;
\item Para as expressões relacionais (maior, menor, maior ou igual, menor ou
	igual, igual e diferente), os elementos da expressão deverão ser do tipo
	inteiro e o resultado um valor booleano;
\item Para as expressões lógicas, os elementos deverão ser booleanos
	e o resultado deverá também ser booleano;
\end{itemize}

De notar que várias expressões podem ser compostas, cuja verificação estará
descrita na análise semântica. Adicionalmente, existe uma relação de precedência
das operações, bem como de fatores. Um fator é uma expressão aninhada, um // ou
uma variável. De igual modo, subprogramas e operações unárias? (uma vez que são
funções) fazem parte deste conjunto. Consequentemente, um fator é prioritário
em relação a todas as operações, uma vez que é o elemento atómico de uma
expressão.

Em seguida, temos os termos, que são compostos por operações multiplicativas
entre fatores, sendo estas a multiplicação, a divisão inteira, e módulo. Poderá
também ser incluída a operação \emph{E LÓGICO}, por razões que
posteriormente serão explicadas. As expressões seguintes na escala de
prioridades são as expressões aditivas (soma e subtração). Poderá ser incluída
a operação \emph{OU LÓGICO}, por razões que tal como a inclusão de
\emph{E LÓGICO}, que serão posteriormente explicitadas. As expressões de menor
prioridade são as expressões relacionais (\emph{menor}, \emph{maior},
\emph{menor ou igual}, \emph{maior ou igual}, \emph{igualdade}
e \emph{desigualdade}).

Para a execução do programa é necessário definir as instruções e operações que
o definem. O programa pode efetuar cálculos utilizando as expressões previamente
mencionadas. No entanto, é necessário guardar o resultado. Nas linguagem
imperativas, a interação base é a atribuição, que pode ser atribuição do valor
de uma expressão a uma variável. Esta poderá ser uma variável atómica ou
a posição de um \emph{array}.  As restantes instruções terão por base estes
cálculos de expressões e atribuições, usando restrições de controlo de fluxo,
tais como \emph{if then else}, \emph{while} e \emph{do while}.

Devido à necessidade de armazenar o valor de expressões, torna-se necessário 
declarar as variáveis, alocando espaço em memória para as mesmas. As declarações 
podem designar o nome da variável, o seu tipo, e o seu valor, não // redeclarações 
da mesma variável. Neste projeto, considerar-se-á que todas as variáveis são globais, 
ou seja, declaradas antes da execução do programa, e assumem número inteiro.   

Para subprogramas, há que ter em conta que podem ter ou não parâmetros,
e devolver ou não um valor. Visto que a alocação de memória é efetuada numa
\emph{stack} virtual, é necessário, no momento da sua //, criar uma \emph{frame}
no topo da \emph{stack} com todas as variáveis locais declaradas, e parâmetros
alocados em memória. Para identificar o programa principal, a linguagem usa um
sistema de níveis, em que um nível assume o valor 0 para o programa principal ou
o valor 1 para subprogramas. Finalmente, o início e fim de cada programa ou
subprograma deve estar devidamente assinalado.  




