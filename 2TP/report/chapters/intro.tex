\chapter*{Introdução}
\addcontentsline{toc}{chapter}{Introdução} 
\label{intro}

O presente relatório tem como objetivo documentar o processo de desenvolvimento de 
uma linguagem de programação imperativa simples, ou LPIS, e respetivo compilador, que
deve ser capaz de gerar pseudo-código Assembly da máquina virtual VM, utilizada neste 
projeto. Para este fim, é necessário criar uma gramática independente de contexto que 
defina a linguagem, e establecer as regras de tradução para o Assembly da VM. 
 

\section*{Metas e objetivos} 

Este projeto prentende aumentar a experiência nos campos da engenharia de linguagens e 
programação generativa, através do desenvolvimento de linguagens e a utilização de geradores 
de compiladores baseados em gramáicas tradutores, o Yacc neste caso.
 
Adicionalmente, este projeto tem como objetivos secundários aumentar a capacidade de 
desenvolvimento de gramáticas indepenentes de contexto, e melhorar o uso do ambiente 
Linux e da linguagem imperativa C.

\section*{Estrutura do Relatório} 

O relatório está dividido em três capítulos, correspondentes á análise do problema, á 
implementação da solução, e ao resultado dos testes efetuados, por esta ordem. 
No primeiro capítulo, \emph{Análise do Problema}, são expostos os requisitos do problema 
apresentado, e discutidas as estratégias utilizadas para a consequente implementação da
solução do problema. O capítulo dois, \emph{Desenho e Implementação da Solução}, apresenta 
a gramática definida para a LPIS e explicita o funcionamento da análise léxica, sintática 
e semântica da linguagem criada. Este capítulo termina com a especificação das estruturas 
de dados e algoritmos usados na implementação final da solução.
Por fim, o capítulo \emph{Testes e Resultados} expõe o resultado dos testes requiridos á
demonstração do funcionamento da linguagem e compilador desenvolvidos durante o projeto. 



