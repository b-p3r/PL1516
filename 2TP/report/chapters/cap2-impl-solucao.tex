\chapter{Desenho e implementação da solução}
\label{cap:desenho}

\section{Análise semântica estática}
\label{sec:semantica:desenho}

A análise semântica estática complementa a análise lexical e sintática, pois existem 
situações em que a apesar de a análise léxica e sintática estarem corretas, as
sequências de símbolos não têm sentido. A título de exemplo, a data 2016-45-00 está
sintáticamente correcta pois segue o formato aaaa-mm-dd, mas não tem sentido do ponto
de vista semântico.\\

Na LPIS existem situações em que a gramática e análise léxica não são suficientes, 
nomeadamente na verificação de tipos, ou seja, se estes são consistentes na sua definição.
Anteriormente, é mencionada a inferência de tipos em factores, variáveis, arrays, aditivos,
termos, expressões e expressões relacionais. Assim é necessário não só verificar os elementos
de cada operação binária, como inferir o tipo do seu resultado. 
Como já foi mencionado, esta LPIS apenas permite valores inteiros, mas no entanto o resultado
das expressões lógicas e relacionais são do tipo booleano. Por uma questão de consistência, 
considera-se que as instruções terão o tipo "Any".\\

Assim, as instruções assumem o tipo "Any", variáveis atómicas e arrays o tipo "Integer" 
(excepto subprogramas sem valor de retorno), e o resultado de uma operação aditiva
entre factores como variáveis e arrays assumirá o tipo "Integer", bem como o tipo de cada
de cada membro da operação binária aditiva. O resultado de uma operação multiplicativa entre
termos (resultantes da operação aditiva) também tomará o tipo "Integer", bem como ambos
os membros da operação binária em questão. Seguidamente, o resultado de expressões relacionais
deverá ser do tipo "Boolean". No entanto, os membros desta operação binária assumirão o tipo
"Integer". Para concluir, o resultado de uma expressão lógica assumirá o tipo "Boolean" e 
ambos os membros da operação binária assumem também o tipo "Boolean".\\

Para além da verificação de tipos, a análise semântica deve assegurar a existência de etiquetas,
ou "labels", de referência no resultado da geração de código. Dado que a linguagem em causa tem
quatro tipos de estruturas de controlo de fluxo, existiráum mecanismo que cria as "labels" 
para cada tipo de estrutura de controlo e que permita o aninhamento das mesmas. Ou seja, as 
etiquetas devem ter a referência do nível em que estão, e devem ser colocadas no seu devido lugar.
Posteriormente, será descrito o algoritmo e a implementação do mecanismo de criação de "labels".\\

Adicionalmente, as estruturas de controlo devem ser verificadas para confirmar a sua devida utilização.
Por exemplo, se existem "breaks" fora de um "loop" ou "switch". Na gramática desta linguagem não existem
formas de //...\\

Finalmente, é necessário considerar outro tipo de análise semântica: a análise semântica dinâmica.
Ao contrário da anterior, este tipo de análise não é efetuada em tempo de compilação, mas sim
em tempo de execução. Existem exemplos retirados da gramática da linguagem alvo deste projeto, que 
não serão considerados, visto que a máquina virtual VM já os inclui, como é o caso da divisão por
zero, em que os valores retirados da pilha não têm o tipo esperado, ou acessos indefinidos a uma
//? de código. Na especificação léxica da linguagem, definiu-se que todos os números tomarão valores
não negativos, e previu-se a existência de números negativos na gramática // a uma operação unária?. 
Deste modo não é necessária a reespecificação? semântica estática de declarações de arrays com
tamanho zero, resultados negativos em cálculos de índices e divisões por zero, uma vez que estes 
casos fazem parte da análise semântica dinâmica.\\


\subsection{Gramática Independente de Contexto}
\label{subsec:gramatica:desenho}

O conjunto dos símbolos terminais da gramática é o que se segue:

T = {id, num, string, BEGINNING, END, VAR, NOT, AND, OR, READ, WRITE, IF,
	WHILE, DO, ELSE,`[', `]', `;', `,', `(', `)', `*', `/', `\%', `{', `}', `+',
`-', `<', `>', `>=', `<=', `==', `!=', `='}


O conjunto dos símbolos não-terminais da gramática é o que se segue:

NT = {Program, Declarations, Body, InstructionsList, Declaration,
DeclarationsList, Factor, ExpAdditiv, Exp, Variable, Term, Atribution,
Instruction, Else, Constant }
\subsubsection{Axioma}
\label{subsec:subsubsec:axioma:desenho}

Nesta linguagem, um programa é composto por declarações e um corpo.

\begin{grammar}
<Program> ::= <Declarations> <Body> 
\end{grammar}

O corpo do programa terá sempre que ter as palavras reservadas \emph{BEGIN},
para iniciar a execução do programa, e \emph{END}, para terminar a execução do
programa. Entre estas duas palavras reservadas estará um conjunto de instruções.  

\begin{grammar}
<Body> ::= `BEGIN' <InstructionList> `END'
\end{grammar}


\subsubsection{Declarações de variáveis}
\label{subsec:subsubsec:declaracoes:desenho}

Assumiu-se que as variáveis seriam todas do tipo inteiro, tendo estas um
identificador, podendo ser variáveis, \emph{arrays} unidimensionais, ou
\emph{arrays} bidimensionais. O tamanho dos \emph{arrays} será sempre um valor
não negativo.   

\begin{grammar}
<Declaration> ::= <id>
\alt <id> `[' <num> `]'
\alt <id> `[' <num> `]' `[' <num> `]' 
\end{grammar}

Uma ou mais declarações formam um conjunto de declarações. Note-se que
é mandatório pelo menos uma declaração.

\begin{grammar}
<DeclarationsList> ::= <Declaration> 
\alt <DeclarationsList> `,' <Declaration> 
\end{grammar}

As declarações devem começar sempre pela palavra reservada \emph{VAR}.

\begin{grammar}
<Declarations> ::= `VAR' <DeclarationsList> `;' 
\end{grammar}


\subsubsection{Expressões}
\label{subsec:subsubsec:expressoes:desenho}

Uma constante é um número não negativo.

\begin{grammar}
<Constant> ::= <num>
\end{grammar}

Uma variável será sempre um identificador, um \emph{array}, com uma expressão
inteira no seu índice, ou índices, se for multidimensional.

\begin{grammar}
<Variable> ::= <id>
\alt <id> `[' <ExpAdditiv> `]'
\alt <id> `[' <ExpAdditiv> `]' `[' <ExpAdditiv> `]' 
\end{grammar} 

Um fator pode ser uma constante, uma variável, uma expressão, uma expressão
negativa, ou a negação de uma expressão.
\begin{grammar}
<Factor> ::= <Constant>
\alt <Variable>
\alt `(' <Exp> `)'
\alt `(' `-' <Exp> `)'
\alt `NOT' <Exp>
\end{grammar}


Um termo será sempre um conjunto de um ou mais fatores, em que as operações que o compõem
serão sempre multiplicativas. Note-se que dado não haver instruções lógicas na
VM, o \emph{AND} terá que ser uma multiplicação entre valores inteiros entre
0 e 1. 

\begin{grammar}
<Term> ::= <Factor>
\alt <Term> `*'  <Factor>
\alt <Term> `/' <Factor>
\alt <Term> `%' <Factor>
\alt <Term> `AND' <Factor>
\end{grammar}


Uma expressão aditiva será sempre um conjunto de um ou mais termos, em que as operações que o compõem
serão sempre aditivas. Note-se que dado não haver instruções lógicas na
VM, o \emph{OR} terá que ser uma soma entre valores inteiros não negativos.

\begin{grammar}
<ExpAdditiv> ::= <Term> 
\alt <ExpAdditiv> `+' <Term>
\alt <ExpAdditiv> `-' <Term> 
\alt <ExpAdditiv> `OR' <Term> 
\end{grammar}

Uma expressão será uma expressão aditiva ou duas expressões aditivas com
determinada relação.

\begin{grammar}
<Exp> ::= <ExpAdditiv>             
\alt  <ExpAdditiv> `>'  <ExpAdditiv> 
\alt  <ExpAdditiv> `<'  <ExpAdditiv> 
\alt  <ExpAdditiv> `>=' <ExpAdditiv> 
\alt  <ExpAdditiv> `<=' <ExpAdditiv> 
\alt  <ExpAdditiv> `==' <ExpAdditiv> 
\alt  <ExpAdditiv> `!=' <ExpAdditiv> 
\end{grammar}


\subsubsection{Instruções}
\label{subsec:subsubsec:instrucoes:desenho}

Uma atribuição será sempre uma variável a tomar o valor de uma expressão
inteira.
\begin{grammar}
<Atribution> ::=  <Variable> `=' <ExpAdditiv> 
\end{grammar}


Uma instrução pode ser uma atribuição, a leitura de uma variável de
\emph{stdin}, a escrita de um valor inteiro ou uma \emph{string} no
\emph{stdout}, ou uma estrutura de controlo com a avaliação de uma expressão
booleana, com um conjunto de instruções associados a cada condição.

\begin{grammar}
<Instruction> ::= <Atribution> `;' 
\alt `READ'  <Variable> `;'
\alt `WRITE' <ExpAdditiv> `;'                      
\alt `WRITE' <string> `;'
\alt `IF' `(' <Exp> `)' `{' <InstructionsList> `}' <Else>
\alt `WHILE `(' <Exp> `)' `{' <InstructionsList> `}' 
\alt `DO'`{' <InstructionsList> `}'`WHILE `(' <Exp> `)' `;' 
\end{grammar}


A bloco seguinte representa a existência ou não de instruções alternativas a um
\emph{if}. 

\begin{grammar}
<Else> ::= <>
\alt `ELSE' '{'<InstructionsList> '}'
\end{grammar}

Uma lista de instruções é um conjunto de uma ou mais instruções. Note-se que
a existência de instruções é mandatória, isto é, um programa ou uma estrutura de
controlo terá sempre pelo menos uma instrução.

\begin{grammar}
<InstructionList> ::= <Instruction>
\alt <InstructionList> <Instruction>  
\end{grammar}












\subsection{Analisador léxico}
\label{subsec:lexico:desenho}

\subsubsection{Expressões Regulares}
\label{subsec:subsubsec:ers:desenho}

\begin{minted}{text}
		\@[Ss][Tt][Rr][Ii][Nn][Gg]
		\@[Pp][Rr][Ee][Aa][Mm][Bb][Ll][Ee]
		\@[Cc][Oo][Mm][Mm][Ee][Nn][Tt]
\end{minted}

\subsubsection{Ficheiro \emph{Flex} com a especificação}
\label{subsec:subsubsection:flex:desenho}


\subsection{Analisador sintático e tradutor}
\label{subsec:sintatico:desenho}

Para o analisador sintático, criaram-se algumas estruturas de suporte ao
\emph{parser}, nomeadamente uma \emph{hashtable} para a tabela de
identificadores, uma biblioteca para os dados referentes a cada identificador,
uma biblioteca para suporte ao \emph{parser}, com toda a informação relativa ao
estado do \emph{parser}-- apontador de endereços na \emph{stack} virtual,
\emph{stacks} para calculo do nível das \emph{etiquetas}-- e, adicionalmente
definiram-se tipos enumerados para serem usados em transversalmente na
aplicação.





\subsubsection{Estruturas de dados}
\label{subsec:subsubsec:estruturas:desenho}

A biblioteca \texttt{entry} possui uma estrutura composta pelos campos
mencionados em secções anteriores: tipo, classe, nível, que são tipos
enumerados, e, endereço base, número de linhas máximo (caso seja uma matriz
) e tamanho máximo (para \emph{arrays} unidimensionais e bidimensionais).
Adicionalmente, considerou-se a criação de uma lista de argumentos, no entanto,
por razões que serão posteriormente explicitadas, decidiu-se não se incluir.

As funções referentes à esta biblioteca, inicializam e desalocam memória e vão
buscar os dados da estrutura, ou atualizam os dados desta estrutura. Para
facilitar a criação de entradas na tabela, com diferentes tipos de classes
(\emph{array}, \emph{matriz} e variável), criaram-se métodos que providenciam
a criação da entrada por classe.


A biblioteca \texttt{program\_status} que guarda informação sobre
o \emph{parsing} tem o formato que se segue;  

\begin{verbatim}
typedef struct stat
{
    char label              [MAX_CONDITION_ROW] [ MAX_LABEL ];
    int  label_stack        [MAX_CONDITION_ROW] [ MAX_LABEL_STACK ];
    int  label_number_size  [MAX_CONDITION_ROW] [ MAX_LABEL_STACK ];
    int spointer            [MAX_CONDITION_ROW] [1];
    int strpointer          [MAX_CONDITION_ROW] [1];
    int size_label_string   [MAX_CONDITION_ROW] [1];
    int addresspointer;
} Program_status;
\end{verbatim}

Basicamente, a estrutura possui \emph{arrays} bidimensionais para
a representação das \emph{stacks} das \emph{labels}. Estas têm 4 linhas, ou
seja, uma linha para cada tipo de estrutura de controlo. Deste modo, a variável
\texttt{label} irá guardar a concatenação dos valores dos níveis de aninhamento,
a variável \texttt{label\_stack} possuirá os contadores para cada nível de
aninhamento e a variável \texttt{label\_number\_size} irá guardar o tamanho da
da \emph{string} resultante do valor de cada nível concatenado. A biblioteca
possui funções comuns às \emph{stacks}, como \texttt{pop}, \texttt{push},
\texttt{top} e, adicionalmente possui funções específicas para a manipulação da
informação das \emph{stacks}, que usa o tipo enumerado
\texttt{CompoundInstruction} para aceder por índice às \emph{stacks}.   

As funções específicas para o cálculo das etiquetas são:

\begin{itemize}
	\item \verb|reset_label_stack|
		Esta função tem por âmbito reinicializar o contador após sair de uma ação 
		semântica, como se poderá ver na secção sobre algoritmos, na posição em 
		\texttt{stack[ stack\_pointer ]}. Note-se que o \texttt{top} é em \texttt{stack
		[stack\_pointer--1 ]} 

  \item \verb|increment_top_label_stack|
		Esta função incrementa o valor do contador na posição \texttt{stack\_pointer
		-1} da \emph{stack} de contadores, ou seja, incrementa uma nova ocorrência
		no mesmo nível.

  \item \verb|char *get_label|              

		Esta função obtém a \emph{string} criada até ao momento, com os valores das
		ocorrências dos níveis concatenados na \emph{string} da \emph{label}.  

  \item \verb|char *push_label|             

		A função \texttt{push\-label} incrementa o valor do \texttt{top} da
		\emph{stack} de contadores,   converte o valor numérico do \texttt{top} para
		uma \emph{string}, calcula o tamanho desta \emph{string} e guarda na
		\emph{stack} de tamanhos das \emph{strings}. Em seguida, concatena
		a \emph{string} convertida `a \emph{string} em construção, sendo esta
		copiada para ser retornada pela função. Adicionalmente, guarda o 
		tamanho da \emph{string} convertida na devida \emph{stack} e avança
		o apontador da \emph{string} em construção por esse tamanho.  

  \item \verb|pop_label|                    

		Obtém o tamanho da \emph{string} concatenada, guardada anteriormente na
		\emph{stack} \texttt{label\_number\_size} e coloca caracteres nulos na pilha
		com a \emph{string} para as \emph{labels}, decrementando o apontador da
		pinha com a \emph{string} no valor desse tamanho. 

\end{itemize}


Note-se que, a descrição desta funções é fundamental para a compreensão do
código, uma vez que serão utilizadas mais adiante no documento.

De igual modo, possui funções para o cálculo de endereços para cada classe de
variável, onde o apontador para a \emph{stack} virtual é incrementado pelo
tamanho da variável.\ Outras funções são a inserção de um identificador com os
seu devidos valores na tabela de identificadores, bem como a procura de um
identificador e remoção de todos os identificadores.\ Por último, a função
\texttt{check\_type} compara dois tipos.\  

Além destas bibliotecas, foram definidos novos tipos, enumerados,  no cabeçalho
\texttt{types.h}, que são os seguintes:


\begin{itemize}
	\item \verb|CompoundInstruction|;
Este tipo carateriza tipos de estruturas de controlo diferentes para acesso por
índice aos \emph{arrays} multidimensionais, representam as diferentes
\emph{stacks} para as \emph{labels}.   

		\begin{itemize}
			\item \texttt{if\_inst}
			\item \texttt{else\_inst}
			\item \texttt{while\_inst}
			\item \texttt{do\_while\_inst}
		\end{itemize}
	\item \verb|Class|;a
		O tipo \texttt{Class} serve para diferenciar as categorias, ou classes, de
		objetos que podem ser declarados e invocados.
		\begin{itemize}
			\item \texttt{Variable}
			\item \texttt{Array}
			\item \texttt{Matrix}
			\item \texttt{Function}
			\item \texttt{Procedure}
			\item \texttt{Nothing}
		\end{itemize}
	\item \verb|Level|;
		O nível tem o propósito de diferenciar programas e subprogramas.
		\begin{itemize}
			\item \texttt{Program}
			\item \texttt{Subprogram}
		\end{itemize}
	\item \verb|Type|;
		Os tipos das variáveis e expressões podem ser os seguintes:
		\begin{itemize}
			\item \texttt{Any}
			\item \texttt{Integer}
			\item \texttt{Boolean}
				Embora se possa atribuir o tipo \texttt{Any} a instruções, não existe
				relevância para o fazer. 
		\end{itemize}
\end{itemize}


\subsubsection{Algoritmos}
\label{subsec:subsubsection:algoritmos:desenho}

\begin{figure}[<+htpb+>]
	\centering
	\psfig{figure=}
	\caption{}
	\label{fig:}
\end{figure}


\begin{figure}[<+htpb+>]
	\centering
	\psfig{figure=}
	\caption{}
	\label{fig:}
\end{figure}

\begin{figure}[<+htpb+>]
	\centering
	\psfig{figure=}
	\caption{}
	\label{fig:}
\end{figure}

\begin{figure}[<+htpb+>]
	\centering
	\psfig{figure=}
	\caption{}
	\label{fig:}
\end{figure}



\url{http://lh3lh3.users.sourceforge.net/udb.shtml.}








